\begin{enumerate}
\item Suppose that $f: \R^2 \to \R^2$ is linear.  
      Prove there exists a matrix $\BA\in\R^{2\times 2}$ such that
      $f$ is given by $f(\Bu) = \BA\Bu$.  
      Hint: Each $\Bu  = \left[\!\!\begin{array}{c} u_1 \\ u_2\end{array}\!\!\right] \in \R^2$
            can be written as $\Bu = u_1 \Be_1 + u_2 \Be_2$, where
                 \[ \Be_1 = \left[\!\begin{array}{c} 1 \\ 0\end{array}\!\right], \qquad
                    \Be_2 = \left[\!\begin{array}{c} 0 \\ 1\end{array}\!\right]. \] 
      Since $f$ is linear, we have $f(\Bu) =  u_1 f(\Be_1) + u_2 f(\Be_2)$.
      Your formula for the matrix $\BA$ may include the vectors $f(\Be_1)$ and $f(\Be_2)$.

\item Now we want to generalize the result in part~(a): Show that if 
      $f: \R^n \to \R^m$ is linear, then there exists a matrix $\BA\in\Rmn$
      such that $f(\Bu) = \BA\Bu$ for all $\Bu\in\Rn$.\\[0.5em]
      (Thus any linear function that maps $\R^n$ to $\R^m$ can be written 
        as a matrix-vector product.)
\end{enumerate}

%%%%%%%%%%%%%%%%%%%%%%%%%%%%%%%%%%%%%%%%%%%%%%%%%%%%%%%%%%%%%%%%%%%%%%%%%%%%%%%%

\ifthenelse{\boolean{showsols}}{
\begin{solution}
\begin{enumerate}
\item Write $\Bu\in\R^2$ in the form
         \[   
         \Bu = \left[\begin{array}{cc}u_1 & u_2\end{array}\right]. 
         \]
      Any matrix $\BA\in\R^{2\times 2}$ can be written as
          \[ \BA = \left[\begin{array}{cc} \Ba_1 & \Ba_2\end{array}\right],\]
      where $\Ba_1, \Ba_2\in\R^2$ are the columns of $\BA$.
      Now the matrix-vector product $\BA\Bu$ is a linear combination
      of the columns of $\BA$:
          $$ \BA\Bu = \left[\begin{array}{cc} \Ba_1 & \Ba_2\end{array}\right] \left[\begin{array}{c}u_1\\ u_2\end{array}\right]
                    = u_1 \Ba_1 + u_2 \Ba_2. \eqno{(*)} $$
      We are trying to find a formula for $\BA$ (or, equivalently, for
      $\Ba_1$ and $\Ba_2$) such that $f(\Bu) = \BA\Bu$.
      Using the hint, we have
          $$ f(\Bu) = u_1 f(\Be_1) + u_2 f(\Be_2). \eqno{(**)} $$
      Comparing $(*)$ and $(**)$  and equating like terms, we see that
          \[ \Ba_1 = f(\Be_1), \qquad \Ba_2 = f(\Be_2),\]
      and hence
          \[ \BA = \left[\begin{array}{cc}f(\Be_1) & f(\Be_2)\end{array}\right].\]

\item We follow the same idea as in part~(a). Write $\Bu\in\Rn$ as
           \[ \Bu = 
           \left[\begin{array}{c} u_1 \\ \vdots \\ u_n\end{array}\right]
           \]
      and $\BA\in\Rmn$ by column,
          \[ \BA = \left[\begin{array}{cccc} \Ba_1 & \Ba_2 & \cdots & \Ba_n\end{array}\right],\]
      where $\Ba_1, \ldots, \Ba_n \in \Rm$.  
      Equating like terms in
          $$ \BA\Bu = \left[\begin{array}{cccc} \Ba_1 & \Ba_2 & \cdots & \Ba_n\end{array}\right] 
           \left[\begin{array}{c} u_1\\ u_2\\ \vdots \\ u_n\end{array}\right]
                    = u_1 \Ba_1 + u_2 \Ba_2 + \cdots u_n \Ba_n$$
      and
          $$ f(\Bu) = u_1 f(\Be_1) + u_2 f(\Be_2) + \cdots u_n f(\Be_n),$$
      we arrive at
          \[ \BA = \left[\begin{array}{cccc}f(\Be_1) & f(\Be_2) & \cdots & f(\Be_n)\end{array}\right].
          \]

      \end{enumerate}
\end{solution}}{}
