Our model of the vibrating string predicts that motion induced by an 
initial pluck will propagate forever with no loss of energy.
In practice we know this is not the case: a string 
eventually slows down due to various types of \emph{damping}.
For example, \emph{viscous damping}, a model of air resistance,
acts in proportion to the velocity of the string.
The partial differential equation becomes
\[ u_{tt}(x,t) = u_{xx}(x,t) - 2@d @u_t(x,t),\]
where $d> 0$ controls the strength of the damping.
Impose homogeneous Dirichlet boundary conditions,
\[ u(0,t) = u(1,t) = 0\]
and suppose we know the initial position and velocity of the pluck:
\[ u(x,0) = u_0(x), \qquad u_t(x,0) = v_0(x).\]
In our previous language, we write this PDE in the form 
\[ u_{tt} = -L u - 2@d@u_t,\]
where the operator $L$ is defined as $Lu = -u_{xx}$ with
boundary conditions $u(0)=u(1)=0$; as you know well by now,
this operator has eigenvalues $\lambda_k = k^2\pi^2$
and eigenfunctions $\psi_k(x) = \sqrt{2} \sin(k \pi x)$.
We will look for solutions to the PDE of the form
\[ u(x,t) = \sum_{k=1}^\infty a_k(t) \psi_k(x).\]
For simplicity, assume that $d\in(0,\pi)$.

\begin{enumerate}
\item From the differential equation and this form for $u(x,t)$,
      show that the coefficients $a_k(t)$ must satisfy the
      ordinary differential equation
\[ a_k''(t) = -\lambda_k a_k(t) - 2 d a_k'(t).\]

\item Show that the following function satisfies the differential equation in part~(a):
\[ a_k(t) = C_1 \exp\!\big(\big(-\!d+\sqrt{d^2-k^2\pi^2}\big)t\big)
            + C_2 \exp\!\big(\big(-\!d-\sqrt{d^2-k^2\pi^2}\big)t\big)\]
      for arbitrary constants $C_1$ and $C_2$.  (Don't fret about the fact that we
      have square roots of negative numbers; proceed in the same way you 
      would for an exponential with real argument.)

\item Now assume that the string starts with zero displacement
      ($u_0(x) = 0$) but some velocity 
     \[ v_0(x) = \sum_{k=1}^\infty b_k(0)  \psi(x).\]
      Determine the values of the constants $C_1$ and $C_2$ in part~(b)
      for these initial conditions. 

\item Suppose we have $u_0(x)=0$ and initial velocity 
      $v_0(x) = x \sin(3\pi x)$, for which
      \[ b_k(0) = {-6@k@\sqrt{2} (1+(-1)^k) \over (k^2 -9)^2 \pi^2} \mbox{\ \ for $k\ne 3$}, 
    \qquad b_3(0) = {\sqrt{2}\over 4}.\]

      Take damping parameter $d=1$, and plot the solution $u(x,t)$ 
      (using 20 terms in the series) at times $t = 0.15, 0.3, 0.6, 1.2, 2.4$.  
      (You may superimpose these on one well-labeled plot; for clarity,
      set the vertical scale to $[-0.1,0.1]$.)

\item Take the same values of $u_0$ and $v_0$ used in part~(d).
      Plot the solution at time $t=2.5$ for $d = 0, .5, 1, 3$
      on one well-labeled plot, again using vertical scale $[-0.1,0.1]$.
      How does the solution depend on the damping parameter $d$?
\end{enumerate}

%%%%%%%%%%%%%%%%%%%%%%%%%%%%%%%%%%%%%%%%%%%%%%%%%%%%%%%%%%%%%%%%%%%%%%%%%%%%%%%%
\ifthenelse{\boolean{showsols}}{\input damp_sol}{}
