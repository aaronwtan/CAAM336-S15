
%All parts of this question should be done by hand.
%
%Let the inner product $\ip{\cdot,\cdot}:\mbox{ }C[0,1]\times C[0,1]\rightarrow\R$ be defined by
%\[
%\ip{v,w}=\int_0^1v(x)w(x)\,dx
%\]
%and let the norm $\norm{\cdot}:\mbox{ }C[0,1]\rightarrow\R$ be defined by
%\[
%\norm{v}=\sqrt{\ip{v,v}}.
%\]
Let the symmetric linear operator $L: C^2_M[0,1]\to C[0,1]$ be defined by
\[
L v = -v''
\]
where
\[
C^2_M[0,1] = \{ w \in C^2[0,1] : w'(0) = w(1) = 0\}.
\]
%Since $C^2_M[0,1]$ is a subspace of $C[0,1]$,
%\[
%\ip{Lv,w}=\ip{v,Lw}\mbox{ for all }v,w\in C^2_M[0,1].
%\]
Let $N$ be a positive integer and let $f\in C[0,1]$ be defined by
\[
f(x)=\left\{\begin{array}{rl}1-2x&\mbox{if }x\in\left[0,{1\over2}\right];\\0&\mbox{otherwise}.\end{array}\right.
\]
\\
\begin{enumerate}
\item The operator $L$ has eigenvalues $\lambda_n$ with corresponding eigenfunctions
\[
\phi_n(x) = \sqrt{2} \cos\left({2n-1\over 2}\pi x\right)
\]
for $n=1,2,\ldots$. We have that, for $m,n=1,2,\ldots$,
\[
\ip{\phi_m,\phi_n}=\left\{\begin{array}{rl}1&\mbox{if }m=n;\\0&\mbox{if }m\ne n.\end{array}\right.
\]
Obtain a formula for the eigenvalues $\lambda_n$ for $n=1,2,\ldots$.
\\
\item Compute $f_N$, the best approximation to $f$ from ${\rm span}\left\{\phi_1,\ldots,\phi_N\right\}$ with respect to the norm $\norm{\cdot}$.  Plot $f_N$ for $N = 1,2,3,4,5,6$.
\\
\item Use the spectral method to obtain a series solution to the problem of finding $\tilde{u}\in C^2[0,1]$ such that
\[
-\tilde{u}''(x) = f(x),\quad 0<x<1
\]
and
\[
\tilde{u}'(0) = \tilde{u}(1) = 0.
\]
%\\
%\item What is the best approximation to $\tilde{u}$ from ${\rm span}\left\{\phi_1,\ldots,\phi_N\right\}$ with respect to the norm $\norm{\cdot}$?
\\
\item By shifting the data, obtain an infinite series solution to the problem of finding $u\in C^2[0,1]$ such that
\[
-u''(x) = f(x),\quad 0<x<1
\]
and
\[
u'(0) = u(1) = 1.
\]
\\
\item Let ${u}_N$ be the series solution that you obtained in part (d) but with $\infty$ replaced by $N$, i.e.
\[
{u}_N(x) = p(x) + \sum_{j = 1}^N \alpha_j \phi_j(x)
\]
where $p(x)$ is the linear function added to match the nonzero boundary conditions in (d).  

Plot ${u}_N$ for $N=1,2,3,4,5,6$.

%\item Plot the solution $u_N(x)$ for $N = 1,2,3,4,5,6$.  
\end{enumerate}


%%%%%%%%%%%%%%%%%%%%%%%%%%%%%%%%%%%%%%%%%%%%%%%%%%%%%%%%%%%%%%%%%%%%%%%%%%%%%%%%

\ifthenelse{\boolean{showsols}}{\input sm_sol}{}

