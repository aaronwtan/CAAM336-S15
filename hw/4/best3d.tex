The equation $x_1 + x_2 + x_3 = 0$ defines a plane in $\R^3$ that passes through the origin.
\begin{enumerate}
\item Find two linearly independent vectors in $\R^3$ whose span is this plane.
\item Find the point in this plane closest (in the standard Euclidean norm,
$\norm{\Bz} = \sqrt{\Bz^T\Bz}$) to the vector
\[ 
\Bv = \left[\begin{array}{c}
1\\ 0\\ 1
\end{array}
\right]
\]
by formulating this as a best approximation problem.
(You may use MATLAB to invert a matrix.)

\end{enumerate}

%%%%%%%%%%%%%%%%%%%%%%%%%%%%%%%%%%%%%%%%%%%%%%%%%%%%%%%%%%%%%%%%%%%%%%%%%%%%%%%%

\ifthenelse{\boolean{showsols}}{\begin{solution}
\begin{enumerate}
\item  Since two linearly independent vectors determine a plane, 
       we simply need to find two linearly independent vectors 
       that satisfy $x_1 + x_2 + x_3 = 0$.
       One can do this by inspection, for example, and find 
         \[ 
         \left[
         \begin{array}{c}
         1 \\ -1 \\ 0
         \end{array}
         \right], \qquad 
                  \left[
         \begin{array}{c}
         0 \\ 1 \\ -1
         \end{array}
         \right].
         \]
       However, it would be nice to have an orthogonal basis for this space.
       To do that, pick one vector, say the first vector given above;
       set the second vector to be $(\alpha, \beta, \gamma)^T$.
       We would like the this vector to be in the plane:
        \[ 
        \alpha + \beta + \gamma = 0
        \]
       and to be orthogonal to the first vector:
         \[ 
         \left[\begin{array}{c}
         1 \\ -1 \\ 0
         \end{array}\right]^T 
         \left[\begin{array}{c}
         \alpha \\ \beta \\ \gamma
         \end{array}\right] = \alpha - \beta + 0 = 0.
         \]
       This gives two equations in three unknowns, which will be satisfied
       if $\beta=\alpha$ and $\gamma = -2\alpha$ for any $\alpha$, 
       i.e., we have the vector
                 \[ 
                 \left[\begin{array}{c}\alpha \\ \alpha  \\ -2\alpha\end{array}\right].
                 \] 
       With $\alpha =1 $, we have two orthogonal vectors whose span is the desired plane:
         \[ \Bx = 
         	\left[\begin{array}{c}1 \\ -1 \\ 0\end{array}\right], \qquad 
	\By = \left[\begin{array}{c}1 \\ 1 \\ -2\end{array}\right].
	\]
       
\item  The closest point in the plane to the vector $\Bv$ is found solving
       the usual best-approximation problem matrix equation:
       \[  \left[\begin{array}{cc} \Bx^T\Bx & \Bx^T \By \\ \By^T\Bx & \By^T\By
       \end{array}\right] 
         \left[\begin{array}{c}c_1\\ c_2\end{array}\right]
        = \left[\begin{array}{c} \Bx^T\Bv \\ \By^T \Bv \end{array}\right],\]
       that is, 
       \[  \left[\begin{array}{cc} 2 & 0 \\ 0 & 6\end{array}\right] 
\left[\begin{array}{c}c_1\\ c_2\end{array}\right]
        = \left[\begin{array}{c} 1 \\ -1\end{array}\right].
        \]
       The orthogonality of the vectors $\Bx$ and $\By$ make this an easy problem to solve:
        \[ c_1 = 1/2, \qquad c_2 = -1/6.\]
       Thus, the best approximation to $\Bv = (1, 0, 1)^T$ is the vector
         \[ 
\widehat{\Bv} = c_1 \Bx + c_2 \By = \left[\begin{array}{c}1/2-1/6 \\ -1/2-1/6 \\ 1/3\end{array}\right] 
                                      = \left[\begin{array}{c} 1/3 \\ -2/3 \\ 1/3\end{array}\right].
                                      \]
       We can verify this answer by checking 
       (1) that $\widehat{\Bv}$ is in the desired plane: $1/3-2/3+1/3 = 0$, 
        and (2) verifying that the error 
            \[ \Bv - \widehat{\Bv} = \left[\begin{array}{c}2/3 \\ -2/3 \\ 2/3\end{array}\right]\]
        is orthogonal to the two basis vectors $\Bx$ and $\By$ for the plane,
        $(\Bv-\widehat{\Bv})^T\Bx = (\Bv-\widehat{\Bv})^T\By = 0$.
 \end{enumerate}

\end{solution}}{}

