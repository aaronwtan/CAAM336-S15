\documentclass[10pt]{article}
\usepackage{fullpage}
\def\CV{{\cal V}}
\def\CW{{\cal W}}
\def\R{{\bf R}}
\def\Rmn{\R^{m\times n}}
\def\Rnn{\R^{n\times n}}
\def\Rm{\R^{m}}
\def\Rn{\R^{n}}
\def\Bb{\bf b}
\def\Be{\bf e}
\def\Bu{\bf u}
\def\Bx{\bf x}
\def\BA{\bf A}
\def\BB{\bf B}
\def\BX{\bf X}
\def\dop{{\rm d}}
\def\ip#1{(#1)}
\def\norm#1{\|#1\|}
\parindent0em\parskip1em
\pagestyle{empty}
\begin{document}
\begin{center}
\large \textsf{\textbf{CAAM 336 $\cdot$ DIFFERENTIAL EQUATIONS}\\[0.5em]
 Problem Set 3}
\end{center}

Posted Friday 5 September 2008.  Due Friday 12 September 2008, 5pm.

\vspace*{-1em}
\begin{enumerate}
\item {[18 points]}\\  
Recall that a function $f: \CV \to \CW$ that maps a vector space $\CV$ to
a vector space $\CW$ is a \emph{linear operator}
provided (1) $f(u+v) = f(u)+f(v)$ for all $u, v$ in $\CV$,
and (2) $f(\alpha v) = \alpha f(v)$ for all $\alpha\in \R$ and $v\in\CV$.

Demonstrate whether each of the following functions is a linear operator.\\
(Show that both properties hold, or give an example showing that one 
of the properties must fail.)

\begin{enumerate}
\item $f:\R^n\to \R^m$, $f(\Bu) = \BA \Bu$ for a fixed matrix $\BA\in\Rmn$.
\item $f:\R^n\to \R^m$, $f(\Bu) = \BA \Bu + \Bb$ for a fixed matrix $\BA\in\Rmn$ and fixed vector $\Bb\in\Rm$.
\item $f:\R^2 \to \R$,  $f(\Bx) = \Bx^T \Bx$.
\item $f:\Rnn\to \Rnn$, $f(\BX) = \BA\BX + \BX\BB$ for fixed matrices $\BA, \BB\in\Rnn$.
\item $L: C^1[0,1] \to C[0,1]$, ${\displaystyle{ Lu = u {\dop u \over \dop x}}}$.
\item $L: C^2[0,1] \to C[0,1]$, ${\displaystyle{ Lu = {\dop^2 u \over \dop x^2} - \sin(x) {\dop u \over \dop x} + \cos(x) u}}$.
\end{enumerate}

\vspace*{.5em}
\item {[20 points]}\\[-1.5em]  
\begin{enumerate}
\item Suppose that $f: \R^2 \to \R^2$ is linear.  
      Prove there exists a matrix $\BA\in\R^{2\times 2}$ such that
      $f$ is given by $f(\Bu) = \BA\Bu$.  
      Hint: Each $\Bu  = \left[\!\!\begin{array}{c} u_1 \\ u_2\end{array}\!\!\right] \in \R^2$
            can be written as $\Bu = u_1 \Be_1 + u_2 \Be_2$, where
                 \[ \Be_1 = \left[\!\begin{array}{c} 1 \\ 0\end{array}\!\right], \qquad
                    \Be_2 = \left[\!\begin{array}{c} 0 \\ 1\end{array}\!\right]. \] 
      Since $f$ is linear, we have $f(\Bu) =  u_1 f(\Be_1) + u_2 f(\Be_2)$.
      Your formula for the matrix $\BA$ may include the vectors $f(\Be_1)$ and $f(\Be_2)$.

\item Now we want to generalize the result in part~(a): Show that if 
      $f: \R^n \to \R^m$ is linear, then there exists a matrix $\BA\in\Rmn$
      such that $f(\Bu) = \BA\Bu$ for all $\Bu\in\Rn$.\\[0.5em]
      (Thus any linear function that maps $\R^n$ to $\R^m$ can be written 
        as a matrix-vector product.)
\end{enumerate}

\vspace*{.5em}
\item {[20 points]}\\ 
Determine whether each of the following functions $\ip{\cdot,\cdot}$ 
determines an inner product on the vector space $\CV$.  
If not, demonstrate which property (or properties) 
of the inner product are violated.

\begin{center}\begin{tabular}{ll}
 (a)\ $\CV = C^1[0,1]$, $\displaystyle{\ip{u,v} = \int_0^1 u'(x) v'(x)\, dx}$ & 
 (b)\ $\CV = C[0,1]$: $\displaystyle{\ip{u,v} = \int_0^1 |u(x)| |v(x)|\, dx}$ \ \ \ \ \ \\[0.75em]
 (c)\ $\CV = C[0,1]$: $\displaystyle{\ip{u,v} = \int_0^1 u(x) v(x) e^{-x}\, dx}$ & 
 (d)\ $\CV = C^1[0,1]$: $\displaystyle{\ip{u,v} = \int_0^1 u(x) v'(x)\, dx}$ \\
\end{tabular}\end{center}

\vspace*{.5em}
\item {[10 points]}\\  
Suppose $\CV$ is a vector space with an associated inner product.
The angle $\theta$ between $u$ and $v \in \CV$ is defined via the equation
  \[ \ip{u,v} = \norm{u} \norm{v} \cos \theta.\]
Let $\CV=C[0,1]$ and $\ip{u,v} = \int_0^1 u(x) v(x)\, dx$.
      Compute the angle $\theta$ between $u(x) = x^n$ and $v(x) = x^m$ 
       for nonnegative integers $m$ and $n$.

\vfill\hfill \emph{please turn over}

\newpage
\item {[32 points + 8 bonus]}\\ 
Suppose $N\ge 1$ is an integer and define $h = 1/(N+1)$ and $x_j = jh$ for $j=0,\ldots, N+1$.

We can approximate the differential equation
\[ {d^2 \over dx^2} u = f(x), \quad x\in (0,1),\]
\[u(0)=u(1) = 0\]
by the matrix equation
\[ {1\over h^2} \left[\begin{array}{rrrrr}
              -2 & 1 \\[0.25em]
               1 & -2 & 1 \\
                 &  1  & -2 & \ddots \\
                 & & \ddots & \ddots & 1 \\[0.25em]
                 & & & 1 & -2 
               \end{array}\right]
          \left[\begin{array}{c} u_1 \\[0.25em] u_2 \\[0.25em] \vdots \\[0.25em] u_{N-1} \\[0.25em] u_N \end{array}\right]
 =   \left[\begin{array}{c} f(x_1) \\[0.25em] f(x_2) \\[0.25em] \vdots \\[0.25em] f(x_{N-1}) \\[0.25em] f(x_N) \end{array}\right],\]
where $u_j \approx u(x_j)$.  (Entries of the matrix that are not specified are zero.)

\begin{enumerate}
\item Suppose that $f(x) = 25 \pi^2 \cos (5\pi x)$.\\
      Compute and plot the approximate solutions obtained when $N=8, 16, 32, 64, 128$.\\
      You may superimpose these on one plot.  
       To solve the linear systems, you may use MATLAB's `backslash' command: 
        \verb| u = A\f|.
      
      For each value of $N$ compute the maximum error $|u_j - u(x_j)|$, given that
      the true solution is  \[u(x) = 1-2x-\cos(5\pi x).\]
      Plot this error using a \verb|loglog| plot 
      with error on the vertical axis and $N$ on the horizontal axis.
     
\vspace*{1.5em} 
\item Explain what adjustments to the right hand side of the matrix equation 
      are necessary to accommodate the inhomogeneous Dirichlet boundary conditions 
      \[ u(0)=1,\quad  u(1)=2.\]
      Compute and plot solutions for $N=8, 32, 128$. 

\vspace*{1.5em} 
\item This part of the problem is an optional bonus worth 8 extra points:\\
      Now suppose that we have mixed boundary conditions
           \[ u(0) = 1, \quad {d u\over dx}(1) = -5.\]
      The matrix equation will now have the form
\[ {1\over h^2} \left[\begin{array}{rrrrrr}
              -2 & 1 \\[0.25em]
               1 & -2 & 1 \\
                 &  1  & -2 & \ddots \\
                 & & \ddots & \ddots & 1 & 0 \\[0.25em]
                 & & & 1 & -2 & * \\[0.25em]
                 & & & * &  * & * \end{array}\right]
          \left[\begin{array}{c} u_1 \\[0.25em] u_2 \\[0.25em] \vdots \\[0.25em] u_{N-1} \\[0.25em] u_N \\[0.25em] u_{N+1}\end{array}\right]
 =   \left[\begin{array}{c} f(x_1) - *\\[0.25em] f(x_2) \\[0.25em] \vdots \\[0.25em] f(x_{N-1}) \\[0.25em] f(x_N)\\[0.25em] * \end{array}\right].\]
      Specify what the entries marked by $*$ should be, keeping in mind the approximation
           \[ {d u \over dx}(1) \approx {u_{N+1}-u_N \over h}.\]
      Compute and plot solutions for $N=8, 32, 128$. 
\end{enumerate} 

\end{enumerate}
\end{document}
