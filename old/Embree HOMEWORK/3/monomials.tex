Suppose $\CV$ is a vector space with an associated inner product.
The angle $\angle(u,v)$ between $u$ and $v \in \CV$ is defined via the equation
  \[ \ip{u,v} = \norm{u} \norm{v} \cos \angle(u,v).\]
Let $\CV=C[0,1]$ and $\ip{u,v} = \int_0^1 u(x) v(x)\, dx$.
      Compute $\cos \angle(x^n,x^m)$ between $u(x) = x^n$ and $v(x) = x^m$ 
       for nonnegative integers $m$ and $n$.
      What happens to $\angle(x^n,x^{n+1})$ as $n\to\infty$ ?

%%%%%%%%%%%%%%%%%%%%%%%%%%%%%%%%%%%%%%%%%%%%%%%%%%%%%%%%%%%%%%%%%%%%%%%%%%%%%%%%

\ifthenelse{\boolean{showsols}}{\begin{solution}
This result is a simple calculation.  For integers $n,m\ge 0$, we have
\[ (x^n, x^m) = \int_0^1 x^{n+m}\, dx = {1\over n+m+1}.\]
We also have 
\[  \norm{x^n}^2 = (x^n, x^n) = {1\over 2n+1}.\]
Altogether, we have the following formula for the angle between $x^n$ and $x^m$:
\[ \cos \angle(x^n,x^m) = {(x^n, x^m) \over \norm{x^n} \norm{x^m}}
               = {\sqrt{(2n+1)(2m+1)} \over n+m+1}.\]
As $m=n+1$ and $n\to\infty$, we see that $\cos \angle(x^n,x^{n+1}) \to 1$,
implying that the angle between $x^n$ and $x^{n+1}$ shrinks to zero as
$n\to\infty$.  
This observation is all that is required for full credit, 
but a little manipulation yields even greater insight.  
We can equivalently write
\[ \cos \angle(x^n,x^m) = \sqrt{ 1 - {(n-m)^2 \over (n+m+1)^2}},\]
which for larger $m$ and $n$ gives the approximation
\[ \cos \angle(x^n,x^m) \approx 1 - {1\over 2} {(n-m)^2 \over (n+m+1)^2}
                         - {1\over 8} {(n-m)^4 \over (n+m+1)^4}.\]
Note that for small $\theta$,
\[ \hspace*{-9.25em}\cos \theta \approx 1 - {1\over 2} \theta^2 
                         + {1\over 24} \theta^4.\]
Comparing terms, we arrive at the first-order approximation
\[ \angle(x^n,x^m) \approx {|n-m| \over n+m+1}\]
suitable when $n+m$ is large and best when $|n-m|$ is small as well.
This approximation confirms the intuition 
we obtain from comparing graphs of $x^n$ and $x^m$ for large $m$ and $n$:
the graphs look very similar, so we expect the `angle' between these
function in the inner product space to be small.
This result of this problem signals that the usual basis for polynomials, 
$1, x, x^2, x^3, \ldots$ may introduce computational challenges.
We prefer to use an orthogonal basis whenever possible.

\end{solution}}{}

