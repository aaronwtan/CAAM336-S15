\documentclass[10pt]{article}
\usepackage{fullpage,336notes,ifthen}
\parindent0em\parskip1em
\newboolean{showsols}
\pagestyle{empty}
\newenvironment{solution}{\vspace*{1em} \hrule\vspace*{0.5em} {\sf Solution.} }
                         {\vspace*{0.75em}\hrule}
\begin{document}
\setboolean{showsols}{false}
%\vspace*{-3em}
\begin{center}
\large \textsf{\textbf{CAAM 336 $\cdot$ DIFFERENTIAL EQUATIONS}\\[0.5em]
 Problem Set 11 \ifthenelse{\boolean{showsols}{$\cdot$ Solutions}}{}}
\end{center}

Posted Saturday 29 November 2008.  (Typos fixed 2 December.) Due Friday 5 December 2008, 5pm.\\
This problem set is worth 100 points, plus a 20 point bonus.\\
Your lowest homework score for the semester will be dropped when 
computing your final mark.
\begin{enumerate}

\item {[50 points]}\\ 
On Problem Set~10, you solved the heat equation on a two-dimensional square domain.
Now we will investigate the wave equation on the same domain,
a model of a vibrating membrane stretched over a square frame---that 
is, a square drum:
\[ u_{tt}(x,y,t) = u_{xx}(x,y,t) + u_{yy}(x,y,t),\]
with $0\le x\le 1$, and $0\le y\le 1$, and $t\ge 0$.
Take homogeneous Dirichlet boundary conditions
\[ u(x,0,t) = u(x,1,t) = u(0,y,t) = u(1,y,t) = 0\]
for all $x$ and $y$ such that $0\le x\le 1$ and $0\le y\le 1$ and all $t\ge 0$,
and consider the initial conditions
\[  u(x,y,0) = u_0(x,y) = \sum_{j=1}^\infty \sum_{k=1}^\infty a_{j,k}(0) \psi_{j,k}(x,y), 
       \qquad
   u_t(x,y,0) = v_0(x,y) 
               = \sum_{j=1}^\infty \sum_{k=1}^\infty b_{j,k}(0) \psi_{j,k}(x,y).\]
      Here $\psi_{j,k}(x,y) = 2 \sin(j \pi x)\sin(k \pi y)$, 
      for $j, k\ge 1$,  are the eigenfunctions of the operator
       \[ L u = -(u_{xx} + u_{yy}),\]
      with homogeneous Dirichlet boundary conditions given in Problem Set~10.
      You may use without proof that these eigenfunctions are orthogonal, 
      and use the eigenvalues $\lambda_{j,k} = (j^2+k^2)\pi^2$ computed 
      for Problem Set~10.

\begin{enumerate}
\item We wish to write the solution to the wave equation in the form
       \[ u(x,y,t) = \sum_{j=1}^\infty \sum_{k=1}^\infty a_{j,k}(t) \psi_{j,k}(x,y).\]
      Show that the coefficients $a_{j,k}(t)$ obey the ordinary differential equation
       \[ a_{j,k}''(t) = -\lambda_{j,k} a_{j,k}(t) \]
      with initial conditions
       \[ a_{j,k}(0), \qquad
         a'_{j,k}(0) = b_{j,k}(0)\]
      derived from the initial conditions $u_0$ and $v_0$.

\item Write down the solution to the differential equation in part~(a).

\item Use your solution to part~(b) to write out a formula for the solution $u(x,y,t)$.
       
\item Suppose the string begins with zero velocity, $v_0(x)=0$, and displacement
         \[ u_0(x,y) = 200xy (1-x) (1-y)(x-1/4)(y-1/4) 
                     = \sum_{j=1}^\infty \sum_{k=1}^\infty 
                        {100 (5+7(-1)^j)(5+7(-1)^k)\over j^3 k^3 \pi^6} \psi_{j,k}(x,y). \]

      Submit surface (or contour) plots of the solution at times 
      $t=0, 0.5, 1.0, 1.5, 2.5$, using $j=1,\ldots 10$ and $k=1,\ldots 10$
      in the series.
\end{enumerate}

\item {[50 points]}\\ 
Our model of the vibrating string predicts that motion induced by an 
initial pluck will propagate forever with no loss of energy.
In practice we know this is not the case: a string 
eventually slows down due to various types of \emph{damping}.
For example, \emph{viscous damping}, a model of air resistance,
acts in proportion to the velocity of the string.
The partial differential equation becomes
\[ u_{tt}(x,t) = u_{xx}(x,t) - 2@\delta@u_t(x,t),\]
where $\delta> 0$ controls the strength of the damping.
Impose homogeneous Dirichlet boundary conditions,
\[ u(0,t) = u(1,t) = 0\]
and suppose we know the initial position and velocity of the pluck:
\[ u(x,0) = u_0(x), \qquad u_t(x,0) = v_0(x).\]
In our previous language, we write this PDE in the form 
\[ u_{tt} = -L u - 2@\delta@u_t,\]
where the operator $L$ is defined as $Lu = -u_{xx}$ with
boundary conditions $u(0)=u(1)=0$; as you know well by now,
this operator has eigenvalues $\lambda_k = k^2\pi^2$
and eigenfunctions $\psi_k(x) = \sqrt{2} \sin(k \pi x)$.
We will look for solutions to the PDE of the form
\[ u(x,t) = \sum_{k=1}^\infty a_k(t) \psi_k(x).\]
For simplicity, assume that $\delta\in(0,\pi)$.

\begin{enumerate}
\item From the differential equation and this form for $u(x,t)$,
      show that the coefficients $a_k(t)$ must satisfy the
      ordinary differential equation
\[ a_k''(t) = -\lambda_k a_k(t) - 2 \delta a_k'(t).\]

\item Show that the following function satisfies the differential equation in part~(a):
\[ a_k(t) = C_1 \exp\!\big(\big(-\!\delta+\sqrt{\delta^2-k^2\pi^2}\big)t\big)
            + C_2 \exp\!\big(\big(-\!\delta-\sqrt{\delta^2-k^2\pi^2}\big)t\big)\]
      for arbitrary constants $C_1$ and $C_2$.  (Don't fret about the fact that we
      have square roots of negative numbers; proceed in the same way you 
      would for an exponential with real argument.)

\item Now assume that the string starts with zero displacement
      ($u_0(x) = 0$) but some velocity 
     \[ v_0(x) = \sum_{k=1}^\infty b_k(0)  \psi(x).\]
      Determine the values of the constants $C_1$ and $C_2$ in part~(b)
      for these initial conditions. 

\item Suppose we have $u_0(x)=0$ and initial velocity 
      $v_0(x) = x \sin(3\pi x)$, for which
      \[ b_k(0) = {-6@k@\sqrt{2} (1+(-1)^k) \over (k^2 -9)^2 \pi^2} \mbox{\ \ for $k\ne 3$}, 
    \qquad b_3(0) = {\sqrt{2}\over 4}.\]

      Take damping parameter $\delta=1$, and plot the solution $u(x,t)$ 
      (using 20 terms in the series) at times $t=.15, .3, .6, 1.2, 2.4$.  
      (You may superimpose these on one well-labeled plot; for clarity,
      set the vertical scale to $[-0.1,0.1]$.)

\item Take the same values of $u_0$ and $v_0$ used in part~(d).
      Plot the solution at time $t=2.5$ for $\delta = 0, .5, 1, 3$
      on one well-labeled plot, again using vertical scale $[-0.1,0.1]$.
      How does the solution depend on the damping parameter $\delta$?
\end{enumerate}

\item[bonus] {[20 points]}\\ 
\def\bmatrix#1{\left[\matrix{#1}\right]}
In class we saw that the wave equation $u_{tt} = u_{xx}$ could be written
as a first order system.  We introduce the velocity variable
\[ v(x,t) = u_t(x,t),\]
then notice that the wave equation is equivalent to the first order equation
\[ {\partial \over \partial t} \bmatrix{u(x,t)\cr v(x,t)}
    = \bmatrix{ 0 & I \cr \partial^2/\partial x^2 & 0} 
      \bmatrix{u(x,t)\cr v(x,t)}\]
where $I$ is the identity operator. 
(The first row of this system gives $u_t = v$, while the second row
gives $v_t = u_{xx}$, which is just another way of writing $u_{tt} = u_{xx}$.)
We also have the usual initial conditions,
\[ \bmatrix{u(x,0) \cr v(x,0)} = \bmatrix{u_0(x) \cr v_0(x)}.\]
This system can be analyzed in terms of its eigenvalues and eigenfunctions.
In this problem, we will use the same approach to study the damped wave equation
from Problem~2.
\begin{enumerate}
\item Verify that the damped wave equation $u_{tt} = u_{xx} - 2@\delta@u_t$ is equivalent
      to the system
\[ {\partial \over \partial t} \bmatrix{u(x,t)\cr v(x,t)}
    = \bmatrix{ 0 & I \cr \partial^2/\partial x^2 & -2\delta I} 
      \bmatrix{u(x,t)\cr v(x,t)}.\]
\item For any value of $\delta$, the operator
      \[ A(\delta) = \bmatrix{ 0 & I \cr \partial^2/\partial x^2 & -2@\delta@I}\]
      always has the same eigenvectors,
      \[ \Psi_{\pm k}(x) = \bmatrix{ \sin(k \pi x)\cr  \lambda_{\pm k} \sin(k\pi x)}.\]
      Use these eigenvectors to determine the eigenvalues $\lambda_{\pm k}$.
\item Plot the eigenvalues from part~(b) in the complex plane for $\delta=0, \pi/2, \pi$, and
      $3\pi/2$.
\item What value of $\delta\ge 0$ optimizes the \emph{asymptotic decay} of the system?
      That is, what value of $\delta$ moves \emph{all} the eigenvalues as far to the
      left in the complex plane as possible?  Said yet another way, which $\delta$ 
      minimizes the real part of the rightmost eigenvalue?
\item Confirm this result by plotting the solution at $t = 1.5$ for $\delta = 2.5, 3.2, 20$
      on the same plot with vertical axis $[-.002,.002]$, using the same values for
      $u_0$ and $v_0$ as in part~(d) of Problem~2.  (You should simply adapt the code
      you developed for Problem~2 to produce these plots.)
\end{enumerate}
\end{enumerate}

\end{document}
