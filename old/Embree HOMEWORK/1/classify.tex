For each of the following equations, specify whether each is
(a)~an ODE or a PDE; 
(b)~determine its order; 
(c)~specify whether it is linear or nonlinear.
For those that are linear, specify whether they are 
(d)~homogeneous or inhomogeneous,
and (e)~whether they have constant or variable coefficients.
\begin{center} \begin{tabular}{lcl}
(1.1)\ \  $\displaystyle {d v \over d x} + {2 \over x} v = 0$ 
& &
(1.2)\ \ $\displaystyle {\partial v \over \partial t} 
                     - 3 {\partial v \over \partial x}  = x-t $\\[2em]
(1.3)\ \  $\displaystyle {\partial u \over \partial t} 
                      - {\partial \over \partial x} 
                         \left[ 2 u {\partial u \over \partial x}\right] = 0 $
& &
(1.4)\ \  $\displaystyle {\partial u \over \partial t} 
                   + u {\partial u \over \partial x} 
                   + {\partial^3 u \over \partial x^3} = 0$ \\[2em]  % KDV
(1.5)\ \  $\displaystyle {d^2 y \over dx^2 } 
                   - \mu (1-y^2) {d y \over d x} + y 
                   = 0$   % Van der Pol
& &
(1.6)\ \  $\displaystyle {d^2 \over dx^2} \left[ \rho(x) {d^2 u \over d x^2}\right] 
                    = \sin(x)$  % beam equation
\end{tabular}\end{center}

%%%%%%%%%%%%%%%%%%%%%%%%%%%%%%%%%%%%%%%%%%%%%%%%%%%%%%%%%%%%%%%%%%%%%%%%%%%%%%%%
\ifthenelse{\boolean{showsols}}{\begin{solution}

\begin{enumerate}
\item[(1.1)] ODE, first order, linear, homogeneous, variable coefficient\\
      The $2/x$ factor in front of the $v$ is the variable coefficient.
\item[(1.2)] PDE, first order, linear, inhomogeneous, constant coefficient\\
      The $x-t$ term on the right, which does not involve $v$, makes the
      equation inhomogeneous.
\item[(1.3)] PDE, second order, nonlinear\\
      Using the product rule for partial derivatives, we can write this
      equation in the equivalent form
        \[ {\partial u \over \partial t} 
            - 2 \left({\partial u \over \partial x}\right)^{\!\!2}
            - 2 u \left({\partial^2 u \over \partial x^2}\right) = 0.\]
      The second and third terms on the left hand side 
      make this equation nonlinear.
\item[(1.4)] PDE, third order, nonlinear\\
      The $u(\partial u / \partial x)$ term makes this equation nonlinear.
      This a version of the famous Korteweg-de Vries (KdV) equation that 
      describes shallow water waves.
\item[(1.5)] ODE, second order, nonlinear\\
      The $(1-y^2) (dy/dt)$ term makes this ODE nonlinear.
\item[(1.6)] ODE, fourth order, linear, inhomogeneous, variable coefficient\\
      Using the product rule for partial derivatives, we can write this
      equation in the equivalent form
      \[{d^2 \rho\over dx^2} {d^2u \over dx^2} 
              + 2 {d\rho \over dx} {d^3 u \over dx^3}
              + \rho(x) {d^4 u\over dx^4} = \sin(x),\]
      hence we can see that it is fourth order.
      This equation, attributed to Euler, describes the deflection 
      of a one-dimensional beam with a static load of $\sin(x)$;
      $\rho(x)$ describes the elasticity of the material that 
      constitutes the beam.
   
\end{enumerate}
\end{solution}}{}

