Consider the operator $L:C_\ell^1[0,1] \to C[0,1]$ defined by
\[ Lu = {du\over dx},\]
where 
\[ C_\ell^1[0,1] = \{ u \in C^1[0,1], u(0) = 0\}.\]

\begin{enumerate}
\item Show that $L$ is a linear operator.
\item Is $L$ symmetric?
\item Show that $L$ has \emph{no} eigenvalues, that is, 
      demonstrate that there exist no nonzero $u \in C_\ell^1[0,1]$ 
      and $\lambda\in\C$ for which $Lu = \lambda u$.
\end{enumerate}

%%%%%%%%%%%%%%%%%%%%%%%%%%%%%%%%%%%%%%%%%%%%%%%%%%%%%%%%%%%%%%%%%%%%%%%%%%%%%%%%

\ifthenelse{\boolean{showsols}}{\begin{solution}

\begin{enumerate}
\item Suppose that $u, v \in C_\ell^1[0,1]$ and $\alpha, \beta\in \R$.
      Then 
         \[ L(\alpha u + \beta v) 
            = {d \over dx} (\alpha u + \beta v) 
            = \alpha {du\over dx} + \beta {dv\over dx}
            = \alpha Lu + \beta Lv.\]
       Hence $L$ is a linear operator.
 
\item No, $L$ is \emph{not symmetric}.  To show this, we need only
      exhibit two functions $u, v \in C_\ell^1[0,1]$ such that 
      $(Lu,v) \ne (u,Lv)$.
      For example, take $u(x) = x$ and $v(x) = x^2$.
      Then
         \begin{eqnarray*}
               (Lu, v) &=& \int_0^1 {du \over dx}(x) v(x)\, dx
                           = \int_0^1 x^2 \, dx = \Big[{x^3 \over 3}\Big]_0^1 = 1/3;\\[0.5em]
               (u, Lv) &=& \int_0^1 u(x) {dv \over dx}(x) \, dx
                           = \int_0^1 x (2x) \, dx = \Big[{2x^3 \over 3}\Big]_0^1 = 2/3.
         \end{eqnarray*}
       Hence $(Lu,v) \ne (u,Lv)$, and so $L$ is not symmetric.
    
       {[\textbf{GRADERS:} Please deduct 2~points if either (or both) functions 
         $u$ and $v$ are not in $C_\ell^1[0,1]$.]}

\item Suppose that there exists some function $u \in C_\ell^1[0,1]$ such that
      $L u = \lambda u$ for some $\lambda\in\C$.
      This means that $du/dx = \lambda u$, which can only be satisfied if
               \[ u(x) = c e^{\lambda x}\]
      for some constant $c$.
      Unfortunately, in order to have $u\in C_\ell^1[0,1]$, we must have
				  $u(0)=0$, which can only occur if $c=0$ since $e^{\lambda\cdot 0} = 1 \ne 0$
      for all $\lambda\in \C$.  If $c=0$ then $u(x) = 0$ for all $x\in[0,1]$, 
      which violates the requirement that $u$ be nonzero.
      Hence $L$ is an example of a linear operator with no eigenvalues.
\end{enumerate}
\end{solution}}{}

