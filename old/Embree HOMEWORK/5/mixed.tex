Consider the linear operator $L_b:C_b^2[0,1] \to C[0,1]$ defined by
\[ L_b u = -{d^2 u \over dx^2},\]
where 
\[ C_b^2[0,1] = \Big\{ u\in C^2[0,1]: {du\over dx}(0) = u(1) = 0\Big\}.\]
\begin{enumerate}
\item Is $L_b$ symmetric?
\item What is the null space of $L_b$?\\
      That is, find all $u\in C_b^2[0,1]$ such that $L_b u (x) = 0$ for all $x\in[0,1]$.
\item Show that $(L_b u,u)> 0$ for all nonzero $u\in C^2_b[0,1]$
      and explain why this implies that $\lambda > 0$ for all eigenvalues $\lambda$.
\item Find the eigenvalues and eigenfunctions of $L_b$.
\end{enumerate}

%%%%%%%%%%%%%%%%%%%%%%%%%%%%%%%%%%%%%%%%%%%%%%%%%%%%%%%%%%%%%%%%%%%%%%%%%%%%%%%%

\ifthenelse{\boolean{showsols}}{\begin{solution}
\begin{enumerate}
\item Yes, \emph{$L_b$ is symmetric}.\\
      To prove this, we shall show that for any $u,v \in C_b^2[0,1]$, 
      we have $(L_bu, v) = (u,L_bv)$.
      He we shall use primes to denote derivation with respect to $x$.
      Integrating by parts twice, we have
       \begin{eqnarray*}
        (L_bu, v) &=& \int_0^1 -u''(x) v(x)\, dx  \\[0.5em]
                &=& -[u'(x) v(x)]_0^1 + \int_0^1 u'(x) v'(x)\, dx \\[0.5em]
                &=& -[u'(x) v(x)]_0^1 + [u(x) v'(x)]_0^1 - \int_0^1 u(x) v''(x)\, dx.
       \end{eqnarray*}
       Since $u, v\in C_b^2[0,1]$ we have $u'(0)=0$ and $v(1) = 0$, and hence
       the first term in square brackets must be zero.  
       Again using $u, v\in C_b^2[0,1]$ we have $v'(0)=0$ and $u(1) = 0$,  
       and hence the second term in square brackets is also zero.  
       It follows that
        \[ (L_b u, v) = \int_0^1 u(x) (-v''(x))\, dx = (u, L_b v).\]

\item  The general solution to the differential equation 
          \[ -{d^2 u\over dx^2} = 0\]
        has the form
          \[ u(x) = A + B x\]
        for constants $A$ and $B$ that are determined by the boundary conditions.
        In order for $u$ to be in $C_b^2[0,1]$, we must have $u'(0)=0$; 
        since $u'(x) = B$, we have $B=0$.  Now $u\in C_b^2[0,1]$ also requires
        $u(1) = 0$, and since $u(1)=A$, we conclude that $A=0$ too, meaning that
        $u(x) = A+Bx = 0$ for all $x\in[0,1]$.  
        Thus, the only element of the null space is the zero function,
        that is, the null space is trivial, $\ker(L_b) = \{0\}$.

\item   {[\textbf{GRADERS}: please count $(L_b u,u)> 0$ for 5~points,
            and count $\lambda> 0$ for 4~points.]}

        We wish to show that $(L_b u, u) > 0$ for all $\in C_b^2[0,1]$.
        Using the first integration by parts from part~(a), we have
          \begin{eqnarray*}
               (L_b u, u) &=&  -[u'(x) u(x)]_0^1 + \int_0^1 u'(x) u'(x)\, dx \\[0.5em]
                          &=&  \int_0^1 (u'(x))^2\, dx.
          \end{eqnarray*}
         Thus, $(L_b u, u)$ is the integral of a nonnegative function, so it is nonnegative.

         
          This statement implies that all eigenvalues are non-negative, since
           \[ \lambda (u, u) = (\lambda u, u) = (L_b u, u) \ge 0,\]
          and we know that $(u,u)>0$ for all nonzero $u$ due to the positivity
          of the inner product. 

          One can show that $u'(x) \ne 0$ for some $x\in[0,1]$: otherwise
          $u$ would be a constant, and the only constant that satisfies the boundary
          conditions is $u(x)=0$, which is prohibited from consideration in the problem
          statement.  Thus, we can say $(L_b u, u) > 0$ for all nonzero $u$.

\item   Suppose that $L_b u = \lambda u$.  The general solution to the equivalent
        differential equation
                   \[  -{d^2 u\over dx^2} (x) = \lambda u(x)\]
        has the form
                   \[ u(x) = A \sin(\sqrt{\lambda} x) + B \cos(\sqrt{\lambda}x).\]
        Now we must find values of the constants $A$, $B$, and $\sqrt{\lambda}$
        that will satisfy the boundary conditions.  Since
                    \[ u'(x) = A\sqrt{\lambda} \cos(\sqrt{\lambda}x)
                               - B \sqrt{\lambda} \sin(\sqrt{\lambda}x)\]
        and thus
                    \[ u'(0) = A \sqrt{\lambda},\]
        the condition $u'(0) = 0$ implies that $A=0$.
        On the other hand, the condition $u(1) = 0$ implies that
                    \[ u(1) = B \cos(\sqrt{\lambda}) = 0,\]
        which can be achieved with nonzero $B$ provided that 
        $\sqrt{\lambda} = (n-1/2)\pi$ for integers $n$.
        We thus have the eigenvalues
                    \[ \lambda_n = (n-1/2)^2 \pi^2\]
        with corresponding eigenfunctions
                    \[ u_n(x) = \cos(\sqrt{\lambda_n} x)\]
        or any nonzero scaling of the same function, such as the normalized version:
                    \[ u_n(x) = \sqrt{2} \cos(\sqrt{\lambda_n} x).\]
        Since $n = 0, -1, -2, \ldots$ give the same eigenvalues and eigenvectors
        as $n = 1, 2, 3, \ldots$, we can restrict $n$ to the positive integers.
\end{enumerate}
\end{solution}}{}

