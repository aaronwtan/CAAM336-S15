
Many fluid dynamics problems lead to 
\emph{advection--diffusion} equations, the 
simplest example of which is
\[ u''(x) + c u'(x) = f(x),\]
for $x\in[0,1]$ with $u(0)=u(1) = 0$.
(The $u''$ term describes diffusion of a fluid;
the constant $c$ describes the strength with which
the fluid advects across the domain through the $cu'$ term.)

\vspace*{1em}
Define the linear operator $L: C^2_D[0,1] \to C[0,1]$ by $L u = u'' + c u'$.

\vspace*{1em}
Determine all the eigenvalues and eigenfunctions of $L$.\\
Are the eigenfunctions orthogonal?  Explain.

\ifthenelse{\boolean{showsols}}{
\begin{solution}
The eigenvalues are 
\[ \lambda_n = {-{c^2\over 4} - n^2 \pi^2}, \quad n = 1, 2, \ldots\]
with corresponding eigenfunctions
 \[ \psi_n(x) = e^{-cx/2} \sin(n \pi x), \qquad n=1,2,\ldots.\]
If $c\ne 0$, then these eigenfunctions are not orthogonal
(and the operator is not symmetric). 

To receive credit, students must show how they derived these eigenvalues and eigenfunctions.

\textbf{GRADERS}: please let me know those students who got this bonus problem correct.
\end{solution} }{}

