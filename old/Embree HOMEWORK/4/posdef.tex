A symmetric matrix $\BA\in\Rnn$ is \emph{positive definite} provided that
\[ \Bx^T\BA\Bx > 0\]
for all nonzero $\Bx\in\Rn$.

Show that a symmetric matrix $\BA$ is positive definite 
if and only if all its eigenvalues are positive.

This means that you need to prove two things:  
\begin{enumerate}
\item If $\BA$ is symmetric and positive definite and $\BA\Bu = \lambda \Bu$
    for $\Bu\ne \Bzero$, then $\lambda>0$;
\item If $\BA$ is symmetric and all its eigenvalues are positive, 
    then $\BA$ must be positive definite, 
    i.e., $\Bx^T\BA\Bx>0$ for all nonzero $\Bx\in\Rn$.  
    Hint: Write $\Bx$ as a weighted sum of the eigenvectors of $\BA$.
\end{enumerate}
   
\ifthenelse{\boolean{showsols}}{\begin{solution}
Proof.  Suppose that $\BA$ is symmetric and positive definite, and that $\BA\Bu = \lambda\Bu$
for some $\Bu\ne 0$.  Then 
\[ \Bu^*\BA\Bu = \lambda\Bu^*\Bu = \lambda \norm{\Bu}^2.\]
Since $\BA$ is positive definite, $\Bu^*\BA\Bu$, and since $\Bu\ne 0$, $\norm{\Bu}>0$.
Hence can write $\lambda$ as the ratio of two positive numbers,
 \[ \lambda = {\Bu^*\BA\Bu \over \norm{\Bu}^2},\]
which must be positive.

Now suppose that $\BA$ is symmetric and all its eigenvalues are positive.
The Spectral Theorem ensures that we can construct a set of eigenvectors
$\{\Bu_1, \Bu_2, \ldots, \Bu_n\}$ that are orthogonal and each of unit
length, i.e., $\Bu_j^*\Bu_j^{}=1$ and $\Bu_j^*\Bu_k^{}=0$ provided $j\ne k$.
For any $\Bx\in\C^n$, write 
\[ \Bx = \sum_{j=1}^n y_j \Bu_j,\]
i.e., $\Bx = \BU\By$, where $\BU=[\Bu_1\ \Bu_2\ \cdots \Bu_n]$.
In this basis we have
\[ \Bx^*\BA\Bx = (\BU\By)^*\BA(\BU\By) = \sum_{j=1}^n \lambda_j^{} |y_j|^2.\]
Since $\lambda_j>0$ for $j=1,\ldots, n$ by assumption and $|y_j|^2\ge 0$ for all $j=1,\ldots n$,
we have $\Bx^*\BA\Bx \ge 0$.  The only way to achieve $\Bx^*\BA\Bx = 0$ 
is for $|y_j|=0$ for $j=1,\ldots, n$, i.e., when $\By=\Bzero$, and hence $\Bx=\Bzero$.
It follows that if $\Bx\ne \Bzero$, then $\Bx^*\BA\Bx>0$, and hence $\BA$ is positive definite.
\quad\qed

[{\bf GRADERS}: I have used complex arithmetic here, but it is acceptable if 
the solution is restricted to real vectors.]
\end{solution}}{} 
