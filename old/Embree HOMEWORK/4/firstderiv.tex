Consider the operator $L:C_\ell^1[0,1] \to C[0,1]$ defined by
\[ Lu = {du\over dx},\]
where the space $C_\ell^1[0,1]$ imposes a Dirichlet boundary condition
on the left-hand side of the domain,
\[ C_\ell^1[0,1] = \{ u \in C^1[0,1], u(0) = 0\},\]
with the usual inner product
\[ (u,v) = \int_0^1 u(x) v(x)\,dx.\]


\begin{enumerate}
\item Show that $L$ is a linear operator.
\item Is $L$ symmetric?
\item Show that $L$ has \emph{no} eigenvalues, that is, 
      demonstrate that there exist no nonzero $u \in C_\ell^1[0,1]$ 
      and $\lambda\in\C$ for which $Lu = \lambda u$.
\item How would your answer to part~(c) change if we removed the 
      boundary condition by replacing the space $C_\ell^1[0,1]$ with 
      $C^1[0,1]$ ?
\end{enumerate}

\vspace*{2em}
Optional.  Suppose you want to create a discretization of this
operator using the first-order finite difference approximation
\[ u'(x_j) \approx {u(x_j+h) - u(x_j) \over h} = {u_{j+1} - u_j \over h},\]
where $u_j = u(x_j)$ on a grid $x_j = j h$ for $h=1/(N+1)$.
Imposing the equation $u'(x) = \lambda u(x)$ at grid points 
$x_0, \ldots, x_N$ and using the left-boundary condition $u(x_0)=u_0=0$
gives a matrix eigenvalue problem $\BA\Bu = \lambda \Bu$ of the form
\[ {1\over h} \bmatrix{1  \cr -1 & 1 \cr &  \ddots & \ddots \cr & & -1 & 1}
     \bmatrix{u_1 \cr u_2 \cr \vdots \cr u_{N+1}}
  = \lambda
     \bmatrix{u_1 \cr u_2 \cr \vdots \cr u_{N+1}}.\]

What are the eigenvalues of the matrix $\BA$?   \\
How many linearly independent eigenvectors does $\BA$ have? \\
How do you interpret this in light of your answer to problem 4(c)?

%%%%%%%%%%%%%%%%%%%%%%%%%%%%%%%%%%%%%%%%%%%%%%%%%%%%%%%%%%%%%%%%%%%%%%%%%%%%%%%%

\ifthenelse{\boolean{showsols}}{\begin{solution}

\begin{enumerate}
\item Suppose that $u, v \in C_\ell^1[0,1]$ and $\alpha, \beta\in \R$.
      Then 
         \[ L(\alpha u + \beta v) 
            = {d \over dx} (\alpha u + \beta v) 
            = \alpha {du\over dx} + \beta {dv\over dx}
            = \alpha Lu + \beta Lv.\]
       Hence $L$ is a linear operator.
 
\item No, $L$ is \emph{not symmetric}.  To show this, we need only
      exhibit two functions $u, v \in C_\ell^1[0,1]$ such that 
      $(Lu,v) \ne (u,Lv)$.
      For example, take $u(x) = x$ and $v(x) = x^2$.
      Then
         \begin{eqnarray*}
               (Lu, v) &=& \int_0^1 {du \over dx}(x) v(x)\, dx
                           = \int_0^1 x^2 \, dx = \Big[{x^3 \over 3}\Big]_0^1 = 1/3;\\[0.5em]
               (u, Lv) &=& \int_0^1 u(x) {dv \over dx}(x) \, dx
                           = \int_0^1 x (2x) \, dx = \Big[{2x^3 \over 3}\Big]_0^1 = 2/3.
         \end{eqnarray*}
       Hence $(Lu,v) \ne (u,Lv)$, and so $L$ is not symmetric.
    
       {[\textbf{GRADERS:} Please deduct 2~points if either (or both) functions 
         $u$ and $v$ are not in $C_\ell^1[0,1]$.]}

\item Suppose that there exists some function $u \in C_\ell^1[0,1]$ such that
      $L u = \lambda u$ for some $\lambda\in\C$.
      This means that $du/dx = \lambda u$, which can only be satisfied if
               \[ u(x) = c e^{\lambda x}\]
      for some constant $c$.
      However, to have $u\in C_\ell^1[0,1]$, we must have
				  $u(0)=0$, which can only occur if $c=0$ since $e^{\lambda\cdot 0} = 1 \ne 0$
      for all $\lambda\in \C$.  If $c=0$ then $u(x) = 0$ for all $x\in[0,1]$, 
      which violates the requirement that $u$ be nonzero.
      Hence $L$ is an example of a linear operator with no eigenvalues.

\item Now remove the boundary condition, so we consider $Lu = du/dx'$ but on
      the domain $C^1[0,1]$.  As in part~(c), $Lu = \lambda u$ impies that
               \[ u(x) = c e^{\lambda x}.\]
      As this solution $u \in C^1[0,1]$, it is a suitable eigenfunction for
      all nonzero $c$, \emph{regardless of the value of $\lambda$}.
      Hence, all $\lambda \in \C$ are eigenvalues of this operator.

      To answer the question precisely as stated, the answer to part~(c) 
      changes in that we go from having no eigenvalues in part~(c), 
      to having all $\lambda\in\C$ as eigenvalues in part~(d).
      Clearly, the domain on which the operator is posed plays a major
      role in determining the eigenvalues.
      
      [\textbf{GRADERS}: Do not take any points off in the student did not
       mention that $\lambda$ could be complex.] 
\end{enumerate}

Optional part:  the matrix $\BA$ has only one eigenvalue, $\lambda = 1/h$, with
only one linearly independent eigenvector, $\Bv = [0, \ldots, 0, 1]^T$.  
This is consistent with the answer to part~(c), as when $h\to 0$, the eigenvalue
$\lambda = 1/h$ \emph{diverges} to $\infty$:  It does not \emph{converge} to
a finite eigenvalue in the complex plane.  Similarly, the corresponding 
eigenvector does not \emph{converge} to any $C^1[0,1]$ function: it becomes
a better and better approximation to the zero function, with a jump to one
at the right boundary.
\end{solution}}{}

