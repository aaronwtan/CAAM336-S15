The equation $x_1 + x_2 + x_3 = 0$ defines a plane in $\R^3$ that passes through the origin.
\begin{enumerate}
\item Find two linearly independent vectors in $\R^3$ whose span is this plane.
\item Find the point in this plane closest (in the standard Euclidean norm,
$\norm{\Bz} = \sqrt{\Bz^T\Bz}$) to the vector
\[ \Bv = \bmatrix{1\cr 0\cr 1}\]
by formulating this as a best approximation problem.
(You may use MATLAB to invert a matrix.)

\end{enumerate}

%%%%%%%%%%%%%%%%%%%%%%%%%%%%%%%%%%%%%%%%%%%%%%%%%%%%%%%%%%%%%%%%%%%%%%%%%%%%%%%%

\ifthenelse{\boolean{showsols}}{\begin{solution}
\begin{enumerate}
\item  Since two linearly independent vectors determine a plane, 
       we simply need to find two linearly independent vectors 
       that satisfy $x_1 + x_2 + x_3 = 0$.
       One can do this by inspection, for example, and find 
         \[ \bmatrix{1 \cr -1 \cr 0}, \qquad \bmatrix{0 \cr 1 \cr -1}.\]
       However, it would be nice to have an orthogonal basis for this space.
       To do that, pick one vector, say the first vector given above;
       set the second vector to be $(\alpha, \beta, \gamma)^T$.
       We would like the this vector to be in the plane:
        \[ \alpha + \beta + \gamma = 0\]
       and to be orthogonal to the first vector:
         \[ \bmatrix{1 \cr -1 \cr 0}^T \bmatrix{\alpha \cr \beta \cr \gamma } 
                   = \alpha - \beta + 0 = 0.\]
       This gives two equations in three unknowns, which will be satisfied
       if $\beta=\alpha$ and $\gamma = -2\alpha$ for any $\alpha$, 
       i.e., we have the vector
                 \[ \bmatrix{\alpha \cr \alpha  \cr -2\alpha}.\] 
       With $\alpha =1 $, we have two orthogonal vectors whose span is the desired plane:
         \[ \Bx = \bmatrix{1 \cr -1 \cr 0}, \qquad \By = \bmatrix{1 \cr 1 \cr -2}.\]
       
\item  The closest point in the plane to the vector $\Bv$ is found solving
       the usual best-approximation problem matrix equation:
       \[  \bmatrix{ \Bx^T\Bx & \Bx^T \By \cr \By^T\Bx & \By^T\By} 
         \bmatrix{c_1\cr c_2}
        = \bmatrix{ \Bx^T\Bv \cr \By^T \Bv},\]
       that is, 
       \[  \bmatrix{ 2 & 0 \cr 0 & 6} 
         \bmatrix{c_1\cr c_2}
        = \bmatrix{ 1 \cr -1}.\]
       The orthogonality of the vectors $\Bx$ and $\By$ make this an easy problem to solve:
        \[ c_1 = 1/2, \qquad c_2 = -1/6.\]
       Thus, the best approximation to $\Bv = (1, 0, 1)^T$ is the vector
         \[ \widehat{\Bv} = c_1 \Bx + c_2 \By = \bmatrix{1/2-1/6 \cr -1/2-1/6 \cr 1/3} 
                                      = \bmatrix{ 1/3 \cr -2/3 \cr 1/3}.\]
       We can verify this answer by checking 
       (1) that $\widehat{\Bv}$ is in the desired plane: $1/3-2/3+1/3 = 0$, 
        and (2) verifying that the error 
            \[ \Bv - \widehat{\Bv} = \bmatrix{2/3 \cr -2/3 \cr 2/3}\]
        is orthogonal to the two basis vectors $\Bx$ and $\By$ for the plane,
        $(\Bv-\widehat{\Bv})^T\Bx = (\Bv-\widehat{\Bv})^T\By = 0$.
          
\end{enumerate}

\end{solution}}{}

