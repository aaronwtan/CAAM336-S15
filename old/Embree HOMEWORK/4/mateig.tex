Suppose that
  \[ \BA = \bmatrix{ 3 & 0 & 0 \cr 0 & 0 & 1 \cr 0 & 1 & 0}.\]
\begin{enumerate} 
\item Compute \emph{by hand} the eigenvalues and eigenvectors of this matrix.

\item Verify \emph{by hand} that these eigenvectors are orthogonal.

\item Solve the linear system $\BA\Bx = \Bb$ using the spectral method, where
       \[ \Bb = \bmatrix{ 2 \cr -1 \cr 3}\]
\end{enumerate}  

%%%%%%%%%%%%%%%%%%%%%%%%%%%%%%%%%%%%%%%%%%%%%%%%%%%%%%%%%%%%%%%%%%%%%%%%%%%%%%%%

\ifthenelse{\boolean{showsols}}{\begin{solution}

\begin{enumerate}
\item For this matrix $\BA$ we have
\[ \lambda\BI - \BA = \bmatrix{ \lambda-3 & 0 & 0 \cr 0 & \lambda & -1 \cr 0 & -1 & \lambda},\]
and hence the characteristic polynomial is 
\[ \det(\lambda\BI - \BA) = (\lambda-3)(\lambda^2 + 1) = (\lambda-3)(\lambda-1)(\lambda+1).\]
The eigenvalues of $\BA$ are the roots of the characteristic polynomial, which we label
\[ \lambda_1 = -1, \qquad
   \lambda_2 = 1, \qquad
   \lambda_3 = 3.\]

{[\textbf{GRADERS:} Below it is fine if students leave the eigenvectors in 
a general form (e.g., $\Bu_1 = [0, u_2, -u_2]^T$) at this stage, and wait
until part~(c) to pick concrete entries.]}

To compute the eigenvectors associated with each eigenvalue, we look for a 
nonzero vector in the null space of $\lambda_j \BI - \BA$.
\begin{itemize}
\item[$\lambda_1=-1$:]   We seek $\Bu = (u_1, u_2, u_3)^T$ that makes the following vector zero:
\[ (\lambda_1 \BI-\BA) \Bu = \bmatrix{-4 & 0 & 0 \cr 0 & -1 & -1 \cr 0 & -1 & -1}
                             \bmatrix{u_1 \cr u_2\cr u_3}
                           = \bmatrix{-4u_1 \cr -(u_2+u_3) \cr -(u_2+u_3)}.\]
The only way to make this vector zero is to set $u_1 = 0$ and $u_2 = -u_3$.
Thus any vector of the form
\[ \Bu_1 = \bmatrix{ 0 \cr u_2 \cr -u_2}, \quad u_2 \ne 0\]
is an eigenvector associated with the eigenvalues $\lambda_1=-1$.
We select
\[ \Bu_1 = {1\over \sqrt{2}} \bmatrix{0 \cr 1 \cr -1}\]
so that $\norm{\Bu_1}=1$.
\item[$\lambda_2=1$:]  We now seek $\Bu = (u_1, u_2, u_3)^T$ that makes the following vector zero:
\[ (\lambda_2 \BI-\BA) \Bu = \bmatrix{-2& 0 & 0 \cr 0 & 1 & -1 \cr 0 & -1 & 1}
                             \bmatrix{u_1 \cr u_2\cr u_3}
                           = \bmatrix{-2u_1 \cr u_2-u_3 \cr u_3-u_2}.\]
Any vector of the form
\[ \Bu_2 = \bmatrix{ 0 \cr u_2 \cr u_2}, \quad u_2 \ne 0\]
is an eigenvector associated with the eigenvalues $\lambda_2=1$.
We select
\[ \Bu_2 = {1\over \sqrt{2}} \bmatrix{0 \cr 1 \cr 1}.\]

\item[$\lambda_3=3$:]  We now seek $\Bu = (u_1, u_2, u_3)^T$ that makes the following vector zero:
\[ (\lambda_3 \BI-\BA) \Bu = \bmatrix{0& 0 & 0 \cr 0 & 3 & -1 \cr 0 & -1 & 3}
                             \bmatrix{u_1 \cr u_2\cr u_3}
                           = \bmatrix{0 \cr 3u_2-u_3 \cr 3u_3-u_2}.\]
To make the second component zero we need $u_2 = u_3/3$, while to make
the third component zero we need $u_3 = u_2/3$.  The only way to accomplish
both is to set $u_2=u_3=0$.  Thus any vector of the form
\[ \Bu_3 = \bmatrix{ u_1 \cr 0 \cr 0}, \quad u_1 \ne 0\]
is an eigenvector associated with the eigenvalues $\lambda_3=3$.
We select
\[ \Bu_3 = \bmatrix{1 \cr 0 \cr 0}.\]
\end{itemize}

\item One can quickly check that
      \[ \Bu_1^T \Bu_2 = \Bu_2^T\Bu_1 = (1/\sqrt{2})(0\cdot0 + 1\cdot 1 + (-1)\cdot 1) = 0.\]
      \[ \Bu_1^T \Bu_3 = \Bu_3^T\Bu_1 = (0\cdot1 + (1/\sqrt{2})\cdot 0 + (-1/\sqrt{2})\cdot 0) = 0.\]
      \[ \Bu_2^T \Bu_3 = \Bu_3^T\Bu_2 = (0\cdot1 + (1/\sqrt{2})\cdot 0 + (1/\sqrt{2})\cdot 0) = 0.\]

\item The spectral method gives $\Bx$ as a linear combination of the eigenvectors:
         \[ \Bx = \sum_{j=1}^3 {\Bu_j^T\Bb \over \lambda_j} \Bu_j.\]
      We compute
      \begin{eqnarray*}
          \Bu_1^T\Bb &=& 0\cdot 2 + (1/\sqrt{2}) \cdot (-1) + (-1/\sqrt{2}) \cdot 3 = -2\sqrt{2} \\[0.25em]
          \Bu_2^T\Bb &=& 0\cdot 2 + (1/\sqrt{2}) \cdot (-1) + (1/\sqrt{2}) \cdot 3 = \sqrt{2}\\[0.25em]
          \Bu_3^T\Bb &=& 1\cdot 2 + 0 \cdot (-1) + 0 \cdot 3 = 2,
      \end{eqnarray*}
and hence
\[ \Bx =   {-2 \sqrt{2} \over -1} \bmatrix{ 0 \cr 1/\sqrt{2} \cr -1/\sqrt{2}}
         + {\sqrt{2} \over 1} \bmatrix{ 0 \cr 1/\sqrt{2} \cr 1/\sqrt{2}}
         + {2 \over 3} \bmatrix{1 \cr 0 \cr 0}
       =   \bmatrix{2/3 \cr 3\cr -1}.\]
We can multiply $\BA\Bx$ out to verify that the desired $\Bb$ is obtained.
\end{enumerate} 
\end{solution}}{}

