Suppose $u$ represents the temperature distribution in a homogeneous bar,
and assume that both ends are \emph{perfectly insulated} 
and there is no external source of heat.
(As before, the sides of the bars are perfectly insulated, too.)

\begin{enumerate}
\item What will the boundary conditions be at the left and right ends of the bar?\\
      (Is the temperature fixed at the ends, which would give Dirichlet boundary
       conditions?  Can you say something about the heat flux at each boundary,
       which would give Neumann conditions?)

\item Because the bar is perfectly insulated over its entire surface, 
      it is clear from a physical perspective that the total energy 
      in the bar must remain constant.  
      Show that the mathematical model we developed in class 
      captures this fact; that is, write down a formula for the
      energy in the entire bar and show that, in this perfectly 
      insulated case, it is constant in time.
      (Hint: Write the rate of change of the energy in the bar,
         then use the differential equation and boundary conditions
         to show this must be zero.)
      
\end{enumerate}

%%%%%%%%%%%%%%%%%%%%%%%%%%%%%%%%%%%%%%%%%%%%%%%%%%%%%%%%%%%%%%%%%%%%%%%%%%%%%%%%

\ifthenelse{\boolean{showsols}}{\begin{solution}
\begin{enumerate}
\item Since the bar is insulated at its ends, heat does not flow through 
      the ends, i.e., the heat flux is zero: $q(0,t)=q(\ell,t)=0$.  
      Since the material is homogeneous ($\kappa$ is constant),
      this heat flux is proportional to $\d u/\d t$ by
      Fourier's law of heat conduction.  Thus, the insulated bar should
      have homogeneous Neumann conditions:
          \[ {\d u\over \d x}(0,t) = {\d u \over \d x}(1,t) = 0.\]

\item We assume that the temperature in the bar varies slightly from the
      nominal temperature $T_0$, so that we relationship between energy
      and heat capacity is essentially linear.  
      If $E_0$ represents the energy in the bar if the bar had constant
      temperature $T_0$ everywhere,
      then we can express the total energy in the bar as
          \[ E(t) = E_0 + \int_0^\ell \rho c (u(s,t) - T_0)\, ds.\]
      We wish to show that energy is constant in time, i.e., 
      we want to show that the time derivative of $E(t)$ is zero.
      First we note that
          \begin{eqnarray*}
            {d \over dt} E(t) 
                &=& {d \over dt} \big( E_0 - \rho c T_0 \ell\big) +
                       {d \over dt} \int_0^\ell \rho c u(s,t)\, ds \\[0.5em]
                &=& \int_0^\ell \rho c {\d u \over \d t} (s,t)\, ds.
          \end{eqnarray*}
      Since the temperature in the bar obeys the heat equation, we have
             \[ \rho c {\d u \over \d t}(x,t) = \kappa {\d^2 u \over \d x^2} (x,t),\]
      and we can substitute the right hand side in the left hand side of our
      expression for $dE/dt$:
         \[ {d \over dt} E(t) = \int_0^\ell \kappa {\d^2 u \over \d x^2} (s,t)\,ds.\]
      We can integrate the right hand side and use the 
      Fundamental Theorem of Calculus to obtain
         \[ {d \over dt} E(t) =  \kappa \Big({\d u \over \d x} (\ell,t) - {\d u\over \d x}(0,t)\Big).\]
      However, since the bar is insulated we have the Neumann boundary conditions from part~(a), which give
         \[ {d \over dt} E(t) =  \kappa (0 - 0) = 0.\]
      Thus we have verified that our mathematical expression for the total energy 
       agrees with our physical intuition: no energy should escape an insulated bar.
\end{enumerate}
\end{solution}}{}

