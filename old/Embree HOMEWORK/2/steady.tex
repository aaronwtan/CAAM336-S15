Consider a bar of metal alloy manufactured such that its thermal conductivity
is $\kappa(x) = 1+\alpha x$ for constant $\alpha$ and $0\le x\le \ell$.
\begin{enumerate}
\item Determine a general formula for the steady-state temperature distribution 
      of this bar, which should include two free constants.
\item Find formulas for these free constants in the case that the ends of the bar are
      submerged in ice baths of $\gamma$~deg on the left and $\delta$~deg on the right.
\item Now find formulas for the free constants in the case that the left end has a fixed 
      \emph{heat flux} of $\gamma$~{\rm J}/({\rm m}^2\cdot {\rm sec})$ 
      and the right end is submerged in an ice bath of
      $\delta$~deg.
\end{enumerate}


%%%%%%%%%%%%%%%%%%%%%%%%%%%%%%%%%%%%%%%%%%%%%%%%%%%%%%%%%%%%%%%%%%%%%%%%%%%%%%%%

\ifthenelse{\boolean{showsols}}{\begin{solution}
\begin{enumerate}
\item  Because the bar is at steady state, we have
         \[ {\d u \over \d t}(x) = 0,\]
       where we have dropped the $t$ argument from $u(x,t)$, 
      since $u$ is independent of $t$ in this context.
       The heat equation becomes
          \[ 0 = {d \over d x} \Big(\kappa(x) {d \over d x} u(x)\Big).\]
        In other words, the quantity on the right under the first derivative must be a constant.
       Integrate this equation once to obtain
          \[ \kappa(x) {d \over d x} u(x) = C_1,\]
       where the constant $C_1$ will be determined later by the boundary conditions.
       Divide by $\kappa(x)$ to obtain the equation
          \[ {d \over d x} u(x) = {C_1\over \kappa(x)}.\]
       Substituting in the particular equation for this bar, $\kappa(x) = 1+\alpha x$, we have
          \[ {d \over d x} u(x) = {C_1\over 1+\alpha x},\]
       which we can integrate once to obtain the general form of the solution
          \[ u(x) = {C_1 \over \alpha} \log(1+\alpha x) + C_2,\]
       where $C_1$ and $C_2$ are constants.
       (As is common in higher mathematics, 
         we use $\log$ to denote the natural logarithm rather than $\ln$.)
  
\item  If the ends are submerged in ice baths of $\gamma$ and $\delta$ degrees, we have
       the Dirichlet boundary conditions
          \[ u(0) = \gamma, \qquad u(\ell) = \delta.\]
        We must find $C_1$ and $C_2$ to satisfy these conditions.
        This gives two equations in two unknowns:
        \begin{eqnarray*}
             \gamma &=& {C_1\over \alpha} \log(1+\alpha \cdot 0) + C_2 \\[0.5em]
             \delta &=& {C_1\over \alpha} \log(1+\alpha \ell) + C_2.
         \end{eqnarray*}
        Since $\log(1+\alpha\cdot 0) = \log(1) = 0$, the first equation reduces to
           \[ \gamma = C_2.\]
        Substituting this formula into the second equation and solving for $C_1$ yields
           \[ C_1 = {\alpha(\delta - \gamma) \over \log (1+\alpha \ell)}.\]
        Thus, the steady state solution with desired Dirichlet boundary conditions is 
           \[ u(x) = (\delta-\gamma) {\log (1+\alpha x) \over \log(1+\alpha \ell)} + \gamma.\]

\item  If the heat flux is fixed at $\gamma$ deg/sec on the left end of the bar,
        we have  \[ q(0) = \gamma.\]
        Fourier's law of heat conduction gives
           \[ q(0) = -\kappa(0) {\d u \over \d x}(x,t),\]
        and hence we have the Neumann boundary condition on the left end:
           \[ {\d u \over \d x}(0) = -{\gamma \over \kappa(0)} = -{\gamma \over 1} = -\gamma.\]
        [\textbf{GRADERS}: I did not emphasize this point, so please do not deduct points if
         students immediately assumed that $\d u(0)/\d x = \gamma$ without appealing to Fourier's law
         of heat conduction.  
        If students instead assumed that $\d u(0)/\d t$ = 0, please only deduct 5~points.
        Students should have seen that this does not provide enough information to solve the problem;
        however, the correct units for heat flux are ${\rm J}/({\rm m}^2\cdot {\rm sec})$, 
          rather than those of ${\rm deg}/{\rm sec}$ given in the problem.]

        The ice bath on the right hand side gives the Dirichlet condition
           \[ u(\ell) = \delta.\]
        To impose the Neumann condition, we must take a derivative of the general solution
          \[ u(x) = {C_1 \over \alpha} \log(1+\alpha x) + C_2,\]
        or just go back to the intermediate step in part~(a), to obtain
          \[ {\d u \over \d x}(x) = {C_1 \over 1+\alpha x},\]
        and so
          \[ {\d u \over \d x}(0) = {C_1}.\]
        Thus our boundary conditions again impose two equations in two unknowns:
        \begin{eqnarray*}
            -\gamma &=& C_1 \\[0.5em]
             \delta &=& {C_1\over \alpha} \log(1+\alpha \ell) + C_2.
         \end{eqnarray*}
         Substituting $C_1 = -\gamma$ into the second equation and solving for $C_2$ gives
           \[ C_2 = \delta + {\gamma \over \alpha} \log(1+\alpha \ell).\]
         With these constants, the solution becomes
           \begin{eqnarray*}
               u(x) &=& {\gamma \over \alpha} \big(\log(1+\alpha \ell) - \log (1+\alpha x)\big) + \delta \\[0.5em]
                    &=& {\gamma \over \alpha} \log\Big({1+\alpha \ell \over 1+\alpha x}\Big) + \delta  
           \end{eqnarray*} 
\end{enumerate}
\end{solution}}{}

