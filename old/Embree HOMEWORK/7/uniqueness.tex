\def\uhat{\widehat{u}}
In this problem you shall demonstrate that the solution to the
weak problem must be \emph{unique} by completing the following 
procedure.

Let $V$ be a vector space endowed with an energy inner product 
$a(\cdot, \cdot)$ and a second inner product $(\cdot,\cdot)$.
For some fixed $f$, suppose that $u$ and $\uhat$ are two 
members of $V$ that satisfy the weak problem, i.e.,
\[ a(u, v) = (f,v) \qquad  \mbox{for all $v\in V$}\]
and
\[ a(\uhat, v) = (f,v) \qquad  \mbox{for all $v\in V$}.\]
Use properties of $a(\cdot, \cdot)$ to conclude that $u=\uhat$.

%%%%%%%%%%%%%%%%%%%%%%%%%%%%%%%%%%%%%%%%%%%%%%%%%%%%%%%%%%%%%%%%%%%%%%%%%%%%%%%%

\ifthenelse{\boolean{showsols}}{\begin{solution}

Since $a(u,v) = (f,v)$ and $a(\uhat,v) = (f,v)$, 
for all $v\in V$ we have
\[ a(u,v) - a(\uhat,v) = 0,\]
which implies that
\[ a(u-\uhat,v) = 0\]
for all $v\in V$.
Both $u$ and $\uhat$ are in the vector space $V$, so $u-\uhat$ is also in $V$.
Set $v=u-\uhat$ to find
\[ a(u-\uhat,u-\uhat) = 0.\]
Since $a(\cdot,\cdot)$ is an inner product, we can only have
$a(u-\uhat,u-\uhat) = 0$ if $u-\uhat = 0$, i.e., $u=\uhat$.

{[\textbf{GRADERS}: The observation that you can take $v=u-\uhat$
(or equivalent) is worth 10~points.]}

\end{solution}}{}

