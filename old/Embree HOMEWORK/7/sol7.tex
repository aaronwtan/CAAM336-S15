\documentclass[10pt]{article}
\usepackage{fullpage,336notes,ifthen,color}
\def\hot#1{\color{red}{#1}}
\parindent0em\parskip1em
\newboolean{showsols}
%\textheight10in
\pagestyle{empty}
\newenvironment{solution}{\vspace*{1em} \hrule\vspace*{0.5em} {\sf Solution.} }
                         {\vspace*{0.75em}\hrule}

\begin{document}
\setboolean{showsols}{true}

\begin{center}
\large \textsf{\textbf{CAAM 336 $\cdot$ DIFFERENTIAL EQUATIONS}\\[0.5em]
 Problem Set 7 \ifthenelse{\boolean{showsols}{$\cdot$ Solutions}}{}}
\end{center}

Posted Wednesday 3 October 2012.  Due Wednesday 10 October 2012, 5pm.


\ifthenelse{\boolean{showsols}}{}{\begin{center}\textbf{\emph{Please write the name of your college on your paper.}}\end{center}}

\begin{center}
This problem set counts for 75~points, i.e., three-quarters the value 
of the earlier problem sets.\\
Because of the Centennial, late papers are due at 5pm on Monday, 15 October.
\end{center}

General advice: You may compute any integrals you encounter using symbolic mathematics 
tools such as WolframAlpha, Mathematica, or the Symbolic Math Toolbox in MATLAB.


\begin{enumerate}
\item {[50 points: 20 points for (a); 10 points each for (b), (c), (d)]} 

\input fem

\vspace*{1em}
\item {[25 points: 7 points for (a); 5 points for (b); 3 points for (c); 10 points for (d)]}\\  \input delta
\end{enumerate}

\end{document}
