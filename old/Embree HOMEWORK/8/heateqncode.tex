{\footnotesize \begin{verbatim}
% Plot the expansion of the initial data, psi(x)

 x = linspace(0,1,200);
 col = 'krbm';
 figure(1), clf
 fm = zeros(size(x));
 for n=0:2:80
    if n==0, an0 = 2/3;                            % psi(x) = 1 for x in [0,1/3], [2/3,1]; 
    else, an0 = sqrt(2)*(sin(n*pi/3)-sin(2*n*pi/3))/(n*pi);% psi(x) = 0 otherwise. 
    end
    if n==0, fm = an0*ones(size(fm));
    else, fm = fm + an0*sqrt(2)*cos(n*pi*x);
    end
    if ismember(n,[ 0 2 4 80]), 
       plot(x, fm, '-','linewidth',2,'color',col(1)), hold on, col = col(2:end);
    end
 end 
 legend('m=0', 'm=2','m=4','m=80',3)
 set(gca,'fontsize',16)
 xlabel('x'), ylabel('\psi_m(x)')
 print -depsc2 heateqn1 

% Compute the solution at at various times.

 psi = (x <= 1/3) | (x >= 2/3);    %  initial condition
 U = [psi];
 col = 'krbmc';
 figure(2), clf
 plot(x, psi, 'linewidth',2,'color',col(1)), hold on, col = col(2:end);
 t = .002:.002:0.1;
 tprint = [.002 .05 0.1];
 for j=1:length(t)
    for n=0:2:20
       if n==0, 
          an0    = 2/3;
          lambda = 0;
          uj     = exp(-lambda*t(j))*an0*ones(size(x)); 
       else
          an0 = sqrt(2)*(sin(n*pi/3)-sin(2*n*pi/3))/(n*pi);
          lambda = n^2*pi^2;
          uj = uj + exp(-lambda*t(j))*an0*(sqrt(2)*cos(n*pi*x));
       end
    end 
    U = [U;uj];
    if ismember(t(j),tprint), 
       plot(x, uj, '-','linewidth',2,'color',col(1)), hold on, col = col(2:end);
    end
 end
 legend('t=0','t=.002','t=.05','t=.1',3)
 set(gca,'fontsize',16)
 xlabel('x'), ylabel('u(x,t)')
 print -depsc2 heateqn2 

 figure(3), clf
 plt = waterfall(x,[0 t],U);
 set(plt,'edgecolor','k')       % make the lines black

 view(-30,50)
 set(gca,'fontsize',14)
 xlabel('x'), ylabel('t'), zlabel('u(x,t)')
 print -depsc2 heateqn3

\end{verbatim}}
