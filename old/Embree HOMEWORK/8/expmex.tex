Consider the following three matrices:
\[ {\rm (i)}\quad \BA = \bmatrix{0 & 1 \cr 1 &  0} \qquad
   {\rm (ii)}\quad \BA = \bmatrix{-50 & 49 \cr 49 & -50}\qquad
   {\rm (iii)}\quad \BA = \bmatrix{0 & 1 \cr -1 & 0}.\]

\begin{enumerate}
\item For each of the matrices (i)--(iii), compute the matrix exponential $e^{t\BA}$.\\[.25em]
      You may use \verb|eig| for the eigenvalues and eigenvectors, 
      but please construct the matrix exponential ``by hand'' (not with \verb|expm|).
      For diagonalizable $\BA = \BV\BLambda\BV^{-1}$, recall the formula 
      $e^{t\BA} = \BV e^{t\BLambdas} \BV^{-1}$.  If you encounter a complex
      eigenvalue $\lambda = \alpha + i \beta$, you may use the formula 
      \[ e^\lambda = e^{\alpha+ i \beta} = e^\alpha (\cos(\beta) + i \sin(\beta)).\]

\item Use your answers in part~(a) to explain the behavior of solutions $\Bx(t)$
      to the differential equation $\Bx'(t) = \BA\Bx(t)$ as $t\to\infty$,
       given that $\Bx(0) = [2,\ 0]^T$ 
      (e.g., specify and explain exponential growth, exponential decay, or neither) for each
      of the three matrices (i)--(iii).  

\item For the matrix~(ii), describe how large one can choose the time step 
      $\Delta t$ so that the forward Euler method applied to $\Bx'(t) = \BA\Bx(t)$,
      \[ \Bx_{k+1} = \Bx_k + \Delta t\BA\Bx_k,\]
will produce a solution that qualitatively matches the behavior
of the true solution (i.e., the approximations $\Bx_k$ should grow, decay,
or remain of the same size as the true solution does).

Answer the same question for the backward Euler method
\[ \Bx_{k+1} = \Bx_k + \Delta t\BA\Bx_{k+1}.\]

\item For the matrix in (iii), describe how the forward Euler method 
      behaves \emph{for all $\Delta t$} as $k\to \infty$ for $\Bx(0) = [1, 1]^T$.  
      Now describe how the backward Euler method must behave as $k\to\infty$
      for the same matrix and initial condition.
\end{enumerate}
%%%%%%%%%%%%%%%%%%%%%%%%%%%%%%%%%%%%%%%%%%%%%%%%%%%%%%%%%%%%%%%%%%%%%%%%%%%%%%%%

\ifthenelse{\boolean{showsols}}{\begin{solution}

[GRADERS: it is acceptable for students to use MATLAB to compute eigendecompositions,
but they should not simply use the {\tt expm} command.  In particular, only give half
credit if students computed $e^{t \BA}$ for a fixed value of $t$.  The correct answer
should depend on the variable $t$.]

\begin{enumerate}
\item We compute the matrix exponentials for each matrix in turn.
\begin{enumerate}
\item[(i)] Note that ${\rm det}(\lambda \BI-\BA) = \lambda^2-1 = (\lambda+1)(\lambda-1)$
and hence the eigenvalues of $\BA$ are $\lambda_1 = -1$ and $\lambda_2 = 1$.
The corresponding (normalized) orthogonal eigenvectors are
\[ \Bu_1 = {\sqrt{2}\over 2} \bmatrix{1 \cr -1}, 
   \qquad
   \Bu_2 = {\sqrt{2}\over 2} \bmatrix{1 \cr 1}.\]
As $\BA$ is symmetric,
if we set $\BU = [\Bu_1\ \Bu_2]$ and $\BLambda = {\rm diag}(\lambda_1, \lambda_2)$,
we have $\BA = \BU\BLambda\BU^*$ and
\[ e^{t\BA} = \BU e^{t\BLambda} \BU^*
            = \BU \bmatrix{e^{-t} & 0 \cr 0 & e^{t}} \BU^*.\]
Multiplying this out gives
\[ e^{t\BA} = {\textstyle{1\over2}} 
              \bmatrix{e^t+e^{-t} & e^t - e^{-t} \cr e^t-e^{-t} & e^{t}+e^{-t}}.\]

\item[(ii)] If we denote the matrix in part~(a) as $\BA_1$, then we find that
      the $\BA$ in part~(b) can be written as
      $\BA = -50\BI + 49\BA_1$,
      from which it follows that $\BA$ has eigenvalues $\lambda_1 = -99$ 
      and $\lambda_2 = -1$ with the same eigenvectors as in part~(a):
\[ \Bu_1 = {\sqrt{2}\over 2} \bmatrix{1 \cr -1}, 
   \qquad
   \Bu_2 = {\sqrt{2}\over 2} \bmatrix{1 \cr 1}.\]
      Again $\BA$ is symmetric, and we have that
      \[ e^{tA} = \BU \eop^{t\BLambda} \BU^*
      = {\textstyle{1\over2}} 
              \bmatrix{e^{-t}+e^{-99t} & e^{-t} - e^{-99t} \cr e^{-t}-e^{-99t} & e^{-t}+e^{-99t}}.\]

\item[(iii)] {[GRADERS: please be a bit more lenient with this problem, as 
       this $\BA$ is nonsymmetric, a case we didn't dwell excessively on in class.]}

      The characteristic polynomial is 
      ${\rm det}(\lambda\BI-\BA) = \lambda^2+1 = (\lambda-i)(\lambda+i)$, 
      where $i^2 = -1$.  Hence the eigenvalues are $\lambda_1 = -i$
      and $\lambda_2 = i$.  The corresponding normalized eigenvectors are 
\[ \Bv_1 = {\sqrt{2}\over 2} \bmatrix{1 \cr -i}, 
   \qquad
   \Bv_2 = {\sqrt{2}\over 2} \bmatrix{1 \cr i}.\]
      Since $\BA$ is not symmetric we write $\BA = \BV\BLambda\BV^{-1}$, where
      $\BV = [\Bv_1\ \Bv_2]$ and $\BLambda = {\rm diag}(\lambda_1, \lambda_2)$,
      and the matrix exponential takes the form
      \[ e^{t\BA} = \BV e^{t\BLambdas} \BV^{-1}
                  = {\textstyle{1\over2}}\bmatrix{e^{it} + e^{-it} & i (e^{-it} - e^{it})
                                         \cr i(e^{it}-e^{-it}) & e^{it}-e^{-it}}.\]
      Note that for real numbers $t$, 
          \[ e^{i t} =  \cos(t) + i@\sin(t)\]
      and
          \[ e^{-i t} =  \cos(-t) + i@\sin(-t) = \cos(-t) - i@\sin(t),\]
      and hence one could arrive at the simple formula (not required):
      \[ e^{t\BA} = \bmatrix{ \cos(t) & \sin(t) \cr -\sin(t) & \cos(t)}.\]

      Alternatively, one can note that $\BA^2 = -\BI$, $\BA^3 = -\BA$, $\BA^4 = \BI$, $\ldots$,
      to obtain from the series expression 
       \[ e^{t\BA} = \BI + t\BA + {\textstyle{1\over2}} t^2\BA^2 + 
                                  {\textstyle{1\over3!}} t^3\BA^3 + \cdots\]
      that
       \[ e^{t\BA} = \left[\begin{array}{cc}
                        1 - {1\over2}t^2 + {1\over 4!}t^4 - {1\over 6!}t^6 + \cdots  & 
                        t - {1\over3!}t^3 + {1\over 5!}t^5 - {1\over 7!}t^7 + \cdots \\[0.5em]
                        -t + {1\over3!}t^3 - {1\over 5!}t^5 + {1\over 7!}t^7 - \cdots &
                        1 - {1\over2}t^2 + {1\over 4!}t^4 - {1\over 6!}t^6 + \cdots
                     \end{array}\right]
                   = \bmatrix{\cos(t) & \sin(t) \cr -\sin(t) & \cos(t)}.\]
       Here we have spotted that the entries in this matrix 
       are the Taylor series for $\sin(t)$ and $\cos(t)$.
    
\end{enumerate}

\item The behavior of $\Bx(t) = e^{t\BA}\Bx(0)$ for $\Bx(0) = [2, 0]^T$ depends on the matrix.
\begin{enumerate}
\item[(i)] For the specified initial condition, we have
           \[ \Bx(t) = e^{t\BA} \Bx(0) = \bmatrix{e^t+e^{-t} \cr e^t-e^{-t}}.\]
Thus, as $t\to\infty$, the solution $\Bx(t)$ blows up.
(In fact, it behaves like $e^t [1\ 1]^T$.)
\item[(ii)] For the given $\Bx(0)$, we have
\[ \Bx(t) = e^{t\BA} \Bx(0) = \bmatrix{e^{-t}+e^{-99t} \cr e^{-t}-e^{-99t}},\]
     and hence $\Bx(t)\to \Bzero$ as $t\to\infty$.
     This must be true since both eigenvalues of $\BA$ are negative.
\item[(iii)] Notice that
          \[  \Bx(t) = e^{t\BA}\Bx(0) = 2 \bmatrix{\cos(t) \cr -\sin(t)},\]
       so $\Bx(t)$ neither grows nor decays.  
       (In fact, $\norm{\Bx(t)}$ is constant!)
\end{enumerate}
\item The eigenvalues for the matrix given by~(ii) are 
$\lambda_1 = -99$ and $\lambda_2 = -1$.
Thus the solution to the equation $d\Bx/dt = \BA\Bx$ 
will decay to zero as $t\to \infty$ for all choices 
of initial condition $\Bx(0)$.

We wish to choose the step size $\Delta t$ for the forward Euler
method so that the iterates $\Bx_k$ decay to zero as $k\to\infty$.
For this equation
\[ \Bx_k = (\BI+\Delta t \BA)^k \Bx_0,\]
and so we need all eigenvalues of the symmetric matrix 
$\BI+\Delta t \BA$ to be less than one in magnitude.
The eigenvalues of $\BI+\Delta t\BA$ are simply
\[ \mu_1 = 1+\Delta t \lambda_1 = 1-99\Delta_t, \qquad
   \mu_2 = 1+\Delta t \lambda_2 = 1-\Delta_t.\]
For all $0<\Delta_t < 2$ we have $|\mu_2|<1$, 
but to get $|\mu_1|<1$ we have a stricter requirement:
\[  0 < \Delta t < 2/99.\]
(Alternatively, one can simply look for $\Delta t$ 
such that $\Delta t \lambda_1, \Delta t \lambda_2 \in (-2,0)$.)

For the backward Euler method, we have
\[ \Bx_k = (\BI-\Delta t \BA)^{-k} \Bx_0,\]
and we need all eigenvalues of $(\BI-\Delta t\BA)^{-1}$
to be less than one in magnitude.  Those eigenvalues are
\[ \mu_1 = {1\over 1-\Delta t \lambda_1} = {1\over 1+99\Delta t}, \qquad
   \mu_2 = {1\over 1-\Delta t \lambda_2} = {1\over 1+\Delta t}.\]
These values are less than one in magnitude for all $\Delta t>0$,
so there is no restriction on $\Delta t$ to obtain $\Bx_k \to 0$ as $k\to\infty$.

\item The diagonalizable matrix $\BA$ given in (iii) has eigenvalues $\lambda_\pm = \pm i$.
      It follows that the forward Euler iterations, given by
       \[ \Bx_k = (\BI + \Delta t\BA)^k \Bx(0)\]
      will behave as $k\to\infty$ like eigenvalues of $(\BI+\Delta t\BA)^k$, i.e.,
      like $(1 + \Delta t \lambda_\pm)^k$. 
      Since
      \[ |1 + \Delta t \lambda_\pm| = |1 \pm i \Delta t| = \sqrt{1+(\Delta t)^2} > 1,\]
      we conclude that the forward Euler iterates will always blow up as $k\to\infty$
      \emph{for any choice of $\Delta t > 0$}.

      On the other hand, the backward Euler iterates,
       \[ \Bx_k = (\BI - \Delta t\BA)^{-k} \Bx(0)\]
      will behave as $k\to\infty$ like eigenvalues of $(\BI-\Delta t\BA)^{-k}$, i.e.,
      like $(1 - \Delta t \lambda_\pm)^{-k}$.  Since 
      \[ \Big|{1\over 1 - \Delta t \lambda_\pm}\Big| 
       = {1 \over |1 \mp i \Delta t|} = {1\over \sqrt{1+(\Delta t)^2}} < 1,\]
      we conclude that the backward Euler iterates will always decay as $k\to\infty$
      \emph{for any choice of $\Delta t > 0$}.
\end{enumerate}

\end{solution}}{}

