Describe how to solve the heat equation
\[ u_t(x,t) = u_{xx}(x,t) + f(x,t), \qquad 0<x<1, \quad t\ge 0 \]
with \emph{inhomogeneous} Neumann boundary conditions
\[ u_x(0,t) = \alpha, \qquad u_x(1,t) = \beta\]
and initial condition $u(x,0) = u_0(x)$.

%%%%%%%%%%%%%%%%%%%%%%%%%%%%%%%%%%%%%%%%%%%%%%%%%%%%%%%%%%%%%%%%%%%%%%%%%%%%%%%%

\ifthenelse{\boolean{showsols}}{\begin{solution}
\def\uhat{\widehat{u}}
\def\fhat{\widehat{f}}
We shall attempt to write the solution $u(x)$ in the form
\[ u(x,t) = \uhat(x,t) + v(x),\]
where $v(x)$ is some function that we shall construct 
that satisfies the boundary conditions
\[ v'(0) = \alpha, \qquad
   v'(1) = \beta,\]
and $\uhat(x,t)$ is some function we shall determine by 
solving the heat equation with homogeneous Neumann boundary conditions:
\[ \uhat_x(0,t) = 0, \qquad
   \uhat_x(1,t) = 0.\]

There are various ways to arrive at the function $v(x)$.
Some students noticed that since we know $v'(x)$ at two points,
we can define $v'(x)$ to be the line that passes through $(0,\alpha)$ and $(1,\beta)$, i.e.,
\[ v'(x) = \alpha + (\beta-\alpha)x.\]
Integrate this polynomial to get
\[ v(x) = \alpha x + {\textstyle{1\over2}}(\beta-\alpha)x^2 +C\]
for any constant $C$.  Taking $C=0$ gives the cleanest form:
\[ v(x) = \alpha x + {\textstyle{1\over2}}(\beta-\alpha)x^2.\]
Given this formula for $v(x)$, we must determine the constraints on $\uhat(x,t)$.
Since $u(x,t) = \uhat(x,t) + v(x)$ is to solve the differential equation $u_t = u_{xx}+f$,
we consider:
\[ u_t(x,t) = \uhat_t(x,t) + v_t(x) = \uhat_t(x,t),\]
since $v(x)$ is independent of $t$, and
\begin{eqnarray*} 
    u_{xx}(x,t) + f(x,t) &=& \uhat_{xx}(x,t) + v_{xx}(x) + f(x,t) \\
                         &=& \uhat_{xx}(x,t) + (\beta-\alpha) + f(x,t) \\
                         &=& \uhat_{xx}(x,t) + \fhat(x,t),
\end{eqnarray*}
where we have defined
\[ \fhat(x,t) = \beta - \alpha + f(x,t).\]
Equating the expressions for $u_t$ and $u_xx+f$, we arrive at the partial differential equation
\[ \uhat_t(x,t) = \uhat_{xx}(x,t) + \fhat(x,t).\]
To fully specify this PDE, we must determine the boundary conditions and initial condition.
At the boundaries, we want
\begin{eqnarray*}
    \alpha &=& u_x(0,t) = \uhat_x(0,t) + v_x(0) = \uhat_x(0,t) + \alpha \\[0.5em]
    \beta  &=& u_x(1,t) = \uhat_x(1,t) + v_x(1) = \uhat_x(1,t) + \beta,
\end{eqnarray*}
so we conclude that
\[ \uhat_x(0,t) = \uhat_x(1,t) = 0,\]
as we were intending (since we already know how to solve equations 
with \emph{homogeneous} Neumann conditions). 

What can be said of the initial condition? We want
\[ u_0(x) = u(x,0) = \uhat(x,0) + v(x),\]
and so we arrive at the initial condition that $\uhat$ must satisfy:
\[ \uhat(x,0) = u_0(x)-v(x).\]

[GRADERS: A complete solutions must specify four essential ingredients:
(i) the function $v$; (ii) the PDE for $\uhat$ with the correct $\fhat$ term;
(iii) the boundary conditions for $\uhat$; (iv) the correct shifted initial
conditions for $\uhat$.  Please deduct 5~points each for any of these ingredients
that are missing.]
\end{solution}}{}

