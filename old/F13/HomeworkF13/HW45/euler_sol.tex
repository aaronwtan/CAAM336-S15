\begin{solution}
\begin{enumerate}
\item {[10 points]} The matrix $\BA$ has eigenvalues $\lambda_1 = -99$ and $\lambda_2 = -1$ and eigenvectors
\[
\Bv_1 = \left[\begin{array}{c}{1\over\sqrt{2}} \\ -{1\over\sqrt{2}}\end{array}\right]
\]
and
\[
\Bv_2 = \left[\begin{array}{c}{1\over\sqrt{2}} \\ {1\over\sqrt{2}}\end{array}\right]
\]
which are such that $\BA\Bv_1=\lambda_1\Bv_1$, $\BA\Bv_2=\lambda_2\Bv_2$, $\Bv_1\cdot\Bv_1=\Bv_2\cdot\Bv_2=1$,  and $\Bv_1\cdot\Bv_2=\Bv_2\cdot\Bv_1=0$. Since $\BA=\BA^T$, if we set
\[
\BV = [\Bv_1\ \Bv_2]=\left[\begin{array}{cc}{1\over\sqrt{2}} & {1\over\sqrt{2}} \\ -{1\over\sqrt{2}} & {1\over\sqrt{2}}\end{array}\right]
\]
and
\[
\BLambda = \left[\begin{array}{cc}\lambda_1 & 0 \\ 0 & \lambda_2\end{array}\right]=\left[\begin{array}{cc}-99 & 0 \\ 0 & -1\end{array}\right]
\]
then we have that
\[
\BA = \BV\BLambda\BV^T
\]
and
\begin{eqnarray*}
e^{t\BA}&=&\BV e^{t\BLambda} \BV^T
\\
&=&\left[\begin{array}{cc}{1\over\sqrt{2}} & {1\over\sqrt{2}} \\ -{1\over\sqrt{2}} & {1\over\sqrt{2}}\end{array}\right] \left[\begin{array}{cc}e^{-99t} & 0 \\ 0 & e^{-t}\end{array}\right] \left[\begin{array}{cc}{1\over\sqrt{2}} & -{1\over\sqrt{2}} \\ {1\over\sqrt{2}} & {1\over\sqrt{2}}\end{array}\right]
\\
&=&\left[\begin{array}{cc}{1\over2}\left(e^{-t}+e^{-99t}\right) & {1\over2}\left(e^{-t} - e^{-99t}\right) \\ {1\over2}\left(e^{-t}-e^{-99t}\right) & {1\over2}\left(e^{-t}+e^{-99t}\right)\end{array}\right].
\end{eqnarray*}
Hence,
\[
\Bx(t) = e^{t\BA} \Bx_0 = \left[\begin{array}{cc}{1\over2}\left(e^{-t}+e^{-99t}\right) & {1\over2}\left(e^{-t} - e^{-99t}\right) \\ {1\over2}\left(e^{-t}-e^{-99t}\right) & {1\over2}\left(e^{-t}+e^{-99t}\right)\end{array}\right]\left[\begin{array}{c}2 \\ 0\end{array}\right] = \left[\begin{array}{c}e^{-t}+e^{-99t} \\ e^{-t}-e^{-99t}\end{array}\right].
\]

\item {[5 points]} Since all of the eigenvalues of $\BA$ are negative, $\norm{\Bx(t)}\rightarrow0$ as $t\rightarrow\infty$.

\item {[5 points]} Now,
\[
\Bx_k = (\BI+\Delta t \BA)^k \Bx_0.
\]
Moreover, the eigenvalues of $\BI+\Delta t \BA$ are $1+\Delta t \lambda_1 = 1-99\Delta t$ and $1+\Delta t \lambda_2 = 1-\Delta t$ and
\[
\BI+\Delta t \BA=\BV \left[\begin{array}{cc}1-99\Delta t & 0 \\ 0 & 1-\Delta t\end{array}\right] \BV^{-1}.
\]
Hence, since all of the eigenvalues of $\BA$ are negative, and $-99<-1$ we can conclude that if
\[
\Delta t<{2\over 99}
\]
then $\norm{\Bx_k}\rightarrow0$ as $k\to\infty$.

\item {[5 points]} Now,
\[
\Bx_k = ((\BI-\Delta t \BA)^{-1})^k \Bx_0.
\]
Moreover, the eigenvalues of $(\BI-\Delta t\BA)^{-1}$ are ${1\over 1-\Delta t \lambda_1} = {1\over 1+99\Delta t}$ and ${1\over 1-\Delta t \lambda_2} = {1\over 1+\Delta t}$ and
\[
(\BI-\Delta t\BA)^{-1}=\BV \left[\begin{array}{cc}{1\over 1+99\Delta t} & 0 \\ 0 & {1\over 1+\Delta t}\end{array}\right] \BV^{-1}.
\]
Hence, since all of the eigenvalues of $\BA$ are negative, we can conclude that there is no restriction on $\Delta t$ to obtain $\norm{\Bx_k}\rightarrow0$ as $k\to\infty$.
\end{enumerate}
\end{solution}