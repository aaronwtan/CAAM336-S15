Consider the temperature function
\[ u(x,t) = e^{-\kappa \theta^2 t/(\rho c)} \sin(\theta x) \]
for constant $\kappa$, $\rho$, $c$, and $\theta$.
\begin{enumerate}
\item Show that this function $u(x,t)$ is a solution of the homogeneous heat equation
 \[  \rho c {\d u \over \d t} = \kappa {\d^2 u\over \d x^2}, 
       \quad \mbox{for $0<x<\ell$ and all $t$.}\]

\item For which values of $\theta$ will $u$ satisfy 
      homogeneous Dirichlet boundary conditions at $x=0$ and~$x=\ell$?

\item Suppose $\kappa = 2.37\;\mbox{W/(cm K)}$,
      $\rho = 2.70\;\mbox{g/cm}^3$, and $c = 0.897\; \mbox{J/(g K)}$
      (approximate values for aluminum found on Wikipedia),
      and that the bar has length $\ell = 10\;\mbox{cm}$.
      Let $\theta$ be such that $u(x,t)$ satisfies
      homogeneous Dirichlet boundary conditions as in part~(b)
      and $u(x,t)\ge 0$ for $0\le x\le \ell$ and all $t$.
 
\vspace*{.25em} 
      Use MATLAB to plot the solution $u(x,t)$ for $0\le x\le \ell$
      and time $0\le t\le 20~{\rm sec}.$\\
      You may choose to do this in one of the following ways:
        (1) Plot the solution for $0\le x\le \ell$ at times
         $t = 0, 4, 8, \ldots, 20$ sec., superimposing all
         six plots on the same axis (helpful commands:
         \verb|linspace|, \verb|plot|, \verb|hold on|);
        (2) Create a three-dimensional plot of the data 
         using \verb|surf|, \verb|mesh|, or \verb|waterfall|.
        In either case, be sure to produce an attractive,
        well-labeled plot.

\end{enumerate}

%%%%%%%%%%%%%%%%%%%%%%%%%%%%%%%%%%%%%%%%%%%%%%%%%%%%%%%%%%%%%%%%%%%%%%%%%%%%%%%%


\ifthenelse{\boolean{showsols}}{\input checksol_sol}{}

