Let N be a positive integer, let $\displaystyle{h={1 \over N+1}}$ and let $x_k=kh$ for $k=0,1,\ldots,N+1$. Let the continuous piecewise linear hat functions $\phi_j\in C[0,1]$ be such that
\[
\phi_0(x) = \left\{\begin{array}{ll}
\displaystyle{x_1-x \over h} & \mbox{if }x\in[x_0,x_1),\\
0& \mbox{otherwise},
\end{array}\right.
\]
and
\[
\phi_j(x) = \left\{\begin{array}{ll}
\displaystyle{{(x-x_{j-1}) \over h}} & \displaystyle{\mbox{if }x\in[x_{j-1},x_j)},
\\[10pt]
\displaystyle{{(x_{j+1}-x) \over h}} & \displaystyle{\mbox{if }x\in[x_j,x_{j+1})},
\\[10pt]
0 & \mbox{otherwise},
\end{array}\right.
\]
for $j=1,\ldots,N$ and
\[
\phi_{N+1}(x) = \left\{\begin{array}{ll}
\displaystyle{x-x_N \over h} & \mbox{if }x\in[x_N,x_{N+1}],\\
0& \mbox{otherwise.}
\end{array}\right.
\]
Let $H^1_D\left(0,1\right)=\{v\in H^1\left(0,1\right):\,v(1)=0\}$. Let the inner product $\ip{\cdot,\cdot}:L^2\left(0,1\right)\times L^2\left(0,1\right)\to\R$ be defined by
\[
\ip{v,w}=\int_0^1v(x)w(x)\,dx
\]
and let the symmetric bilinear form $a\ip{\cdot,\cdot}:H^1\left(0,1\right)\times H^1\left(0,1\right)\to\R$ be defined by
\[
a\ip{v,w}=\int_0^1v'(x)w'(x)\,dx.
\]
Also, let the norm $\left|\left|\left|\cdot\right|\right|\right|:H^1_D\left(0,1\right)\to\R$ be defined by
\[
\left|\left|\left|v\right|\right|\right|=\sqrt{a\ip{v,v}}.
\]
Let $u\in C^2[0,1]$ be the solution of
\begin{eqnarray*}
 -u''(x) &=& f(x),\quad 0<x<1;
\\
 u'(0) &=& \alpha;
\\
u(1) &=& \beta;
\end{eqnarray*}
where $f\in C[0,1]$ and $\alpha,\beta\in\R$. 
\begin{enumerate}
\item It can be shown that
\[
a\ip{u,v}=g(f,\alpha,v)\mbox{ for all }v\in\{w\in C^2[0,1]:w(1)=0\}
\]
where $g(f,\alpha,v)$ is a function of $f$, $\alpha$ and $v$. Obtain a formula for $g(f,\alpha,v)$.

\vspace*{1em}
\item For the remainder of this question we shall just consider the case when
\[
f(x)=12x^2-24x+4.
\]
Note that in this case you obtained a formula for $u$ in a previous homework. For this part we will just consider the case when $\alpha=\beta=0$. In this case, we can obtain finite element approximations $u_N$ to $u$ by finding $u_N\in\rm{span}\{\phi_0,\ldots,\phi_N\}$ such that
\[
a\ip{u_N,v}=g(f,0,v)\mbox{ for all }v\in\rm{span}\{\phi_0,\ldots,\phi_N\}.
\]
Write a code which can obtain $u_N$ and $u_N^*$ where $u_N^*\in\rm{span}\{\phi_1,\ldots,\phi_N\}$ is such that
\[
a\ip{u_N^*,v}=g(f,0,v)\mbox{ for all }v\in\rm{span}\{\phi_1,\ldots,\phi_N\}.
\]
On the same figure, plot $u$ as well as $u_N$ and $u^*_N$ for $N=3$ and $N=7$.

\vspace*{1em}
\item For the case when $\alpha=\beta=0$, plot
\[
\left|\left|\left|u-u_N\right|\right|\right|
\]
and
\[
\left|\left|\left|u-u^*_N\right|\right|\right|
\]
for $N=1,3,7,15,31,63,127,255,511,1023,2047,4095,8191,16383,32767$.

\vspace*{1em}
\item Modify your code so that it can obtain finite element approximations $u_N$ to $u$ by finding $u_N\in\rm{span}\{\phi_0,\ldots,\phi_{N+1}\}$ such that $u_N(1)=\beta$ and
\[
a\ip{u_N,v}=g(f,\alpha,v)\mbox{ for all }v\in\rm{span}\{\phi_0,\ldots,\phi_N\}
\]
for any $\alpha\in\R$ and $\beta\in\R$. For the particular case of $\alpha = -1$ and $\beta = 1$, use your code to obtain $u_N$ for $N=3,7,15,31$ and on the same figure plot $u$ and $u_N$ for $N=3,7,15,31$.

\end{enumerate}

%%%%%%%%%%%%%%%%%%%%%%%%%%%%%%%%%%%%%%%%%%%%%%%%%%%%%%%%%%%%%%%%%%%%%%%%%%%%%%%%
\ifthenelse{\boolean{showsols}}{\input mixedFEM_sol}{}

