\begin{solution}
\begin{enumerate}
\item {[5 points]} The function
\[
u_N=\alpha\phi_0+\sum_{j=1}^{N+1}c_j\phi_j
\]
is such that $u_N(0)=\alpha$ and will be such that
\[
B\ip{u_N,v}=\ip{f,v}+\rho v(1)\mbox{ for all }v\in V_{N,D}
\]
when, for $j=1,2,\ldots,N+1$, the $c_j$ are such that
\begin{eqnarray*}
&&B\left(\alpha\phi_0+\sum_{j=1}^{N+1}c_j\phi_j,\phi_k\right)=\ip{f,\phi_k}+\rho \phi_k(1)\mbox{ for }k=1,2,\ldots,N+1,
\end{eqnarray*}
or equivalently,
\[
\sum_{j=1}^{N+1}c_jB\ip{\phi_j,\phi_k}=\ip{f,\phi_k}+\rho \phi_k(1)-\alpha B\ip{\phi_0,\phi_k}\mbox{ for }k=1,2,\ldots,N+1.
\]
We can write this system of equations in the form
\[
\BK\Bc=\Bb
\]
where $\BK\in\R^{\left(N+1\right)\times\left(N+1\right)}$ is the matrix with entries
\[
K_{jk}=B\ip{\phi_k,\phi_j}
\]
for $j,k=1,2,\ldots,N+1$ and $\Bb\in\R^{N+1}$ is the vector with entries
\[
b_j=\ip{f,\phi_j}+\rho \phi_j(1)-\alpha B\ip{\phi_0,\phi_j}=\left\{\begin{array}{ll}
\ip{f,\phi_1}-\alpha B\ip{\phi_0,\phi_1} & \mbox{if }j=1,
\\
\ip{f,\phi_{N+1}}+\rho & \mbox{if }j=N+1,
\\
\ip{f,\phi_j} & \mbox{otherwise.}
\end{array}\right.
\]
for $j=1,2,\ldots,N+1$.
\\
\item {[6 points]} Since $V_{N,D}$ is a subspace of $H^1_D\left(0,1\right)$, the fact that
\[
B\ip{u,v}=\ip{f,v}+\rho v(1)\mbox{ for all }v\in H^1_D\left(0,1\right)
\]
means that
\[
B\ip{u,v}=\ip{f,v}+\rho v(1)\mbox{ for all }v\in V_{N,D}.
\]
Moreover,
\[
B\ip{u_N,v}=\ip{f,v}+\rho v(1)\mbox{ for all }v\in V_{N,D}.
\]
Therefore the properties satisfied by a symmetric bilinear form allow us to say that, for all $v\in V_{N,D}$,
\begin{eqnarray*}
B\ip{u-u_N,v}&=&B\ip{u,v}-B\ip{u_N,v}
\\
&=&\ip{f,v}+\rho v(1)-\left(\ip{f,v}+\rho v(1)\right)
\\
&=&0.
\end{eqnarray*}
Consequently,
\[
B\ip{u-u_N,v}=0\mbox{ for all }v\in V_{N,D}.
\]
The properties satisfied by a symmetric bilinear form allow us to say that
\begin{eqnarray*}
B\ip{u-u_N,u-u_N}&=&B\ip{u,u-u_N}-B\ip{u_N,u-u_N}
\\
&=&B\ip{u,u}-B\ip{u,u_N}-B\ip{u_N,u}+B\ip{u_N,u_N}
\\
&=&B\ip{u,u}-2B\ip{u,u_N}+B\ip{u_N,u_N}.
\end{eqnarray*}
Now, $u_N-\alpha\phi_0\in V_{N,D}$ and so the fact that
\[
B\ip{u-u_N,v}=0\mbox{ for all }v\in V_{N,D}
\]
means that
\[
B\ip{u-u_N,u_N-\alpha\phi_0}=0
\]
and hence
\[
B\ip{u,u_N}=B\ip{u_N,u_N}+\alpha B\ip{u-u_N,\phi_0}
\]
since the properties satisfied by a symmetric bilinear form mean that
\begin{eqnarray*}
B\ip{u-u_N,u_N-\alpha\phi_0}&=&B\ip{u-u_N,u_N}-\alpha B\ip{u-u_N,\phi_0}
\\
&=&B\ip{u,u_N}-B\ip{u_N,u_N}-\alpha B\ip{u-u_N,\phi_0}.
\end{eqnarray*}
Therefore,
\begin{eqnarray*}
B\ip{u-u_N,u-u_N}&=&B\ip{u,u}-2\left(B\ip{u_N,u_N}+\alpha B\ip{u-u_N,\phi_0}\right)+B\ip{u_N,u_N}
\\
&=&B\ip{u,u}-2B\ip{u_N,u_N}-2\alpha B\ip{u-u_N,\phi_0}+B\ip{u_N,u_N}
\\
&=&B\ip{u,u}-B\ip{u_N,u_N}-2\alpha B\ip{u-u_N,\phi_0}.
\end{eqnarray*}
\\
\item {[9 points]} When $N=1$, $f(x)=2$, $\alpha=0$ and $\rho=0$,
\[
\BK=\left[\begin{array}{cc} B\ip{\phi_1,\phi_1} & B\ip{\phi_2,\phi_1} \\ B\ip{\phi_1,\phi_2} & B\ip{\phi_2,\phi_2}\end{array}\right]
\]
and
\[
\Bb=\left[\begin{array}{c}\ip{f,\phi_1} \\ \ip{f,\phi_2}\end{array}\right]
\]
where
\[
\phi_1(x) = \left\{\begin{array}{ll}
2x & \mbox{if }x\in\left[0,{1 \over 2}\right); \\ 2-2x & \mbox{if }x\in\left[{1 \over 2},1\right];\end{array}\right.
\]
and
\[
\phi_2(x) = \left\{\begin{array}{ll}
0 & \mbox{if }x\in\left[0,{1 \over 2}\right); \\ 2x-1 & \mbox{if }x\in\left[{1 \over 2},1\right];\end{array}\right.
\]
and hence
\[
\phi_1'(x) = \left\{\begin{array}{ll}
2 & \mbox{if }x\in\left(0,{1 \over 2}\right); \\ -2 & \mbox{if }x\in\left({1 \over 2},1\right);\end{array}\right.
\]
and
\[
\phi_2'(x) = \left\{\begin{array}{ll}
0 & \mbox{if }x\in\left(0,{1 \over 2}\right); \\ 2 & \mbox{if }x\in\left({1 \over 2},1\right).\end{array}\right.
\]
Now,
\[
\ip{\phi_1,\phi_1}=\int_0^1\phi_1(x)\phi_1(x)\,dx=\int_0^{1/2}\phi_1(x)\phi_1(x)\,dx+\int_{1/2}^1\phi_1(x)\phi_1(x)\,dx={1 \over 6}+{1 \over 6}={2 \over 6}={1 \over 3};
\]
\[
\ip{\phi_1,\phi_2}=\int_0^1\phi_1(x)\phi_2(x)\,dx=\int_{1/2}^1\phi_1(x)\phi_2(x)\,dx={1 \over 12};
\]
\[
\ip{\phi_2,\phi_1}=\ip{\phi_1,\phi_2}={1 \over 12};
\]
and
\[
\ip{\phi_2,\phi_2}=\int_0^1\phi_2(x)\phi_2(x)\,dx=\int_{1/2}^1\phi_2(x)\phi_2(x)\,dx={1 \over 6}.
\]
Moreover,
\begin{eqnarray*}
a\ip{\phi_1,\phi_1}&=&\int_0^1\phi_1'(x)\phi_1'(x)\,dx
\\
&=&\int_0^{1/2}\phi_1'(x)\phi_1'(x)\,dx+\int_{1/2}^1\phi_1'(x)\phi_1'(x)\,dx
\\
&=&\int_0^{1/2}4\,dx+\int_{1/2}^14\,dx
\\
&=&{4 \over 2}+{4 \over 2}
\\
&=&4;
\end{eqnarray*}
\[
a\ip{\phi_1,\phi_2}=\int_0^1\phi_1'(x)\phi_2'(x)\,dx=\int_{1/2}^1\phi_1'(x)\phi_2'(x)\,dx=\int_{1/2}^1-4\,dx=-{4 \over 2}=-2;
\]
\[
a\ip{\phi_2,\phi_1}=a\ip{\phi_1,\phi_2}=-2;
\]
and
\[
a\ip{\phi_2,\phi_2}=\int_0^1\phi_2'(x)\phi_2'(x)\,dx=\int_{1/2}^1\phi_2'(x)\phi_2'(x)\,dx=\int_{1/2}^14\,dx={4 \over 2}=2.
\]
Consequently,
\[
B\ip{\phi_1,\phi_1}=a\ip{\phi_1,\phi_1}+\ip{\phi_1,\phi_1}=4+{1 \over 3}={12 \over 3}+{1 \over 3}={13 \over 3};
\]
\[
B\ip{\phi_1,\phi_2}=a\ip{\phi_1,\phi_2}+\ip{\phi_1,\phi_2}=-2+{1 \over 12}=-{24 \over 12}+{1 \over 12}=-{23 \over 12};
\]
\[
B\ip{\phi_2,\phi_1}=B\ip{\phi_1,\phi_2}=-{23 \over 12};
\]
and
\[
B\ip{\phi_2,\phi_2}=a\ip{\phi_2,\phi_2}+\ip{\phi_2,\phi_2}=2+{1 \over 6}={12 \over 6}+{1 \over 6}={13 \over 6}.
\]
Furthermore,
\[
\ip{f,\phi_1}=2\int_0^1\phi_1(x)\,dx=2\left(\int_0^{1/2}\phi_1(x)\,dx+\int_{1/2}^1\phi_1(x)\,dx\right)=2\left({1 \over 4}+{1 \over 4}\right)={4 \over 4}=1;
\]
and
\[
\ip{f,\phi_2}=2\int_0^1\phi_2(x)\,dx=2\left(\int_0^{1/2}\phi_2(x)\,dx+\int_{1/2}^1\phi_2(x)\,dx\right)=2\left(0+{1 \over 4}\right)={2 \over 4}={1 \over 2}.
\]
Hence,
\begin{eqnarray*}
\BK&=&\left[\begin{array}{cc} B\ip{\phi_1,\phi_1} & B\ip{\phi_2,\phi_1} \\ B\ip{\phi_1,\phi_2} & B\ip{\phi_2,\phi_2}\end{array}\right]
\\
&=&\left[\begin{array}{cc} {13 \over 3} & -{23 \over 12} \\ -{23 \over 12} & {13 \over 6}\end{array}\right]
\end{eqnarray*}
and
\begin{eqnarray*}
\Bb&=&\left[\begin{array}{c}\ip{f,\phi_1} \\ \ip{f,\phi_2}\end{array}\right]
\\
&=&\left[\begin{array}{c}1 \\ {1 \over 2}\end{array}\right].
\end{eqnarray*}
\\
\item {[5 points]} When $N=1$, $f(x)=2$, $\alpha=-1$ and $\rho=1$,
\[
\BK=\left[\begin{array}{cc} B\ip{\phi_1,\phi_1} & B\ip{\phi_2,\phi_1} \\ B\ip{\phi_1,\phi_2} & B\ip{\phi_2,\phi_2}\end{array}\right]
\]
and
\[
\Bb=\left[\begin{array}{c}\ip{f,\phi_1}+B\ip{\phi_0,\phi_1} \\ \ip{f,\phi_2}+1\end{array}\right]
\]
where
\[
\phi_0(x) = \left\{\begin{array}{ll}
1-2x & \mbox{if }x\in\left[0,{1 \over 2}\right); \\ 0 & \mbox{if }x\in\left[{1 \over 2},1\right];\end{array}\right.
\]
\[
\phi_1(x) = \left\{\begin{array}{ll}
2x & \mbox{if }x\in\left[0,{1 \over 2}\right); \\ 2-2x & \mbox{if }x\in\left[{1 \over 2},1\right];\end{array}\right.
\]
and
\[
\phi_2(x) = \left\{\begin{array}{ll}
0 & \mbox{if }x\in\left[0,{1 \over 2}\right); \\ 2x-1 & \mbox{if }x\in\left[{1 \over 2},1\right];\end{array}\right.
\]
and hence
\[
\phi_0'(x) = \left\{\begin{array}{ll}
-2 & \mbox{if }x\in\left(0,{1 \over 2}\right); \\ 0 & \mbox{if }x\in\left({1 \over 2},1\right);\end{array}\right.
\]
\[
\phi_1'(x) = \left\{\begin{array}{ll}
2 & \mbox{if }x\in\left(0,{1 \over 2}\right); \\ -2 & \mbox{if }x\in\left({1 \over 2},1\right);\end{array}\right.
\]
and
\[
\phi_2'(x) = \left\{\begin{array}{ll}
0 & \mbox{if }x\in\left(0,{1 \over 2}\right); \\ 2 & \mbox{if }x\in\left({1 \over 2},1\right).\end{array}\right.
\]
Now,
\[
\ip{\phi_0,\phi_1}=\int_0^1\phi_0(x)\phi_1(x)\,dx=\int_0^{1/2}\phi_0(x)\phi_1(x)\,dx={1 \over 12};
\]
and
\[
a\ip{\phi_0,\phi_1}=\int_0^1\phi_0'(x)\phi_1'(x)\,dx=\int_0^{1/2}-4\,dx=-{4 \over 2}=-2.
\]
Consequently,
\[
B\ip{\phi_0,\phi_1}=a\ip{\phi_0,\phi_1}+\ip{\phi_0,\phi_1}=-2+{1 \over 12}=-{24 \over 12}+{1 \over 12}=-{23 \over 12}.
\]
Moreover, in part (c) we computed that
\[
B\ip{\phi_1,\phi_1}={13 \over 3};
\]
\[
B\ip{\phi_1,\phi_2}=B\ip{\phi_2,\phi_1}=-{23 \over 12};
\]
\[
B\ip{\phi_2,\phi_2}={13 \over 6};
\]
\[
\ip{f,\phi_1}=1;
\]
and
\[
\ip{f,\phi_2}={1 \over 2}.
\]
Hence,
\begin{eqnarray*}
\BK&=&\left[\begin{array}{cc} B\ip{\phi_1,\phi_1} & B\ip{\phi_2,\phi_1} \\ B\ip{\phi_1,\phi_2} & B\ip{\phi_2,\phi_2}\end{array}\right]
\\
&=&\left[\begin{array}{cc} {13 \over 3} & -{23 \over 12} \\ -{23 \over 12} & {13 \over 6}\end{array}\right]
\end{eqnarray*}
and
\begin{eqnarray*}
\Bb&=&\left[\begin{array}{c}\ip{f,\phi_1}+B\ip{\phi_0,\phi_1} \\ \ip{f,\phi_2}+1\end{array}\right]
\\
&=&\left[\begin{array}{c}1-{23 \over 12} \\ {1 \over 2}+1\end{array}\right]
\\
&=&\left[\begin{array}{c}{12 \over 12}-{23 \over 12} \\ {1 \over 2}+{2 \over 2}\end{array}\right]
\\
&=&\left[\begin{array}{c}-{11 \over 12} \\ {3 \over 2}\end{array}\right].
\end{eqnarray*}
\end{enumerate}
\end{solution}