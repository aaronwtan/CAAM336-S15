\begin{solution}
\begin{enumerate}
\item {[5 points]} If $v\in C^2[0,1]$ and $w\in C^2[0,1]$ then
\[
L(v+w)=-(v+w)''+9(v+w)=-v''-w''+9v+9w=-v''+9v-w''+9w=Lv+Lw
\]
and so $L(v+w)=Lv+Lw$ for all $v,w\in C^2[0,1]$. If $v\in C^2[0,1]$ and $\gamma\in\R$ then
\[
L(\gamma v)=-(\gamma v)''+9(\gamma v)=-\gamma v''+9\gamma v=\gamma(-v''+9v)=\gamma Lv
\]
and so $L(\gamma v)=\gamma Lv$ for all $v\in C^2[0,1]$ and all $\gamma\in\R$. Consequently, $L$ is a linear operator.
\\
\item {[5 points]} For $j=1,2,\ldots,N$, using the approximation
\[
u''(x_j)\approx\frac{u(x_{j-1})-2u(x_j)+u(x_{j+1})}{h^2}
\]
yields that
\begin{eqnarray*}
(Lu)(x_j)&=&-u''(x_j)+9u(x_j)\approx-\frac{u(x_{j-1})-2u(x_j)+u(x_{j+1})}{h^2}+9u(x_j)
\\
&=&-\frac{1}{h^2}u(x_{j-1})+\left(\frac{2}{h^2}+9\right)u(x_j)-\frac{1}{h^2}u(x_{j+1}).
\end{eqnarray*}
So,
\begin{eqnarray*}
&&\left[\begin{array}{c} (Lu)(x_1) \\ (Lu)(x_2) \\ \vdots \\ (Lu)(x_{N-1}) \\ (Lu)(x_N) \end{array}\right]
\\
&\approx&\left[\begin{array}{c} -\frac{1}{h^2}u(x_0)+\left(\frac{2}{h^2}+9\right)u(x_1)-\frac{1}{h^2}u(x_2) \\ -\frac{1}{h^2}u(x_1)+\left(\frac{2}{h^2}+9\right)u(x_2)-\frac{1}{h^2}u(x_3) \\ \vdots \\ -\frac{1}{h^2}u(x_{N-2})+\left(\frac{2}{h^2}+9\right)u(x_{N-1})-\frac{1}{h^2}u(x_N) \\ -\frac{1}{h^2}u(x_{N-1})+\left(\frac{2}{h^2}+9\right)u(x_N)-\frac{1}{h^2}u(x_{N+1}) \end{array}\right]
\\
&=&
\left[\begin{array}{ccccccccc} 
-\frac{1}{h^2} & \frac{2}{h^2}+9 & -\frac{1}{h^2} & 0 & 0 & \cdots & 0 & 0 & 0
\\
0 & -\frac{1}{h^2} & \frac{2}{h^2}+9 & -\frac{1}{h^2} & 0 & \cdots & 0 & 0 & 0
\\
0 & 0 & -\frac{1}{h^2} & \frac{2}{h^2}+9 & -\frac{1}{h^2} & \cdots & 0 & 0 & 0
\\
\vdots & \vdots & \vdots & \ddots & \ddots & \ddots & \vdots & \vdots & \vdots
\\
0 & 0 & 0 & \cdots & -\frac{1}{h^2} & \frac{2}{h^2}+9 & -\frac{1}{h^2} & 0 & 0
\\
0 & 0  & 0 & \cdots & 0 & -\frac{1}{h^2} & \frac{2}{h^2}+9 & -\frac{1}{h^2} & 0
\\
0 & 0 & 0 & \cdots & 0 & 0 & -\frac{1}{h^2} & \frac{2}{h^2}+9 & -\frac{1}{h^2}
 \end{array}\right]
\left[\begin{array}{c} u(x_0) \\ u(x_1) \\u(x_2) \\ \vdots \\ u(x_{N-1}) \\ u(x_N) \\ u(x_{N+1}) \end{array}\right]
\\
&=&\BD\left[\begin{array}{c} u(x_0) \\ u(x_1) \\u(x_2) \\ \vdots \\ u(x_{N-1}) \\ u(x_N) \\ u(x_{N+1}) \end{array}\right]
\end{eqnarray*}
where $\BD\in\R^{N\times\left(N+2\right)}$ is the matrix with entries
\[
D_{jk}=\left\{\begin{array}{rl}
\frac{2}{h^2}+9 & \mbox{if }k=j+1;
\\
-\frac{1}{h^2} & \mbox{if }k=j\mbox{ or }k=j+2;
\\
0 & \mbox{otherwise}.
\end{array}\right.
\]
\\
\item {[10 points]} Since
\[
(Lu)(x)=f(x),\quad0<x<1
\]
we have that
\[
\left[\begin{array}{c} (Lu)(x_1) \\ (Lu)(x_2) \\ \vdots \\ (Lu)(x_{N-1}) \\ (Lu)(x_N) \end{array}\right]=\left[\begin{array}{c} f(x_1) \\ f(x_2) \\ \vdots \\ f(x_{N-1}) \\ f(x_N) \end{array}\right]
\]
for $j=1,\ldots,N$. Using the approximation obtained in the previous part then yields that
\[
\BD\left[\begin{array}{c} u(x_0) \\ u(x_1) \\u(x_2) \\ \vdots \\ u(x_{N-1}) \\ u(x_N) \\ u(x_{N+1}) \end{array}\right]\approx\left[\begin{array}{c} f(x_1) \\ f(x_2) \\ \vdots \\ f(x_{N-1}) \\ f(x_N) \end{array}\right]
\]
Moreover,
\begin{eqnarray*}
\BD\left[\begin{array}{c} u(x_0) \\ u(x_1) \\u(x_2) \\ \vdots \\ u(x_{N-1}) \\ u(x_N) \\ u(x_{N+1}) \end{array}\right]&=&\BD\left(\left[\begin{array}{c} 0 \\ u(x_1) \\u(x_2) \\ \vdots \\ u(x_{N-1}) \\ u(x_N) \\ 0 \end{array}\right]+\left[\begin{array}{c} u(x_0) \\ 0 \\ 0 \\ \vdots \\ 0 \\ 0 \\ u(x_{N+1}) \end{array}\right]\right)
\\
&=&\BD\left(\left[\begin{array}{c} 0 \\ u(x_1) \\u(x_2) \\ \vdots \\ u(x_{N-1}) \\ u(x_N) \\ 0 \end{array}\right]+\left[\begin{array}{c} \alpha \\ 0 \\ 0 \\ \vdots \\ 0 \\ 0 \\ \beta \end{array}\right]\right)
\\
&=&\BD\left[\begin{array}{c} 0 \\ u(x_1) \\u(x_2) \\ \vdots \\ u(x_{N-1}) \\ u(x_N) \\ 0 \end{array}\right]+\BD\left[\begin{array}{c} \alpha \\ 0 \\ 0 \\ \vdots \\ 0 \\ 0 \\ \beta \end{array}\right]
\end{eqnarray*}
since $u(x_0)=u(0)=\alpha$ and $u(x_{N+1})=u(1)=\beta$. Furthermore,
\begin{eqnarray*}
&&\BD\left[\begin{array}{c} 0 \\ u(x_1) \\u(x_2) \\ \vdots \\ u(x_{N-1}) \\ u(x_N) \\ 0 \end{array}\right]
\\
&=&\left[\begin{array}{ccccccccc} 
-\frac{1}{h^2} & \frac{2}{h^2}+9 & -\frac{1}{h^2} & 0 & 0 & \cdots & 0 & 0 & 0
\\
0 & -\frac{1}{h^2} & \frac{2}{h^2}+9 & -\frac{1}{h^2} & 0 & \cdots & 0 & 0 & 0
\\
0 & 0 & -\frac{1}{h^2} & \frac{2}{h^2}+9 & -\frac{1}{h^2} & \cdots & 0 & 0 & 0
\\
\vdots & \vdots & \vdots & \ddots & \ddots & \ddots & \vdots & \vdots & \vdots
\\
0 & 0 & 0 & \cdots & -\frac{1}{h^2} & \frac{2}{h^2}+9 & -\frac{1}{h^2} & 0 & 0
\\
0 & 0  & 0 & \cdots & 0 & -\frac{1}{h^2} & \frac{2}{h^2}+9 & -\frac{1}{h^2} & 0
\\
0 & 0 & 0 & \cdots & 0 & 0 & -\frac{1}{h^2} & \frac{2}{h^2}+9 & -\frac{1}{h^2}
 \end{array}\right]
\left[\begin{array}{c} 0 \\ u(x_1) \\u(x_2) \\ \vdots \\ u(x_{N-1}) \\ u(x_N) \\ 0 \end{array}\right]
\\
&=&\left[\begin{array}{ccccccccc} 
\frac{2}{h^2}+9 & -\frac{1}{h^2} & 0 & 0 & \cdots & 0 & 0
\\
-\frac{1}{h^2} & \frac{2}{h^2}+9 & -\frac{1}{h^2} & 0 & \cdots & 0 & 0
\\
 0 & -\frac{1}{h^2} & \frac{2}{h^2}+9 & -\frac{1}{h^2} & \cdots & 0 & 0
\\
\vdots & \vdots & \ddots & \ddots & \ddots & \vdots & \vdots
\\
0 & 0 & \cdots & -\frac{1}{h^2} & \frac{2}{h^2}+9 & -\frac{1}{h^2} & 0
\\
0  & 0 & \cdots & 0 & -\frac{1}{h^2} & \frac{2}{h^2}+9 & -\frac{1}{h^2}
\\
0 & 0 & \cdots & 0 & 0 & -\frac{1}{h^2} & \frac{2}{h^2}+9
 \end{array}\right]
\left[\begin{array}{c} u(x_1) \\u(x_2) \\ \vdots \\ u(x_{N-1}) \\ u(x_N) \end{array}\right]
\end{eqnarray*}
and so
\[
\BA\left[\begin{array}{c} u(x_1) \\u(x_2) \\ \vdots \\ u(x_{N-1}) \\ u(x_N) \end{array}\right]\approx\Bb
\]
where $\BA\in\R^{N\times N}$ is the matrix with entries
\[
A_{jk}=\left\{\begin{array}{rl}
\frac{2}{h^2}+9 & \mbox{if }k=j;
\\
-\frac{1}{h^2} & \mbox{if }k=j-1\mbox{ or }k=j+1;
\\
0 & \mbox{otherwise}.
\end{array}\right.
\]
and
\[
\Bb=\left[\begin{array}{c} f(x_1) \\ f(x_2) \\ \vdots \\ f(x_{N-1}) \\ f(x_N) \end{array}\right]-\BD\left[\begin{array}{c} \alpha \\ 0 \\ 0 \\ \vdots \\ 0 \\ 0 \\ \beta \end{array}\right].
\]
Now,
\begin{eqnarray*}
&&\BD\left[\begin{array}{c} \alpha \\ 0 \\ 0 \\ \vdots \\ 0 \\ 0 \\ \beta \end{array}\right]
\\
&=&\left[\begin{array}{ccccccccc} 
-\frac{1}{h^2} & \frac{2}{h^2}+9 & -\frac{1}{h^2} & 0 & 0 & \cdots & 0 & 0 & 0
\\
0 & -\frac{1}{h^2} & \frac{2}{h^2}+9 & -\frac{1}{h^2} & 0 & \cdots & 0 & 0 & 0
\\
0 & 0 & -\frac{1}{h^2} & \frac{2}{h^2}+9 & -\frac{1}{h^2} & \cdots & 0 & 0 & 0
\\
\vdots & \vdots & \vdots & \ddots & \ddots & \ddots & \vdots & \vdots & \vdots
\\
0 & 0 & 0 & \cdots & -\frac{1}{h^2} & \frac{2}{h^2}+9 & -\frac{1}{h^2} & 0 & 0
\\
0 & 0  & 0 & \cdots & 0 & -\frac{1}{h^2} & \frac{2}{h^2}+9 & -\frac{1}{h^2} & 0
\\
0 & 0 & 0 & \cdots & 0 & 0 & -\frac{1}{h^2} & \frac{2}{h^2}+9 & -\frac{1}{h^2}
 \end{array}\right]\left[\begin{array}{c} \alpha \\ 0 \\ 0 \\ \vdots \\ 0 \\ 0 \\ \beta \end{array}\right]
\\
&=&\left[\begin{array}{c} -\frac{\alpha}{h^2} \\ 0 \\ 0 \\ \vdots \\ 0 \\ 0 \\ -\frac{\beta}{h^2} \end{array}\right]
\end{eqnarray*}
and so
\[
\Bb=\left[\begin{array}{c} f(x_1) \\ f(x_2) \\ \vdots \\ f(x_{N-1}) \\ f(x_N) \end{array}\right]-\left[\begin{array}{c} -\frac{\alpha}{h^2} \\ 0 \\ 0 \\ \vdots \\ 0 \\ 0 \\ -\frac{\beta}{h^2} \end{array}\right]=\left[\begin{array}{c} f(x_1)+\frac{\alpha}{h^2} \\ f(x_2) \\ \vdots \\ f(x_{N-1}) \\ f(x_N)+\frac{\beta}{h^2} \end{array}\right].
\]
Hence $\Bb\in\R^N$ is the vector with entries
\[
b_j=\left\{\begin{array}{rl}
f(x_1)+\frac{\alpha}{h^2} & \mbox{if }j=1;
\\
f(x_N)+\frac{\beta}{h^2} & \mbox{if }j=N;
\\
f(x_j) & \mbox{otherwise}.
\end{array}\right.
\]
\\
\item {[5 points]} When $N=2$, $h=\frac{1}{2+1}=\frac{1}{3}$ and so $h^2=\frac{1}{3^2}=\frac{1}{9}$ and hence $\frac{1}{h^2}=9$ and $\frac{2}{h^2}=18$. Therefore,
\[
\BA=\left[\begin{array}{cc}  18+9 & -9 \\ -9 & 18+9 \end{array}\right]=\left[\begin{array}{cc}  27 & -9 \\ -9 & 27 \end{array}\right].
\]
Moreover, when $N=2$, $f(x)=18$ and $\alpha=\beta=0$,
\[
\Bb=\left[\begin{array}{c} 18 \\ 18 \end{array}\right].
\]
Consequently, we have that
\[
\left[\begin{array}{cc}  27 & -9 \\ -9 & 27 \end{array}\right]\left[\begin{array}{c} u_1 \\ u_2 \end{array}\right]=\left[\begin{array}{c} 18 \\ 18 \end{array}\right]
\]
and hence
\[
\left[\begin{array}{cc}  3 & -1 \\ -1 & 3 \end{array}\right]\left[\begin{array}{c} u_1 \\ u_2 \end{array}\right]=\left[\begin{array}{c} 2 \\ 2 \end{array}\right].
\]
Therefore,
\begin{eqnarray*}
\left[\begin{array}{c} u_1 \\ u_2 \end{array}\right]&=&\left[\begin{array}{cc}  3 & -1 \\ -1 & 3 \end{array}\right]^{-1}\left[\begin{array}{c} 2 \\ 2 \end{array}\right]
\\
&=&\frac{1}{9-1}\left[\begin{array}{cc}  3 & 1 \\ 1 & 3 \end{array}\right]\left[\begin{array}{c} 2 \\ 2 \end{array}\right]
\\
&=&\frac{1}{8}\left[\begin{array}{cc}  3 & 1 \\ 1 & 3 \end{array}\right]\left[\begin{array}{c} 2 \\ 2 \end{array}\right]
\\
&=&\frac{1}{8}\left[\begin{array}{c} 6+2 \\ 2+6 \end{array}\right]
\\
&=&\frac{1}{8}\left[\begin{array}{c} 8 \\ 8 \end{array}\right]
\\
&=&\left[\begin{array}{c} 1 \\ 1 \end{array}\right].
\end{eqnarray*}
\end{enumerate} 
\end{solution}