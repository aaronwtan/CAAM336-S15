
Let the operator $L:\mbox{ }C^2[0,1]\rightarrow C[0,1]$ be defined by 
\[
Lv=-v''+9v.
\]
Let $u\in C^2[0,1]$ be the solution to the differential equation
\[
-u''(x)+9u(x)=f(x),\quad0<x<1
\]
with boundary conditions
\[
u(0)=\alpha
\]
and
\[
u(1)=\beta
\]
where $f\in C[0,1]$ and $\alpha,\beta\in\R$. Note that
\[
(Lu)(x)=-u''(x)+9u(x)
\]
for all $x\in[0,1]$. Let $N$ be an integer which is such that $N\ge2$ and let $\displaystyle{h = {1\over N+1}}$ and $x_j = jh$ for $j=0,\ldots, N+1$.

\begin{enumerate}
\item Determine whether or not $L$ is a linear operator.
\\
\item By using the approximation
\[
u''(x_j)\approx\frac{u(x_{j-1})-2u(x_j)+u(x_{j+1})}{h^2}
\]
for $j=1,\ldots,N$ we can write
\[
\left[\begin{array}{c} (Lu)(x_1) \\ (Lu)(x_2) \\ \vdots \\ (Lu)(x_{N-1}) \\ (Lu)(x_N) \end{array}\right]\approx\BD\left[\begin{array}{c} u(x_0) \\ u(x_1) \\u(x_2) \\ \vdots \\ u(x_{N-1}) \\ u(x_N) \\ u(x_{N+1}) \end{array}\right]
\]
where $\BD\in\R^{N\times\left(N+2\right)}$. What are the entries of the matrix $\BD$? An acceptable way to present your final answer is
\[
D_{jk}=\left\{\begin{array}{rl}
? & \mbox{if }k=?;
\\
? & \mbox{if }k=?\mbox{ or }k=?;
\\
? & \mbox{otherwise};
\end{array}\right.
\]
with the question marks replaced with the correct values.
\\
\item We can use the differential equation and boundary conditions satisfied by $u$ and the approximation from the previous part to write
\[
\BA\left[\begin{array}{c} u(x_1) \\u(x_2) \\ \vdots \\ u(x_{N-1}) \\ u(x_N) \end{array}\right]\approx\Bb
\]
where $\BA\in\R^{N\times N}$ and $\Bb\in\R^N$. What are the entries of the matrix $\BA$ and the vector $\Bb$? An acceptable way to present your final answer is
\[
A_{jk}=\left\{\begin{array}{rl}
? & \mbox{if }k=?;
\\
? & \mbox{if }k=?\mbox{ or }k=?;
\\
? & \mbox{otherwise};
\end{array}\right.
\]
and
\[
b_j=\left\{\begin{array}{rl}
? & \mbox{if }j=?;
\\
? & \mbox{if }j=?;
\\
? & \mbox{otherwise};
\end{array}\right.
\]
with the question marks replaced with the correct values.
\\
\item Let $f(x)=18$, $\alpha=\beta=0$ and $N=2$. Obtain approximations $u_1$ and $u_2$ to $u(x_1)$ and $u(x_2)$, respectively, by solving
\[
\BA\left[\begin{array}{c} u_1 \\u_2 \end{array}\right]=\Bb
\]
by hand.
\end{enumerate} 


%%%%%%%%%%%%%%%%%%%%%%%%%%%%%%%%%%%%%%%%%%%%%%%%%%%%%%%%%%%%%%%%%%%%%%%%%%%%%%%%

\ifthenelse{\boolean{showsols}}{\input fdm_sol}{}
