
Let the inner product $\ip{\cdot,\cdot}:\mbox{ }C[0,1]\times C[0,1]\rightarrow\R$ be defined by
\[
\ip{v,w}=\int_0^1v(x)w(x)\,dx
\]
and let the norm $\norm{\cdot}:\mbox{ }C[0,1]\rightarrow\R$ be defined by
\[
\norm{v}=\sqrt{\ip{v,v}}.
\]
Let $N$ be a posivitive integer and let $\psi_1,\ldots,\psi_N\in C[0,1]$ be such that $\left\{\psi_1,\ldots,\psi_N\right\}$ is orthonormal with respect to the inner product $\ip{\cdot,\cdot}$. For $g\in C[0,1]$, let
\[
g_N=\sum_{n=1}^N\alpha_n\psi_n
\]
where $\alpha_n=\ip{g,\psi_n}$. Note that $g_N$ is the best approximation to $g$ from ${\rm span}\left\{\psi_1,\ldots,\psi_N\right\}$ with respect to the norm $\norm{\cdot}$. Moreover, let $u\in C^2[0,1]$ be such that
\[
-u''(x)=f(x),\quad 0<x<1
\]
and
\[
u(0)=u(1)=0
\]
with $f\in C[0,1]$ being defined by $f(x)=1$ for all $x\in[0,1]$. Note that $\displaystyle{u(x) = {1 \over 2}x(1-x)}$.
\\
\begin{enumerate}
\item Show that
\[
\norm{g-g_N}^2  = \norm{g}^2 - \sum_{n=1}^N \alpha_n^2.
\]
\\
\item For the remainder of this question we will just consider the case when
\[
\psi_n(x) = \sqrt{2} \sin(n \pi x)\mbox{ for }n=1,2,\ldots.
\]
The best approximation to $f$ from ${\rm span}\{\psi_1, \ldots, \psi_N\}$ with respect to the norm $\norm{\cdot}$ is
\[ 
f_N = \sum_{n=1}^N \ip{f,\psi_n} \psi_n.
\]
Produce a \verb|loglog| plot of $\norm{f-f_N}$ for $N=1,2,\ldots, 1000000$. Note that, for $n=1,2,\ldots$,
\[
\ip{f,\psi_n}={\sqrt{2} \over n\pi}\left(1-\left(-1\right)^n\right).
\]
\\
\item We can use the spectral method to conclude that the best approximation to $u$ from ${\rm span}\{\psi_1, \ldots, \psi_N\}$ with respect to the norm $\norm{\cdot}$ is
\[ 
u_N = \sum_{n=1}^N \ip{u,\psi_n} \psi_n
\]
where
\[
\ip{u,\psi_n}={\ip{f,\psi_n} \over n^2\pi^2}={\sqrt{2} \over n^3\pi^3}\left(1-\left(-1\right)^n\right).
\]
Add a \verb|loglog| plot of $\norm{u-u_N}$ for $N=1,2,\ldots, 1000000$ to the plot that you produced in part (b).

(Be aware that the norm of the error may appear to flatline or become imaginary around $10^{-8}$: this is a consequence of the computer's floating point arithmetic, and so you will not lose points because of this.)
\end{enumerate}


%%%%%%%%%%%%%%%%%%%%%%%%%%%%%%%%%%%%%%%%%%%%%%%%%%%%%%%%%%%%%%%%%%%%%%%%%%%%%%%%

\ifthenelse{\boolean{showsols}}{\input fourerr_sol}{}
