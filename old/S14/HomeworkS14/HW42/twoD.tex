
Let
\[
\Omega=\left\{(x,y):\,0\le x\le 1,\,0\le y\le 1\right\}
\]
and let $f\in  C(\Omega)$ be defined by $f(x,y) = x(1-y)$. In this question we will consider the problem of finding the solution $u(x,y)$ to the steady-state heat equation
\[ 
-(u_{xx}(x,y) +u_{yy}(x,y)) = f(x,y), \qquad 0\le x \le 1, \quad 0\le y\le 1,
\]
with homogeneous Dirichlet boundary conditions
\[
u(x,0)=u(x,1)=u(0,y)=u(1,y)=0, \qquad 0\le x \le 1, \quad 0\le y\le 1.
\]
Let
\[
C^2_D(\Omega)=\left\{v\in C^2(\Omega):\,v(x,0)=v(x,1)=v(0,y)=v(1,y)=0,\,0\le x\le 1,\,0\le y\le 1\right\}.
\]
Let the linear operator $L:\,C^2_D(\Omega)\rightarrow C(\Omega)$ be defined by
\[
\left(L v\right)(x,y) = -\left(v_{xx}(x,y) + v_{yy}(x,y)\right).
\]
Let the inner product $\ip{\cdot,\cdot}:\,C(\Omega)\times C(\Omega)\rightarrow \R$ be defined by
\[
\ip{v,w} =\int_0^1 \int_0^1 v(x,y) w(x,y) \, dx\,dy.
\]
\\
\begin{enumerate}
\item Show that $L$ is symmetric by showing that
\[
\ip{Lv,w}=\ip{v,Lw}\mbox{ for all }v,w\in C^2_D(\Omega).
\]
\\
\item The operator $L$ has eigenvalues $\lambda_{j,k}\in\R$ and eigenfunctions
\[
\psi_{j,k}(x,y) = 2 \sin(j \pi x) \sin(k \pi y)
\]
for $j,k = 1,2,\ldots$, which are such that
\[
L\psi_{j,k}=\lambda_{j,k}\psi_{j,k}
\]
for $j,k = 1,2,\ldots$. Obtain a formula for $\lambda_{j,k}$ for $j,k = 1,2,\ldots$.
\\
\item Let
\[
C^2_D[0,1]=\left\{v\in C^2[0,1]:\,v(0)=v(1)=0\right\}
\]
and let the linear operator $L_1:\,C^2_D[0,1]\rightarrow C[0,1]$ be defined by
\[
L_1w = -w''.
\]
Use what you know about the eigenfunctions of $L_1$ to compute $\ip{\psi_{j,k},\psi_{m,n}}$ for $j,k,m,n = 1,2,\ldots$.
\\
\item The solution to $Lu=f$ that we obtain using the spectral method is
\[
u(x,y) = \sum_{j=1}^\infty \sum_{k=1}^\infty{1\over \lambda_{j,k}} {(f,\psi_{j,k}) \over (\psi_{j,k},\psi_{j,k})}\, \psi_{j,k}(x,y).
\]
Plot
\[
u_N(x,y)= \sum_{j=1}^N \sum_{k=1}^N{1\over \lambda_{j,k}} {(f,\psi_{j,k}) \over (\psi_{j,k},\psi_{j,k})}\, \psi_{j,k}(x,y)
\]
for $N=1,2,3,4,5,10$. Note that, for $j,k = 1,2,\ldots$,
\[
\ip{f, \psi_{j,k}}=2{(-1)^{j+1} \over jk\pi^2}.
\]
Also note that to plot $\psi_{1,1}(x,y) = 2 \sin(\pi x)\sin(\pi y)$ you could use

\lstinputlisting{hint.m}

\end{enumerate}


%%%%%%%%%%%%%%%%%%%%%%%%%%%%%%%%%%%%%%%%%%%%%%%%%%%%%%%%%%%%%%%%%%%%%%%%%%%%%%%%

\ifthenelse{\boolean{showsols}}{\input twoD_sol}{}

