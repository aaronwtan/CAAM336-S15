Consider the linear differential equation,
\begin{eqnarray*}
-u'' + u &=&f \qquad 0<x<1\\
		u(0)&=&0 \\
		u'(1)&=&0
\end{eqnarray*}

Let $f(x) = x(1-x)$. Suppose that $N$ is a positive integer and define $\displaystyle{h = {1\over N+1}}$ and $x_i = ih$ for $i = 0,1\ldots, N+1$. Consider the hat functions $\phi_i\in C[0,1]$, defined as
\[
\phi_i(x) = \left\{
\begin{array}{ll}
\displaystyle{{x-x_{i-1}\over h}} & \mbox{if }x\in [x_{i-1}, x_i);\\
\displaystyle{{x_{i+1}-x\over h}} & \mbox{if }x\in [x_i, x_{i+1});\\
0 & \mbox{otherwise};
\end{array}\right.
\]
for $i=1,\ldots, N+1$. 
\\
Let the stiffness matrix $K$ be defined as
\[
K_{ij} = \int_0^1 \phi_j'(x) \phi_i'(x) dx  
\]
Likewise, let the mass matrix $M$ be defined as 
\[
M_{ij} = \int_0^1 \phi_j(x) \phi_i(x) dx  
\]
\begin{enumerate}
\item  Show that the finite element matrix $A$ for the weak form of the equation
\[
-u'' + u = f
\]
can be defined as $A = K +M$.  Specify the entries $M_{ij}$ and $K_{ij}$ (you may use the results of previous homework).  
\item Show that $A$ is positive definite. \emph{Hint: Use the weak form.}
\item Write \textsc{Matlab} code to solve the finite element system
\[
A\alpha = b
\]
for the approximate solution $u_N(x) = \sum_{j=1}^N \alpha_j \phi_j(x)$ of the differential equation using the finite element method.  Produce a plot that compares the approximate solution $u_N$ for $N = 4$ and $N = 8$ with the true solution
\[
u(x) = -x^2 + x - 2 + \frac{e(2e-1)}{1 + e^2} e^{-x} + \frac{e + 2}{1 + e^2} e^x
\]
\\
\emph{Hint: If you'd like, you can use the Matlab code called posted on the course webpage.
You may also use the Matlab function quad for numerical integration.}

\item Describe what modifications to the load vector $b$ are necessary to compute the solution to the problem with inhomogeneous Neumann boundary condition
\begin{eqnarray*}
-u'' + u &=&f \qquad 0<x<1\\
u(0)&=&0 \\
u'(1)&=&1.
\end{eqnarray*}
Modify your Matlab code to accommodate these changes, and produce a plot of the solution for $N = 4, 8, 16$.  \emph{You do not need to compare against the exact solution for this problem.}
\end{enumerate}

%%%%%%%%%%%%%%%%%%%%%%%%%%%%%%%%%%%%%%%%%%%%%%%%%%%%%%%%%%%%%%%%%%%%%%%%%%%%%%%%

\ifthenelse{\boolean{showsols}}{


\begin{solution}

\begin{enumerate}

\item  Let $V=\{v\in C^2[0,1] : v(0)=0\}$ For $\forall v \in V$ Multiply both side of BVP by the test function $v$
\begin{eqnarray*}
(-u'' + u)v &=&f v
\end{eqnarray*}
Now integrate both sides over $0$ to $1$

\begin{eqnarray*}
\int_0^1(-u'' + u)v dx &=&\int_0^1 f v dx
\end{eqnarray*}

Apply integration by parts to second derivative

\begin{eqnarray*}
\int_0^1(-u'') v dx+ \int_0^1 u v dx &=& \left[-u' v\right]_0^1 + \int_0^1 u'v' dx + \int_0^1 u v dx \\
&=&\left(-u'(1) v(1) + u'(0)v(0)\right) + \int_0^1 u'v' dx +  \int_0^1u v dx \\
&=& \int_0^1 u'v' dx + \int_0^1 u v dx \:\: \mbox{(by the mixed boundary condition and $v(0)=0$)} \\
&=& a(u,v)+ (u,v)
\end{eqnarray*}
Thus the variational form is
\[ a(u,v)+ (u,v)  = (f,v) \qquad \forall v \in V\]

Now define a finite element space $V_n = \{\phi_1, \phi_2,\cdots \phi_{n+1}\}$ and define the finite element solution $u_n = \sum_{j=1}^{n+1} c_j \phi_j$ then we get 

\[ a(\sum_{j=1}^{n+1} c_j \phi_j, \phi_i)+ (\sum_{j=1}^{n+1} c_j \phi_j, \phi_i)  = (f,\phi_i) \qquad  i=1, \cdots n+1\]

Above system can be written also as follows

\[ \sum_{j=1}^{n+1}\left( \underbrace{a( \phi_j, \phi_i)}_{K_{ij}} + \underbrace{(\phi_j, \phi_i)}_{M_{ij}}\right) c_j   = (f,\phi_i) \qquad  i=1, \cdots n+1.\] 

This is a linear system of equation $A c = f $ where $A = K+ M$ and  $A$ is $(N+1)\times (N+1) $ matrix and $f$ is $(N+1)\times 1 $ vector.

Now let us specify entries of $K$ and $M$. Here we should note that the hat function $\phi_i$ defined in the problem for $i=1,2 \cdots, n$. However $\phi_{n+1}$ is only half of the other hat functions. Then
\[
\phi_{n+1}(x) = \left\{
\begin{array}{ll}
\displaystyle{{x-x_{n}\over h}} & \mbox{if }x\in [1-h, 1);\\
0 & \mbox{otherwise};
\end{array}\right.
\]
By the definition of $\phi_i$ we have

\[
\phi_i'(x) = \left\{
\begin{array}{ll}
\displaystyle{{1\over h}} & \mbox{if }x\in [x_{i-1}, x_i);\\
\displaystyle{{-1\over h}} & \mbox{if }x\in [x_i, x_{i+1});\\
0 & \mbox{otherwise};
\end{array}\right.
\]
and 

\[
\phi_{n+1}'(x) = \left\{
\begin{array}{ll}
\displaystyle{{1\over h}} & \mbox{if }x\in [x_{i-1}, x_i);\\
0 & \mbox{otherwise};
\end{array}\right.
\]

When the hat functions $\phi_i$ and $\phi_j$ are not neighbors, i.e., |i-j|>1, 

\[ \int_0^1 \phi_i \phi_j = 0 \:\: \mbox{and }  \int_0^1 \phi_i' \phi_j' = 0 \]
Thus we only need to consider the following cases:
\begin{eqnarray*}
K_{ii} = a(\phi_i, \phi_i) &=& \int_0^1 \phi_i' \phi_i'\\
												&=& \int_{x_{i-1}}^{x_i}  \phi_i' \phi_i' + \int_{x_{i}}^{x_{i+1}}  \phi_i' \phi_i'\\
												&=&\int_{x_{i-1}}^{x_i}  (\frac{1}{h}) (\frac{1}{h}) + \int_{x_{i}}^{x_{i+1}}  (-\frac{1}{h}) (-\frac{1}{h}) \\
												&=& \frac{2}{h}
\end{eqnarray*}
Since $[x_i, x_{i+1}]$ is the only interval where $\phi_i$ and $\phi_{i+1}$ and their derivative  are not $0$ we have 
\begin{eqnarray*}
K_{i(i+1)} = a(\phi_i, \phi_{i+1}) &=& \int_0^1 \phi_i' \phi_{i+1}'\\
												&=& \int_{x_{i}}^{x_{i+1}}  \phi_i' \phi_{i+1}'\\
												&=&\int_{x_{i}}^{x_{i+1}} (-\frac{1}{h}) (\frac{1}{h}) \\
												&=& -\frac{1}{h}
\end{eqnarray*}
By symmetry $K_{i(i+1)}= K_{(i+1)i}$.
\begin{eqnarray*}
K_{(n+1)(n+1)} = a(\phi_{n+1}, \phi_{n+1}) &=& \int_0^1 \phi_{n+1}' \phi_{n+1}'\\
												&=& \int_{1-h}^{1}  \phi_{n+1}' \phi_{n+1}'\\
												&=&\int_{1-h}^{1}  (\frac{1}{h}) (\frac{1}{h}) \\
												&=& \frac{1}{h}
\end{eqnarray*}

Therefore

\[
K= \frac{1}{h}\left[\begin{array}{rrrrrrr}
              2 & -1 & 0 & 0 &  \cdots & 0 \\[0.25em]
               -1 & 2 & -1 &0 & \cdots & 0 \\
                0& -1 & 2 & -1 & \cdots & 0\\
                 \vdots & & &\ddots &  & \vdots \\[0.25em]
                 0 & \cdots &  & -1& 2 & -1 \\ 
                 0 & &\cdots &  & -1 & 1 
                   \end{array}\right]
                   \]
Similar way the matrix $M$, can be defined and using the result of previous homework we can find the integrals.

\begin{eqnarray*}
M_{ii} = (\phi_i, \phi_i) &=& \int_0^1 \phi_i \phi_i\\
												&=& \int_{x_{i-1}}^{x_i}  \phi_i \phi_i + \int_{x_{i}}^{x_{i+1}}  \phi_i \phi_i\\
												&=&\int_{x_{i-1}}^{x_i}  (\frac{x-x_{i-1}}{h}) ( \frac{x-x_{i-1}}{h}) + \int_{x_{i}}^{x_{i+1}}   (\frac{x_{i+1}-x}{h}) (\frac{x_{i+1}-x}{h}) \\
												&=& \frac{2h}{3} \qquad \mbox{by the previous homework}
\end{eqnarray*}
Since $[x_i, x_{i+1}]$ is the only interval where $\phi_i$ and $\phi_{i+1}$ 
\begin{eqnarray*}
M_{i(i+1)} = (\phi_i, \phi_{i+1}) &=& \int_0^1 \phi_i \phi_{i+1}\\
												&=& \int_{x_{i}}^{x_{i+1}}  \phi_i \phi_{i+1}\\
												&=&\int_{x_{i}}^{x_{i+1}}   (\frac{x_{i+1}-x}{h})  (\frac{x-x_{i}}{h}) \\
												&=& \frac{h}{6} \qquad \mbox{by the previous homework}
\end{eqnarray*}
By symmetry $M_{i(i+1)}= M_{(i+1)i}$.
\begin{eqnarray*}
M_{(n+1)(n+1)} = (\phi_{n+1}, \phi_{n+1}) &=& \int_0^1 \phi_{n+1} \phi_{n+1}\\
												&=& \int_{1-h}^{1}  \phi_{n+1} \phi_{n+1}\\
												&=&\int_{1-h}^{1}  \frac{x - x_{n}}{h}  \frac{x - x_{n}}{h}\\
												&=& \frac{h}{3}
\end{eqnarray*}

Therefore

\[
M= \left[\begin{array}{rrrrrrr}
              \frac{2h}{3} &  \frac{h}{6} & 0 & 0 &  \cdots & 0 \\[0.25em]
               \frac{h}{6}  & \frac{2h}{3} & \frac{h}{6} &0 & \cdots & 0 \\
                0& \frac{h}{6} & \frac{2h}{3} & \frac{h}{6} & \cdots & 0\\
                 \vdots & & &\ddots &  & \vdots \\[0.25em]
                 0 & \cdots &  & \frac{h}{6}& \frac{2h}{3} & \frac{h}{6} \\ 
                 0 & &\cdots &  & \frac{h}{6} & \frac{h}{3} 
                   \end{array}\right]
                   \] 


\item We would like to show $A$ is positive definite by showing $x^T A x > 0$. It is obvious $A^T = A$. From part (a) we  have 

\[ \sum_{j=1}^{n+1}\left( a( \phi_j, \phi_i) + (\phi_j, \phi_i)\right) c_j   = (f,\phi_i) \qquad \mbox{for} i=1, \cdots n+1\]
Then  we can say that 
\[
A_{ij}= a( \phi_j, \phi_i) + (\phi_j, \phi_i)
\]
Let $0 \neq x \in R^{n+1}$ 

\begin{eqnarray*}
x^T A x&=& \sum_{i=1}^{n+1} \sum_{j=1}^{n+1} x_i A_{ij}  x_j\\
			 &=& \sum_{i=1}^{n+1} \sum_{j=1}^{n+1}  x_i (a( \phi_i, \phi_j) + (\phi_i, \phi_j)) x_j \\
			 &=& a( \sum_{i=1}^{n+1} x_i\phi_i, \sum_{j=1}^{n+1} x_j\phi_j) + ( \sum_{i=1}^{n+1} x_i\phi_i, \sum_{j=1}^{n+1} x_j\phi_j)
\end{eqnarray*}
Since $x\neq 0 $ and $\phi_i$ is linearly independent 
\[
\sum_{i=1}^{n+1} x_i\phi_i \neq 0
\] 
Thus
 \[
 a( \sum_{i=1}^{n+1} x_i\phi_i, \sum_{j=1}^{n+1} x_j\phi_j) > 0 \:\: \mbox{and} \:\:( \sum_{i=1}^{n+1} x_i\phi_i, \sum_{j=1}^{n+1}  x_j\phi_j) > 0 
 \]

Therefore $ x^T A x >0 $ for any $x\neq 0 $. Thus $A$  is positive definite. 

\item To solve the system first we need to find $(f, \phi_i)$

\begin{eqnarray*}
(f,\phi_i)= \int_0^1 f \phi_i = \int_{x_{i-1}}^{x_{i+1}} f \phi_i &=& \int_{x_{i-1}}^{x_i} f \phi_i + \int_{x_{i}}^{x_{i+1}} f \phi_i\\
																														      	&=& \int_{x_{i-1}}^{x_i} x(1-x) \frac{x-x_{i-1}}{h} \int_{x_i}^{x_{i+1}} x(1-x) \frac{x_{i+1}-x}{h}\\
																																	&=&\frac{-h^2}{12}(2h - 12 i + 12 h i^2)
\end{eqnarray*}

for $i= 1, \cdots, n$. Now, for $i=n+1$ we have 


\begin{eqnarray*}
(f,\phi_{n+1})= \int_0^1 f \phi_{n+1} &=& \int_{1-h}^{1} f \phi_{n+1} \\
																		&=& \int_{1-h}^{1} x(1-x) \frac{x-x_{n}}{h}\\
																		&=& \frac{-h^2}{12}(h-2)																		
\end{eqnarray*}

The requested plot is below.

\begin{center}\includegraphics[scale=0.7]{hw92c.eps}\end{center}
\begin{center}\includegraphics[scale=0.7]{hw92c8.eps}\end{center}

The above plot was produced using the following MATLAB code.

\lstinputlisting{HW92c.m}

\item With the non homogeneous Neumann Boundary conditions our weak formulation becomes as follows

\begin{eqnarray*}
\int_0^1(-u'') v dx+ \int_0^1 u v dx &=& \left[-u' v\right]_0^1 + \int_0^1 u'v' dx + \int_0^1 u v dx \\
&=&\left(-u'(1) v(1) + u'(0)v(0)\right) + \int_0^1 u'v' dx +  \int_0^1u v dx \\
&=& -1 . v(1)+ \int_0^1 u'v' dx + \int_0^1 u v dx \:\: \mbox{(by u'(1)=1 condition and $v(0)=0$)} \\
&=& -1 . v(1)+ a(u,v)+ (u,v)
\end{eqnarray*}

Thus the variational form is
\[ a(u,v)+ (u,v)  = (f,v) + v(1) \qquad \forall v \in V\]

Now define the finite the finite element space $V_n$ then our weak form becomes 

\[ \sum_{j=1}^{n+1}\left( \underbrace{a( \phi_j, \phi_i)}_{K_{ij}} + \underbrace{(\phi_j, \phi_i)}_{M_{ij}}\right) c_j   = (f,\phi_i) + \phi_i(1) \qquad  i=1, \cdots n+1\]

However $\phi_i(1)=0$ for $i=1, \cdots n$. It is only nonzero when $\phi_{n+1}(1)=1$. So the matrix $A$ and the load vector $f$ are same with the part (d). Only we need to modify the last column of the $f$ 

\[
f_{n+1}=(f,\phi_{n+1})+1
\]

The requested plot is below.

\begin{center}\includegraphics[scale=0.7]{hw92d4.eps}\end{center}
\begin{center}\includegraphics[scale=0.7]{hw92d8.eps}\end{center}
\begin{center}\includegraphics[scale=0.7]{hw92d16.eps}\end{center}

The above plot was produced using the following MATLAB code.

\lstinputlisting{HW92d.m}

\end{enumerate}
\end{solution}
}{}

