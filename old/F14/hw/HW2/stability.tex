Consider the time-dependent heat equation with no source
\begin{align*}
\pd{u}{t}{}-\kappa\pd{u}{x}{2} &= 0, \quad x\in (0,1)\\
u(0,t) &= u_0\\
u(1,t) &= u_1\\
u(x,0) &= \psi(x).
\end{align*}
Before we even try to solve this equation over time, it would be good to verify that this equation is \textit{stable} in time --- in other words, that $u(x,t)$ doesn't blow up as $t\rightarrow \infty$.  This can be done by deriving an ``energy estimate''.
\begin{enumerate}
\item Multiply the time-dependent heat equation by the solution $u(x,t)$ and integrate over $x$ in the domain $(0,1)$ to get
\[
\int_0^1 \left(\pd{u}{t}{}u - \kappa\pd{u}{x}{2}u\right) dx= 0
\]
Using the fact that for a function $f(t)$, 
\[
\pd{f}{t}{}f = \frac{1}{2}\pd{f^2}{t}{}
\]
as well as integration by parts, derive the energy estimate
\[
\pd{}{t}{} \int_0^1 u^2 + \kappa \int_0^1 \pd{u}{x}{}^2 = 0.
\]
\item Consider the zero-flux boundary conditions
\begin{align*}
\pd{u}{t}{}-\kappa\pd{u}{x}{2} &= 0, \quad x\in (0,1)\\
u'(0,t) &= 0\\
u'(1,t) &= 0\\
u(x,0) &= \psi(x).
\end{align*}

\item 
%\item We noted in the previous homework that, under boundary conditions
%\[
%u'(0) = u'(1) = 0
%\]
%that the steady state heat equation does not have a unique solution, and that for any $u(x)$ satisfying the steady heat equation, that $u + C$ for any constant $C$ was also a solution to the same steady state heat equation.  We may use the above time-stability 
%
%This implies an important fact: that 
\end{enumerate}


%%%%%%%%%%%%%%%%%%%%%%%%%%%%%%%%%%%%%%%%%%%%%%%%%%%%%%%%%%%%

\ifthenelse{\boolean{showsols}}{\input solve_sol}{
%Rewrite the finite difference approximation at point $x_N$ as
%\[
%\frac{u(x_{N-1}) - 2u(x_N) + u(x_{N+1})}{h^2} = \frac{u(x_{N-1}) - u(x_N)}{h^2} + \frac{u(x_{N+1})-u(x_N)}{h^2}.
%\]
}

