Define the inner product $\ip{u,v}$ to be
\[
\ip{u,v}=\int_0^1u(x)v(x)\,dx
\]
and let the norm $\norm{v(x)}$ be defined by
\[
\norm{v}=\sqrt{\ip{v,v}}.
\]
Let $N$ be a positive integer and let $\phi_1,\ldots,\phi_N\in C[0,1]$ be such that $\left\{\phi_1,\ldots,\phi_N\right\}$ is orthonormal with respect to the inner product $\ip{\cdot,\cdot}$. We wish to approximate a continuous function $f(x)$ with $f_N(x)$
\[
f_N(x)=\sum_{n=1}^N\alpha_n\phi_n(x)
\]
where
\[
\phi_n(x) = \sqrt{2} \sin(n \pi x),\quad n=1,2,\ldots.
\]
and where $\alpha_n=\ip{f,\phi_n}$. (Note that $f_N$ is the best approximation to $g$ from ${\rm span}\left\{\phi_1,\ldots,\phi_N\right\}$ with respect to the norm $\norm{\cdot}$.)

\begin{enumerate}
\item Assume that $f_N \rightarrow f$ as $N\rightarrow \infty$.  Show that, since $\phi_1,\ldots,\phi_N$ are orthonormal, 
\[
\norm{f-f_N}^2  = \norm{f}^2 - \sum_{n=1}^N \alpha_n^2.
\]
\item The best approximation to $f(x) = x(1-x)$ has coefficients $\alpha_n$ which satisfy
\[
\alpha_n = \frac{2\sqrt{2}}{n^3\pi^3}(1-(-1)^n).
\]
Plot the true function $f(x)$ and compare it to $f_N(x)$ for $N = 5$.  On a separate figure, plot the error using the above formula for $N = 1,2,\ldots,100$ on a log-log scale by using \verb|loglog| in \textsc{Matlab}.  
\item Verify that the best approximation to the function $f(x) = 1-x$ (which does not satisfy the same boundary conditions as $\phi_n(x)$!) has coefficients
\[
\alpha_n =  \frac{\sqrt{2}}{\pi n}.
\]
Plot the true function $f(x)$ and compare it to $f_N(x)$ for $N = 100$.  On a separate figure, plot the error using the above formula for $N = 1,2,\ldots,100$ on a log-log scale by using \verb|loglog| in \textsc{Matlab}.  

\vspace{2em}
\emph{You may have noticed that the rate at which the coefficients $\alpha_n \rightarrow 0$ determines how fast the error decreases --- this is not coincidental!}  %Time permitting, we will explore this idea more in the future.  %In the meantime, food for thought: what happens if $\alpha_n = 1/N$, and never decays?  Does $f_N(x)$ approach a particular ``function''?


\end{enumerate}

%%%%%%%%%%%%%%%%%%%%%%%%%%%%%%%%%%%%%%%%%%%%%%%%%%%%%%%%%%%%%%%%%%%%%%%%%%%%%%%%

\ifthenelse{\boolean{showsols}}{\input gibbs_sol}{}