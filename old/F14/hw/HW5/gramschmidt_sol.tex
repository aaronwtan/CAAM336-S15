\begin{solution}
\begin{enumerate}

\item We are going to show that $\ip{\phi_i, \phi_j}=0$ if $1 \leq i \neq j\leq 3$. To check that these formulas yield an orthogonal sequence, first compute $\ip{\phi_1, \phi_2}$ by substituting the above formula for $\phi_2$
\begin{eqnarray*}
\ip{ {\phi}_1, {\phi}_2} &=& \ip{{ \phi}_1,  {v}_2-\frac{({ \phi}_1, { v}_2)}{({ \phi}_1, { \phi}_1)} { \phi}_1} \\
                         &=& \ip{{ \phi}_1, { v}_2}-\ip{{\phi}_1 ,\frac{({\phi}_1, { v}_2)}{({ \phi}_1, { \phi}_1)} { \phi}_1} \\ 
                         &=&\ip{{ \phi}_1, { v}_2}- \frac{({ \phi}_1, { v}_2)}{({ \phi}_1, { \phi}_1)}\ip{{\phi}_1 ,{ \phi}_1} \\ 
                         &=&\ip{{ \phi}_1, { v}_2}-\ip{{ \phi}_1, { v}_2}\\
                         &=& 0.                                
\end{eqnarray*}

Then use the fact that $\ip{ {\phi}_1, {\phi}_2}=0 $, to compute $\ip{\phi_1, \phi_3}$. By substituting again the formula for $\phi_3$

\begin{eqnarray*}
\ip{ {\phi}_1, {\phi}_3} &=& \ip{{ \phi}_1,  { v}_3-\frac{({ \phi}_1, { v}_3)}{({ \phi}_1, { \phi}_1)} { \phi}_1 - \frac{({ \phi}_2, { v}_3)}{({ \phi}_2, { \phi}_2)} { \phi}_2}\\
                         &=&\ip{{ \phi}_1, { v}_3}-\ip{{\phi}_1 ,\frac{({\phi}_1, { v}_3)}{({ \phi}_1, { \phi}_1)} { \phi}_1} - \ip{{\phi}_1 ,\frac{({\phi}_2, { v}_3)}{({ \phi}_2, { \phi}_2)} { \phi}_2} \\ 
                         &=&\ip{{ \phi}_1, { v}_3}- \frac{({ \phi}_1, { v}_3)}{({ \phi}_1, { \phi}_1)}\ip{{\phi}_1 ,{ \phi}_1} - \frac{({\phi}_2, { v}_3)}{({ \phi}_2, { \phi}_2)} \underbrace{\ip{{\phi}_1 , { \phi}_2}}_{=0}\\ 
                         &=&\ip{{ \phi}_1, { v}_3}-\ip{{ \phi}_1, { v}_3}\\
                         &=& 0                                 
\end{eqnarray*}


Similarly, using the symmetry property of inner product $ \ip{ {\phi}_i, {\phi}_j}= \ip{ {\phi}_j, {\phi}_i}$ for all $i,j$. We can show $\ip{ {\phi}_2, {\phi}_3}=0.$
 
\begin{eqnarray*}
\ip{ {\phi}_2, {\phi}_3} &=& \ip{{ \phi}_2,  { v}_3-\frac{({ \phi}_1, { v}_3)}{({ \phi}_1, { \phi}_1)} { \phi}_1 - \frac{({ \phi}_2, { v}_3)}{({ \phi}_2, { \phi}_2)} { \phi}_2}\\
                         &=&\ip{{ \phi}_2, { v}_3}-\ip{{\phi}_2 ,\frac{({\phi}_1, { v}_3)}{({ \phi}_1, { \phi}_1)} { \phi}_1} - \ip{{\phi}_2 ,\frac{({\phi}_2, { v}_3)}{({ \phi}_2, { \phi}_2)} { \phi}_2} \\ 
                         &=&\ip{{ \phi}_2, { v}_3}- \frac{({ \phi}_1, { v}_3)}{({ \phi}_1, { \phi}_1)} \underbrace{\ip{{\phi}_2 ,{ \phi}_1}}_{=0} - \frac{({\phi}_2, { v}_3)}{({ \phi}_2, { \phi}_2)} \ip{{\phi}_2 , { \phi}_2}\\ 
                         &=&\ip{{ \phi}_2, { v}_3}-\ip{{ \phi}_2, { v}_3}\\
                         &=& 0.                                 
\end{eqnarray*}

By symmetry we can conclude that $ \ip{ {\phi}_2, {\phi}_3}= \ip{ {\phi}_3, {\phi}_2}=0 $ and $ \ip{ {\phi}_1, {\phi}_3}= \ip{ {\phi}_3, {\phi}_1}=0 $. This completes the proof. 
  
\item Consider a linear relationship 
\[
\sum_{i = 1}^k \alpha_i\phi_i = 0
\]

which can be written
\[
\alpha_1\phi_1 + \alpha_2\phi_2+ \cdots + \alpha_k\phi_k =0.
\]

If $ 1 \leq i \leq k$ then taking the inner product of  $\phi_i$  with both sides of the equation and using the properties of inner product (\emph{Definition 3.32, page 58}),
\begin{eqnarray*}
\ip{\phi_i , \alpha_1\phi_1 + \alpha_2\phi_2+ \cdots + \alpha_k\phi_k} &=& \ip{ \phi_i , 0 } \\
\ip{\phi_i , \alpha_1\phi_1} + \ip{ \phi_i , \alpha_2\phi_2 } + \cdots + \ip{ \phi_i , \alpha_k\phi_k} &=& 0\\
\alpha_1 \ip{\phi_i , \phi_1} +  \alpha_2 \ip{ \phi_i , \phi_2 } + \cdots + \alpha_k \ip{ \phi_i , \phi_k} &=& 0\\
\alpha_i \ip{\phi_i , \phi_i} &=& 0
\end{eqnarray*}
shows, since  $\phi_i$ is nonzero, that  $\alpha_i$ for $i=1, \cdots , k$ is zero.

\item We want to construct the new orthogonal bases for $V$ by \emph{Gram-Schmidt} procedure  given starting vectors $\{v_1, v_2, v_3\}=\{1, x, x^2\}$. Following the procedure we set
\begin{eqnarray*}
{ \phi}_1 &=& { v}_1 = 1 \\
\end{eqnarray*}
and 
\begin{eqnarray*}
{ \phi}_2 &=& { v}_2-\frac{({ \phi}_1, { v}_2)}{({ \phi}_1, { \phi}_1)} { \phi}_1.
\end{eqnarray*}

We compute
\[
\ip{{ \phi}_1, { v}_2} = \int_{-1}^1 x \, dx = \left[\frac{x^2}{2}\right]_{-1}^{1}= 0
\]
and
\[
\ip{{ \phi}_1, {\phi }_1} = \int_{-1}^1 1 \, dx =  2.
\]
Now we can compute  
\begin{eqnarray*}
{ \phi}_2 &=& x - \frac{0}{2} (1) =x .
\end{eqnarray*}

Finally for $\phi_3$,
\begin{eqnarray*}
{ \phi}_3 &=& { v}_3-\frac{({ \phi}_1, { v}_3)}{({ \phi}_1, { \phi}_1)} { \phi}_1 - \frac{({ \phi}_2, { v}_3)}{({ \phi}_2, { \phi}_2)} { \phi}_2
\end{eqnarray*}
\[
\ip{ \phi_1,  v_3 } = \int_{-1}^1 x^2 \, dx = \left[\frac{x^3}{3}\right]_{-1}^{1}= \frac{2}{3} 
\]
and 
\[
\ip{ \phi_2,  v_3 } = \int_{-1}^1 x^3 \, dx = \left[\frac{x^4}{4}\right]_{-1}^{1}= 0 
\]
and
\[
\ip{ \phi_2,  \phi_2 } = \int_{-1}^1 x^2 \, dx = \left[\frac{x^3}{3}\right]_{-1}^{1}= \frac{2}{3}  
\]

Substituting these inner products into the  equation for $\phi_3$, we get 
\begin{eqnarray*}
{ \phi}_3 &=& x^2-\frac{(2/3)}{2} (1)  - \frac{0}{(2/3)} (x) = x^2 -\frac{1}{3}.
\end{eqnarray*}

This yields $ \{{\phi}_1, {\phi}_2, { \phi}_3\} = \{1, x , x^2 -\frac{1}{3}\} $ as desired. 
\end{enumerate}
\end{solution}