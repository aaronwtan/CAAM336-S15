In this problem we'll consider a linear operator mapping to and from a very specific vector space, and use it to explore what an operator inverse can look like.  

Consider the $V$ defined as
\[
V = \left\{ u(x) = \sum_{j = 1}^N c_j \sin( j \pi x), \quad c_j \in \R \right\}.
\]
In other words, $V$ is the set of all functions that are linear combinations of a finite number of different sine functions.  This means that, for each $u\in V$, there is a set of coefficients $c_1,\ldots, c_N$ that is also associated with $u$.  

\begin{enumerate}
\item Show that $V$ is a subspace of the vector space $C^2_D$, where 
\[
C^2_D = \{ u(x)\in C^2[0,1], \quad u(0) = u(1) = 0\}.
\]
\item Let the operator $L$ be defined as
\[
Lu = -\frac{\partial^2 u}{ \partial x^2}
\]
Show that, for $u\in V$, $Lu\in V$.  This shows that $L$ can be viewed as 
\[
L: V\rightarrow V,
\]
a map from $V$ to $V$.
\item We can define the operator $\tilde{L}: V\rightarrow V$ as
\[
\tilde{L} u =  \sum_{j = 1}^N \frac{c_j}{(j\pi)^2} \sin( j \pi x).
\]
Show that both $L\tilde{L}u = u$ and $\tilde{L}Lu = u$ for any $u\in V$.  
\end{enumerate}
Since both $L\tilde{L}u = u$ and $\tilde{L}Lu = u$ for any $u\in V$, we can refer to $\tilde{L}$ as the inverse $L^{-1}$ of $L:V\rightarrow V$.  

%%%%%%%%%%%%%%%%%%%%%%%%%%%%%%%%%%%%%%%%%%%%%%%%%%%%%%%%%%%%%%%%%%%%%%%%%%%%%%%%

\ifthenelse{\boolean{showsols}}{\input fourier_sol}{}

