\begin{solution}
\begin{enumerate}

\item We are going to show that $\ip{\phi_i, \phi_j}=0$ if $1 \leq i \neq j\leq 3$. To check that these formulas yield an orthogonal sequence, first compute $\ip{\phi_1, \phi_2}$ 
\begin{eqnarray*}
\left({ \phi}_1, {\phi}_2\right) &=& \left({ \phi}_1,  {v}_2-\frac{({ \phi}_1, { v}_2)}{({ \phi}_1, { \phi}_1)} { \phi}_1\right) \\
                                 &=& \left({ \phi}_1, { v}_2\right)- \left({ \phi}_1 ,\frac{({ \phi}_1, { v}_2)}{({ \phi}_1, { \phi}_1)} { \phi}_1\right) \\ 
                                 &=&\left({ \phi}_1, { v}_2\right)- \frac{({ \phi}_1, { v}_2)}{({ \phi}_1, { \phi}_1)}\left({ \phi}_1 ,{ \phi}_1\right) \\ 
                                 &=&\left({ \phi}_1, { v}_2\right)- \left({ \phi}_1, { v}_2\right)\\
                                 &=& 0
                                 
\end{eqnarray*}

 Let $ V$ be an inner product space (i.e. $V$ a vector space with an inner product). Suppose  $ \{{v}_1, {v}_2, { v}_3\} $ is a basis for $V$, and we would like to construct a is possible to construct a new \emph{orthogonal} basis $ \{{ \phi}_1, { \phi}_2, { \phi}_3 \}$ through the following procedure:
\begin{eqnarray*}
{ \phi}_1 &=& { v}_1 \\
{ \phi}_2 &=& { v}_2-\frac{({ \phi}_1, { v}_2)}{({ \phi}_1, { \phi}_1)} { \phi}_1\\
{ \phi}_3 &=& { v}_3-\frac{({ \phi}_1, { v}_3)}{({ \phi}_1, { \phi}_1)} { \phi}_1 - \frac{({ \phi}_2, { v}_3)}{({ \phi}_2, { \phi}_2)} { \phi}_2\\
\vdots\\
{ \phi}_k &=& { v}_k- \sum_{i = 1}^{k-1}\frac{({ \phi}_i, { v}_k)}{({ \phi}_i, { \phi}_i)} { \phi}_i
\end{eqnarray*}
%such that $(\cdot, \cdot)$ is a inner product defined on a vector space $V$, in other words $\ip{\cdot,\cdot}:\mbox{ }V \times V \rightarrow \R$.
This is called the \emph{Gram-Schmidt} procedure.  
\begin{enumerate}
%\item Two or more functions or vectors $ v_1, v_2, ..., v_n$ are called linearly independent if the only scalars $c_1, c_2, ..., c_n$ satisfying 
%\[
%c_1 v_1 + c_2 v_2 + ... +  c_n v_n=0
%\]
%are $c_1 = c_2 = ...=c_n=0.$
%Using the fact  $ \{{v}_1, {v}_2, { v}_3\} $  are linearly independent, show that  $ \{{\phi}_1, {\phi}_2, { \phi}_3\} $ also linearly independent set of vectors.
\item We know that nonzero vectors $v_1, v_2, . . . , v_k \in V$ form an orthogonal set if they are orthogonal to
each other: i.e. if 
\[
\ip{v_i, v_j} = 0, \quad i\neq j.
\]
Show that $\phi_1,\phi_2,\phi_3$ form an orthogonal set, i.e. $\ip{\phi_i, \phi_j}=0$ if $1 \leq i \neq j\leq 3$.  
\item Show that if we have an orthogonal set of vectors $\phi_1,\ldots, \phi_k$, then $\phi_i,\ldots, \phi_k$ are linearly independent as well, i.e. 
\[
\sum_{i = 1}^k \alpha_i\phi_i = 0
\]
is only true if $\alpha_1, \ldots, \alpha_k = 0$. 
\item Since we can define an inner product $\ip{\cdot,\cdot}$ on the function space $C[-1,1]$ as
\[
\ip{u,v} = \int_{-1}^1 u(x)v(x)\, dx,
\]
we can also use the Gram-Schmidt procedure to create orthogonal sets of \emph{functions}.  Using the Gram-Schmidt procedure above, compute the orthogonal vectors $ \{{\phi}_1, {\phi}_2, { \phi}_3\} $ given starting vectors $\{v_1, v_2, v_3\}=\{1, x, x^2\}$. 
\end{enumerate}
\end{solution}