\begin{solution}

Let $V$ be defined as
\[
V = \left\{ u\in C_D^2[0,1]: u(x) = \sum_{j = 1}^N c_j \sin( j \pi x), \quad c_j \in \R \right\}.
\]
\begin{enumerate}
\item Suppose $u(x) \in V$.  Then, 
\[
u(0) =  \sum_{j = 1}^N c_j \sin( 0), \qquad  u(1) = \sum_{j = 1}^N c_j \sin( j \pi )
\]
Since $\sin(0) = 0$, $u(0) = 0$.  Similarly, since $j$ is an integer in the above expression, $\sin(j\pi 1) = \sin(j\pi) = 0$, since sine evaluated at any multiple of $\pi$ is zero, and $u(1) = 0$ as well.  Since the sum of sines is a continuous function, any $u(x)\in V$ is also contained in $C^2_D[0,1]$, and $V$ is contained in $C^2_D[0,1]$.  
 
 
Next, we just need to show that $V$ itself is a vector space.  If $u(x)$ and $v(x)$ are in the set of $V$, then we have the representations
\[
u(x)= \sum_{j = 1}^N d_j \sin( j \pi x), \qquad v(x)= \sum_{j = 1}^N e_j \sin( j \pi x)
\]
for $d_j, e_j \in \R$.  We wish to show that, for any $\alpha, \beta \in \R$, the sum $w(x) = \alpha u(x) +\beta v(x)$ is contained in $V$ as well.  
\begin{eqnarray*}
w(x)= \alpha u(x) +\beta v(x)&=& \alpha \sum_{j = 1}^N d_j \sin( j \pi x) + \beta \sum_{j = 1}^N e_j \sin( j \pi x)\\
&=& \sum_{j = 1}^N \underbrace{\alpha d_j}_{\tilde{d}_j} \sin( j \pi x) + \sum_{j = 1}^N \underbrace{\beta e_j}_{\tilde{e}_j} \sin( j \pi x)\\
&=& \sum_{j = 1}^N \underbrace{(\tilde{d}_j + \tilde{e}_j)}_{c_j} \sin( j \pi x)\\
&=& \sum_{j = 1}^N c_j \sin( j \pi x) 
\end{eqnarray*}
Therefore $w \in V$, and since $V$ is contained in $C^2_D[0,1]$ as well, $V$ is a subspace of $C_D^2[0,1]$.

\item The operator $L$ be defined as
\[
Lu = -\frac{\partial^2 u}{\partial x^2}.
\]
We are going to show that for all $u\in V$, $Lu\in V$ as well. Let
\[
u= \sum_{j = 1}^N c_j \sin( j \pi x) \in V 
\]
for $c_j \in \R$. Then
\begin{eqnarray*}
Lu = -\frac{\partial^2 u}{\partial x^2} &=& -\frac{\partial^2}{ \partial x^2}\left( \sum_{j = 1}^N c_j \sin( j \pi x)\right)\\
                               &=&-\frac{\partial}{\partial x}\left[ \sum_{j = 1}^N c_j \frac{\partial (\sin( j \pi x))}{\partial x} \right]\\
                               &=&-\frac{\partial}{\partial x}\left[ \sum_{j = 1}^N c_j (j \pi ) \cos( j \pi x)  \right]\\
                               &=& -\sum_{j = 1}^N c_j (j \pi ) \frac{ \partial {(\cos( j \pi x))}}{\partial x}\\
                               &=& \sum_{j = 1}^N c_j (j \pi )^2 \sin( j \pi x).
\end{eqnarray*}
If we define $\tilde{c}_j= c_j (j \pi )^2 \in \R$ then 
\[
Lu = \sum_{j = 1}^N \tilde{c}_j \sin( j \pi x)
\]
which is also in $V$ for all $ \tilde{c}_j \in \R$. This shows that $L$ can be viewed as 
\[
L: V\rightarrow V,
\]
a map from $V$ to $V$.

\item We define the operator $\tilde{L}: V\rightarrow V$ as
\[
\tilde{L} u =  \sum_{j = 1}^N \frac{c_j}{(j\pi)^2} \sin( j \pi x).
\]
We will show that $L\tilde{L}u = u$ and $\tilde{L}Lu = u$ for any $u\in V$. First, let us show $L\tilde{L}u = u$. 

From part (b) we can deduce that if $u= \sum_{j = 1}^N c_j \sin( j \pi x) $
\[
Lu= L\left(\sum_{j = 1}^N c_j \sin( j \pi x)\right)= \sum_{j = 1}^N c_j (j \pi )^2 \sin( j \pi x)
\]
Then
\begin{eqnarray*}
L\tilde{L}u = L (\tilde{L}u) &=& L \left(\sum_{j = 1}^N \frac{c_j}{(j\pi)^2} \sin( j \pi x)\right)\\
														 &=& \sum_{j = 1}^N \frac{c_j (j \pi)^2}{(j\pi)^2} \sin( j \pi x)\\
														 &=& \sum_{j = 1}^N c_j \sin( j \pi x)\\
														 &=&u
\end{eqnarray*}

Now, let us show $\tilde{L}Lu=u$ as well. Remember that from problem if $u= \sum_{j = 1}^N c_j \sin( j \pi x) $ then $\tilde{L}u$ is given as follows
\[
\tilde{L} u = \tilde{L}\left(\sum_{j = 1}^N c_j \sin( j \pi x)\right) = \sum_{j = 1}^N \frac{c_j}{(j\pi)^2} \sin( j \pi x)
\]

Then, similarly
\begin{eqnarray*}
\tilde{L}Lu = \tilde{L}(Lu) &=& \tilde{L} \left(\sum_{j = 1}^N c_j (j\pi)^2 \sin( j \pi x)\right)\\
														 &=& \sum_{j = 1}^N \frac{c_j (j \pi)^2}{(j\pi)^2} \sin( j \pi x)\\
														 &=& \sum_{j = 1}^N c_j \sin( j \pi x)\\
														 &=&u
\end{eqnarray*}

Since both $L\tilde{L}u = u$ and $\tilde{L}Lu = u$ for any $u\in V$, we can refer to $\tilde{L}$ as the inverse $L^{-1}$ of $L:V\rightarrow V$.  

\end{enumerate}
\end{solution}