\begin{solution}
\begin{enumerate}
\item {[6 points]} If $u\in C_z^1[-1,1]$ and $v\in C_z^1[-1,1]$ then
\[
B\ip{u,v}=a\ip{u,v}+\ip{u,v}=a\ip{v,u}+\ip{v,u}=B\ip{u,v}
\]
since $a\ip{u,v}=a\ip{v,u}$ and $\ip{u,v}=\ip{v,u}$ because $a\ip{\cdot,\cdot}$ and $\ip{\cdot,\cdot}$ are inner products on $C_z^1[-1,1]$. So
\[
B\ip{u,v}=B\ip{u,v}\mbox{ for all }u,v\in C_z^1[-1,1].
\]

If $u,v,w\in C_z^1[-1,1]$ and $\alpha,\beta\in\R$ then
\begin{eqnarray*}
&&B\ip{\alpha u+\beta v,w}
\\
&=&a\ip{\alpha u+\beta v,w}+\ip{\alpha u+\beta v,w}
\\
&=&\alpha a\ip{u,w}+\beta a\ip{v,w}+\alpha\ip{u,w}+\beta\ip{v,w}
\\
&=&\alpha\left(a\ip{u,w}+\ip{u,w}\right)+\beta\left(a\ip{v,w}+\ip{v,w}\right)
\\
&=&\alpha B\ip{u,w}+\beta B\ip{v,w}
\end{eqnarray*}
since $a\ip{\alpha u+\beta v,w}=\alpha a\ip{u,w}+\beta a\ip{v,w}$ and $\ip{\alpha u+\beta v,w}=\alpha\ip{u,w}+\beta\ip{v,w}$ because $a\ip{\cdot,\cdot}$ and $\ip{\cdot,\cdot}$ are inner products on $C_z^1[-1,1]$. So
\[
B\ip{\alpha u+\beta v,w}=\alpha B\ip{u,w}+\beta B\ip{v,w}\mbox{ for all }u,v,w\in C_z^1[-1,1]\mbox{ and all }\alpha,\beta\in\R.
\]

If $u\in C_z^1[-1,1]$ then $B\ip{u,u}\ge0$ since $a\ip{u,u}\ge0$ and $\ip{u,u}\ge0$ because $a\ip{\cdot,\cdot}$ and $\ip{\cdot,\cdot}$ are inner products on $C_z^1[-1,1]$. Moreover, if $B\ip{u,u}=0$ then
\[
a\ip{u,u}+\ip{u,u}=0,
\]
or equivalently,
\[
a\ip{u,u}=-\ip{u,u}
\]
and, since $a\ip{u,u}\ge0$ and $\ip{u,u}\ge0$, this can only hold when $a\ip{u,u}=\ip{u,u}=0$ which then implies that $u=0$ because of either the fact that $a\ip{u,u}=0$ only if $u=0$ since $a\ip{\cdot,\cdot}$ is an inner product on $C_z^1[-1,1]$ or the fact that $\ip{u,u}=0$ only if $u=0$ since $\ip{\cdot,\cdot}$ is an inner product on $C_z^1[-1,1]$. So
\[
B\ip{u,u}\ge0\mbox{ for all }u\in C_z^1[-1,1]
\]
with
\[
B\ip{u,u}=0\mbox{ only if }u=0.
\]

Consequently, $B\ip{\cdot,\cdot}$ is an inner product on $C_z^1[-1,1]$.
\\
\item {[2 points]} The best approximation to $f$ from ${\rm span}\{v_1\}$ with respect to the norm $\norm{\cdot}$ is
\[
b_1(x)={\ip{f,v_1} \over \ip{v_1,v_1}}v_1(x)=0
\]
since $\ip{f,v_1}=0$.
\\
\item {[2 points]} The best approximation to $f$ from ${\rm span}\{v_1\}$ with respect to the norm $\norm{\cdot}_a$ is
\[
b_2(x)={a\ip{f,v_1} \over a\ip{v_1,v_1}}v_1(x)={-2 \over 6}v_1(x)={-1 \over 3}{\sqrt{3} \over \sqrt{2}}x={-1 \over \sqrt{6}}x.
\]
\\
\item {[3 points]} The definition of $B\ip{\cdot,\cdot}$ means that
\[
B\ip{v_1,v_1}=a\ip{v_1,v_1}+\ip{v_1,v_1}=6+1=7
\]
and
\[
B\ip{f,v_1}=a\ip{f,v_1}+\ip{f,v_1}=-2+0=-2.
\]
Therefore, the best approximation to $f$ from ${\rm span}\{v_1\}$ with respect to the norm $\norm{\cdot}_B$ is
\[
b_3(x)={B\ip{f,v_1} \over B\ip{v_1,v_1}}v_1(x)={-2 \over 7}{\sqrt{3} \over \sqrt{2}}x=-{\sqrt{6} \over 7}x.
\]
\\
\item {[5 points]} We first compute that
\[
v_1'(x)={\sqrt{3} \over \sqrt{2}}
\]
and
\[
v_2'(x)={\sqrt{3} \over \sqrt{2}}(6x-1)
\]
and then compute that
\begin{eqnarray*}
a\ip{v_1,v_2}=a\ip{v_2,v_1}&=&\int_{-1}^1(2+x)v_1'(x)v_2'(x)\,dx
\\
&=&\int_{-1}^1(2+x){\sqrt{3} \over \sqrt{2}}{\sqrt{3} \over \sqrt{2}}(6x-1)\,dx
\\
&=&{3 \over 2}\int_{-1}^16x^2+11x-2)\,dx
\\
&=&{3 \over 2}\left[2x^3+{1 \over 2}x^2-2x\right]_{-1}^1
\\
&=&{3 \over 2}\left(2+{1 \over 2}-2-\left(-2+{1 \over 2}+2\right)\right)
\\
&=&0.
\end{eqnarray*}
Therefore, $v_1$ and $v_2$ is orthogonal with respect to the inner product $a\ip{\cdot,\cdot}$ and hence the best approximation to $f$ from ${\rm span}\{v_1,v_2\}$ with respect to the norm $\norm{\cdot}_a$ is
\begin{eqnarray*}
b_4(x)&=&{a\ip{f,v_1} \over a\ip{v_1,v_1}}v_1(x)+{a\ip{f,v_2} \over a\ip{v_2,v_2}}v_2(x)
\\
&=&{-2 \over 6}v_1(x)+{-22 \over 66}v_2(x)
\\
&=&-{1 \over 3}{\sqrt{3} \over \sqrt{2}}x-{1 \over 3}{\sqrt{3} \over \sqrt{2}}(3x^2-x-1)
\\
&=&{1 \over \sqrt{6}}(1-3x^2).
\end{eqnarray*}
\\
\item {[7 points]} We first compute that
\begin{eqnarray*}
\ip{v_1,v_2}=\ip{v_2,v_1}&=&\int_{-1}^1{\sqrt{3} \over \sqrt{2}}x{\sqrt{3} \over \sqrt{2}}(3x^2-x-1)\,dx
\\
&=&{3 \over 2}\int_{-1}^1(3x^3-x^2-x)\,dx
\\
&=&{3 \over 2}\left[{3 \over 4}x^4-{1 \over 3}x^3-{1 \over 2}x^2\right]_{-1}^1
\\
&=&{3 \over 2}\left({3 \over 4}-{1 \over 3}-{1 \over 2}-\left({3 \over 4}+{1 \over 3}-{1 \over 2}\right)\right)
\\
&=&{3 \over 2}\left(-{2 \over 3}\right)
\\
&=&-1.
\end{eqnarray*}
Therefore, $v_1$ and $v_2$ are not orthogonal with respect to the inner product $\ip{\cdot,\cdot}$ and hence the best approximation to $f$ from ${\rm span}\{v_1,v_2\}$ with respect to the norm $\norm{\cdot}$ is
\[
b_5(x)=c_1v_1(x)+c_2v_2(x)
\]
where the coefficients $c_1,c_2\in\R$ are such that
\[
\left[\begin{array}{cc} \ip{v_1,v_1} & \ip{v_1,v_2} \\ \ip{v_1,v_2} & \ip{v_2,v_2} \end{array}\right]\left[\begin{array}{c} c_1 \\ c_2 \end{array}\right]=\left[\begin{array}{c} \ip{f,v_1} \\ \ip{f,v_2} \end{array}\right]
\]
and hence are such that
\[
\left[\begin{array}{cc} 1 & -1 \\ -1 & \displaystyle{{17 \over 5}} \end{array}\right]\left[\begin{array}{c} c_1 \\ c_2 \end{array}\right]=\left[\begin{array}{c} 0 \\ \displaystyle{-{12 \over \pi^2}} \end{array}\right],
\]
or equivalently,
\[
\left[\begin{array}{c} c_1-c_2 \\ \displaystyle{-c_1+{17 \over 5}c_2} \end{array}\right]=\left[\begin{array}{c} 0 \\ \displaystyle{-{12 \over \pi^2}} \end{array}\right].
\]
Now, the first rows implies that $c_1=c_2$ which mean that the second row yields that $\displaystyle{\left({17 \over 5}-1\right)c_2=-{12 \over \pi^2}}$ which implies that $\displaystyle{{12 \over 5}c_2=-{12 \over \pi^2}}$ and hence $\displaystyle{c_2=-{5 \over \pi^2}}$. Therefore, the best approximation to $f$ from ${\rm span}\{v_1,v_2\}$ with respect to the norm $\norm{\cdot}$ is
\[
b_5(x)=c_1v_1(x)+c_2v_2(x)=c_2(v_1(x)+v_2(x))=-{5 \over \pi^2}{\sqrt{3} \over \sqrt{2}}(x+3x^2-x-1)={5 \over \pi^2}{\sqrt{3} \over \sqrt{2}}(1-3x^2).
\]
\end{enumerate}
\end{solution}