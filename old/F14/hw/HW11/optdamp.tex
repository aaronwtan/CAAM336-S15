\def\bmatrix#1{\left[\matrix{#1}\right]}
In class we will see that the wave equation $u_{tt} = u_{xx}$ can
be written as a first order system.  We will introduce the velocity variable
\[ v(x,t) = u_t(x,t),\]
then notice that the wave equation is equivalent to the first order equation
\[ {\partial \over \partial t} \bmatrix{u(x,t)\cr v(x,t)}
    = \bmatrix{ 0 & I \cr \partial^2/\partial x^2 & 0} 
      \bmatrix{u(x,t)\cr v(x,t)}\]
where $I$ is the identity operator (i.e., $I v = v$ for any function $v$). 
The first row of this system gives $u_t = v$, while the second row
gives $v_t = u_{xx}$, which is just another way of writing $u_{tt} = u_{xx}$.

We also have the usual initial conditions,
\[ \bmatrix{u(x,0) \cr v(x,0)} = \bmatrix{u_0(x) \cr v_0(x)}.\]
This system can be analyzed in terms of its eigenvalues and eigenfunctions.
In this problem, we will use the same approach to study the damped wave equation
from Problem~2.
\begin{enumerate}
\item Verify that the damped wave equation $u_{tt} = u_{xx} - 2@d@u_t$ is equivalent
      to the system
\[ {\partial \over \partial t} \bmatrix{u(x,t)\cr v(x,t)}
    = \bmatrix{ 0 & I \cr \partial^2/\partial x^2 & -2d I} 
      \bmatrix{u(x,t)\cr v(x,t)}.\]
\item For any value of $d$, the operator
      \[ A(d) = \bmatrix{ 0 & I \cr \partial^2/\partial x^2 & -2@d@I}\]
      has all its eigenvectors of the same form:
      \[ \Psi_{\pm k}(x) = \bmatrix{ \sin(k \pi x)\cr  \lambda_{\pm k} \sin(k\pi x)}.\]
      Use these eigenvectors to determine the eigenvalues $\lambda_{\pm k}$.
\item Plot the eigenvalues from part~(b) in the complex plane for $d=0, \pi/2, \pi$, and
      $3\pi/2$.
\item What value of $d\ge 0$ optimizes the \emph{asymptotic decay} of the system?
      That is, what value of $d$ moves \emph{all} the eigenvalues as far to the
      left in the complex plane as possible?  Said yet another way, which $d$ 
      minimizes the real part of the rightmost eigenvalue?
      Does larger damping always imply faster asymptotic decay?
\item Confirm this result by plotting the solution at $t = 1.5$ for $d = 2.5, 3.2, 20$
      on the same plot with vertical axis $[-.002,.002]$, using the same values for
      $u_0$ and $v_0$ as in part~(d) of Problem~2.  (You should simply adapt the code
      you developed for Problem~2 to produce these plots.)
   
\end{enumerate}

%%%%%%%%%%%%%%%%%%%%%%%%%%%%%%%%%%%%%%%%%%%%%%%%%%%%%%%%%%%%%%%%%%%%%%%%%%%%%%%%
\ifthenelse{\boolean{showsols}}{\input optdamp_sol}{}


