
Consider the following wave equation:
\begin{align*}
   u_{tt} -u_{xx} &= 0, \qquad -\infty < x < \infty, \:\: t \geq 0 .
\end{align*}  
\begin{enumerate}    
\item Let $u_t(x,0) = \gamma(x) = 0$.  What is the domain of dependence of the solution at $x = t = 1$?
\item Let the wave equation have initial conditions
\begin{align*}
          u(x, 0) &= {\displaystyle{\psi (x) = \left\{\begin{array}{ll}
                      	x +	1 &  \:\: \mbox{if} \:\: -1 \leq x < 0 ,\\
       									1 - x & \:\: \mbox{if} \:\: 0 \leq x < 1 ,\\
       									0    & \:\:  \mbox{otherwise}
                               \end{array}\right.}}\\
     u_t (x, 0) &= \gamma (x) = 0 ,  \qquad -\infty < x < \infty,      
\end{align*}  
Find the exact solution to this problem in terms of $\psi(x)$, and write the solution at $t =1/2$ and $t = 10$.
\item What qualitative differences would you expect between the solution to the
heat equation on $-\infty < x < \infty$ with initial condition given above and the solution to
the wave equation? 
%\item What consequences do these differences have on solution methods (Fourier series, finite elements) for these equations?

\end{enumerate}  

%%%%%%%%%%%%%%%%%%%%%%%%%%%%%%%%%%%%%%%%%%%%%%%%%%%%%%%%%%%%%%%%%%%%%%%%%%%%%%%%%%%%%%%%%%%%%%%%%%%%%%%%%%%%%%%%%%%%%%%%%%%%%5

 \ifthenelse{\boolean{showsols}}{
\begin{solution}
\begin{enumerate}

\item Recall that the general solution of the wave equation on an unbounded domain
      takes the form
     \[ u(x,t) = {\textstyle{1\over2}}\big(\psi(x-t) + \psi(x+t)\big) 
                   + {\textstyle{1\over2}}\int_{x-t}^{x+t} \gamma(s)\, ds.\]
      From this it follows that 
     \[ u(1,1) = {\textstyle{1\over2}}\big(\psi(0) + \psi(0)\big) 
                   + {\textstyle{1\over2}}\int_{0}^{0} \gamma(s)\, ds = \psi(0)\]
      Since the solution at $x=1$ and $t=1$ depends on $\psi$ at $x=0$, we note that the 
      domain of dependence is a single point which is $x=0$.
\item D'Alembert�s general solution to the wave equation on $-\infty < x < \infty$ gives (with $c = 1$):
\[
u(x, t) = \frac{1}{2} \left[ \psi(x-t) + \psi(x+t) \right] + \frac{1}{2} \int_{x-t}^{x+t} \gamma(s) \: ds = \frac{1}{2} \left[ \psi(x-t) + \psi(x+t)\right] \]

Therefore,
\[
 u(x, 1/2) = \frac{1}{2} \left[ \psi(x-1/2) + \psi(x+1/2)\right] = {\displaystyle{ \left\{\begin{array}{ll}
                      	\frac{1}{2}(x+\frac{3}{2} ) &  \:\: \mbox{if} \:\: \frac{-3}{2}  \leq x < \frac{-1}{2}  ,\\\\
       									\frac{1}{2}  &  \:\: \mbox{if} \:\: \frac{-1}{2}  \leq x < \frac{1}{2}  ,\\\\
       									-\frac{1}{2}(x-\frac{3}{2} ) &  \:\: \mbox{if} \:\: \frac{1}{2}  \leq x < \frac{3}{2}  ,\\\\
       									0    & \:\:  \mbox{otherwise}
                               \end{array}\right.}}
\]
and
\[
 u(x, 10) = \frac{1}{2} \left[ \psi(x-10) + \psi(x+10) \right] = {\displaystyle{ \left\{\begin{array}{ll}
                      	\frac{1}{2}(x+11 ) &  \:\: \mbox{if} \:\: -11  \leq x < -10  ,\\\\
                      	-\frac{1}{2}(x+10 ) &  \:\: \mbox{if} \:\: -10  \leq x < -9  ,\\\\
       									\frac{1}{2} (x -9 )  &  \:\: \mbox{if} \:\:  9 \leq x < 10  ,\\ \\
       								  -\frac{1}{2}(x+11 ) &  \:\: \mbox{if} \:\:   10 \leq x < 11  ,\\\\
       									0    & \:\:  \mbox{otherwise}
                               \end{array}\right.}}
\]


\item As we saw in our solutions to the heat equation on a finite interval, heat
equation solutions are smooth; the initial profile immediately becomes smooth $(C^{\infty})$,
widens, and flattens out in time (in this case approaches $0$) as the heat spreads throughout
the bar. The fact that the temperature profile becomes $(C^{\infty})$ immediately (i.e., for
any $t > 0$) implies that the heat has an infinite speed of propagation (which is, of course,
unphysical in that it violates the theory of relativity). On the other hand, the profile
itself will not propagate left or right, but will remain centered at $x = 0$, with its maximum
value also at $x = 0$. In contrast, as seen in part (a), the wave equation does not
yield a smooth solution, but rather preserves any sharp corners or discontinuities in the
initial data, and the solution is composed of left- and right-going waves that propagate
at a finite speed $c = 1$.
The consequences these differences have on solutions methods relate primarily to the
difference in smoothness of the solutions. Since the wave equation preserves the singularities
(discontinuities, corners, etc.) in the initial data, the effectiveness of the solution
method is limited by its ability to accurately represent such singularities. As we discussed
in class, however, Fourier series converge more slowly for singular functions than for smooth functions and piecewise-linear basis functions provide a relatively poor approximation
to singular functions. Therefore, either more terms in the Fourier series
representation or more basis functions in the finite element discretization are required
to achieve a desired accuracy in solving the wave equation than in solving the heat
equation.
\end{enumerate}
 \end{solution}
}{}