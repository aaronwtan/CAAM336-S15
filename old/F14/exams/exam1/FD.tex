% finite diff problem
% 4 nodes
%Solve a 2x2 system (given how to solve)
%Inhomogeneous Dirichlet
%Homogeneous/Inhomogeneous Neumann 
%Robin boundary conditions

Consider the steady state heat equation
\[
- \pd{u(x)}{x}{2} = f(x), \quad 0 < x < 1.
\]
We wish to discretize the above system using finite differences at four points 
\[
x_0 = 0, \quad x_1 = \frac{1}{3},\quad x_2 = \frac{2}{3}, \quad x_3 = 1.
%0 = x_0, \quad x_1, \quad x_2, \quad x_3 = 1.
\]
such that $h = 1/3$.  We wish to solve for the values of the solution $u(x_i) = u_i$ for $0<x_i <1$.
\begin{enumerate}
\item Write out the finite difference equations for the above differential equation at all points $0< x_i <1$, using the second order finite difference approximation
\[
u''(x_i) \approx \frac{u_{i+1}-2u_i + u_{i-1}}{h^2}
\]
Consider the inhomogeneous boundary conditions 
\[
u(0) = 1, \qquad u(1) = 2.
\]
Write down the resulting finite difference equations for these boundary conditions.  
\item Consider the inhomogeneous boundary conditions
\[
u(0) = 1, \qquad -u'(1) = 2.
\]
Use the backwards difference approximation
\[
u'(1) \approx \frac{u_{3}-u_2}{h}
\]
to impose the above boundary conditions.  Write down the resulting finite difference equations for these boundary conditions.  
\item Consider now the case of a Robin or Cauchy boundary condition on the left
\[
u(0) = 1, \qquad -u'(1) + \alpha u(1) = 2+2\alpha
\]
Use the backward difference for $u'(1)$ in (c) to modify the finite difference equations to accommodate a Robin boundary condition.  \emph{Hint: eliminate $u_3$ from the system using the above boundary condition.  You can check that if $\alpha = 0$, you recover the answer to (b), and if $\alpha \rightarrow \infty$, you recover the answer to (a).}
\item Solve, for the above Robin boundary condition, the $2\times 2$ finite difference system for the values of $u_1$ and $u_2$, and recover the value of $u_3$ from the given boundary condition.  You may use the fact that the inverse of a $2\times 2$ matrix is
\[
\bmatrix{a & b \cr c & d}^{-1} = \frac{1}{ad-bc}\bmatrix{d & -b \cr -c & a}.
\]
Give solution values $u_1, u_2$ in terms of $\alpha$ and $f(x_i)$, and explain how to compute $u_3$ given $u_2$.  

\emph{If you are unable to do part (c), you may solve the system from part (b) for half credit.}

\end{enumerate}


\ifthenelse{\boolean{showsols}}{
\begin{solution}
\begin{enumerate}
\item Since we wish to satisfy the finite difference version of the equation 
\[
-u''(x_i) \approx \frac{u_{i+1}-2u_i + u_{i-1}}{h^2} = f(x_i)
\]
at both points $x_1, x_2$, we have
\begin{eqnarray*}
-u''(x_1) &\approx -\frac{1}{h^2}(u_{2}-2u_1 + u_{0}) = f(x_1)\\
-u''(x_2) &\approx -\frac{1}{h^2}(u_{3}-2u_2 + u_{1}) = f(x_2).
\end{eqnarray*}
Since $u_0 = u(x_0) = u(0) = 1$, we may replace it with the value of the boundary condition, and likewise with $u_3 = u(x_3) = u(1) = 2$.  Then, the above equations become
\begin{eqnarray*}
\frac{1}{h^2}(-u_{2} + 2u_1 ) = f(x_1) + \frac{1}{h^2}\\
\frac{1}{h^2}(-u_{3} + 2u_2 ) = f(x_2) + \frac{2}{h^2}.
\end{eqnarray*}
\item Since the first equation 
\[
-u''(x_1) \approx \frac{1}{h^2}(-u_{2} + 2u_1 ) = f(x_1) + \frac{1}{h^2}
\]
does not involve $u_3$, it does not change.  The second equation
\[
-u''(x_2) \approx -\frac{1}{h^2}(u_{3}-2u_2 + u_{1}) = f(x_2)
\]
will be modified to accommodate the new boundary condition.  Approximating $u'(1) \approx \frac{u_{3}-u_2}{h}$, we have
\[
-u'(1) \approx -\frac{u_{3}-u_2}{h} = 2.
\]
Then, we may rewrite the second finite difference equation as
\[
\frac{1}{h}\left(-\frac{u_3 - u_2}{h} -\frac{1}{h}(-u_2 + u_1)\right) = \frac{1}{h}\left(2 + \frac{1}{h}(u_2 - u_1)\right)= f(x_2)
\]
Rearranging unknowns to the left hand side gives
\[
\frac{1}{h^2}\left(u_2 - u_1\right)= f(x_2) - \frac{2}{h}.
\]
\item Once again, the first equation does not change, since the left boundary condition is the same as before.  The Cauchy/Robin boundary condition can be approximated as
\[
-u'(1)  + \alpha u(1) \approx -\frac{u_{3}-u_2}{h}  + \alpha u_3 = 2 + 2\alpha. 
\]
Rearranging, we can use the above to define $u_3$ in terms of $u_2$ 
\[
u_3 = \frac{2 + 2\alpha - u_2/h}{\alpha - 1/h}.
\]
We can substitute this directly into the second equation to get
\[
-\frac{1}{h^2} \left( \frac{2 + 2\alpha - u_2/h}{\alpha - 1/h}-2u_2 + u_{1}\right) = f(x_2)
\]
Rearranging gives us the second finite difference equation
\[
\frac{1}{h^2}\left(\left(2 - \frac{1}{h(\alpha - 1/h)}\right)u_2 - u_1\right) = f(x_2) + \frac{2+2\alpha}{h^2(\alpha-1/h)}.
\]
A quick check shows that if $\alpha = 0$, we recover part (b), while if $\alpha \rightarrow \infty$, we recover part (a).  
\item We can put the above equations into matrix form:
\[
\frac{1}{h^2}\bmatrix{a & b \cr c & d}\bmatrix{u_1\cr u_2} = \bmatrix{f \cr g}
\]
where 
\[
a = 2, \quad b = -1
\]
correspond to the first equation, and 
\[
c = 2 + \frac{1}{h(\alpha - 1/h)}, \quad d = -1
\]
and 
\[
f = f(x_1) + \frac{1}{h^2}, \quad g = f(x_2) + \frac{2+2\alpha}{h^2(\alpha - 1/h)}.
\]




\end{enumerate}
\end{solution}

}{}