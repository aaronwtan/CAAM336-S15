\begin{solution}
\begin{enumerate}
\item We want to show $\ip{\cdot,\cdot}$  is an inner product on $\CV$.
\begin{itemize}
\item The mapping is symmetric, as 
\[
\ip{f,g} = \int_0^1 f(x) g(x) \, dx + \int_0^1 f'(x) g'(x) \, dx = \int_0^1 g(x) f(x) \, dx + \int_0^1 g'(x) f'(x) \, dx = \ip{g,f}
\]
for all $f,g\in C^1[0,1]$.

\item The mapping is also linear in the first argument since
\begin{eqnarray*}
\ip{\alpha f + \beta g, h} &=& \int_0^1 (\alpha f(x) + \beta g(x)) h(x) \, d x +  \int_0^1 (\alpha f'(x) + \beta g'(x)) h'(x) \, d x
\\
&=&  \alpha \int_0^1 f(x) h(x) \, dx +  \beta \int_0^1 g (x) h (x) \, dx + \alpha \int_0^1 f'(x) h'(x) \, dx +  \beta \int_0^1 g'(x) h'(x) \, dx\\
&=&  \alpha \left(\int_0^1 f(x) h(x) \, dx + \int_0^1 f'(x) h'(x) \, dx \right) +  \beta \left(\int_0^1 g (x) h (x) \, dx  +  \beta \int_0^1 g'(x) h'(x) \, dx \right)
\\
&=& \alpha \ip{f,h} + \beta \ip{g,h}
\end{eqnarray*}
for all $f,g,h\in C^1[0,1]$ and all $\alpha,\beta\in\R$. Together with symmetry it implies inner product linear also second argument.

\item The mapping is also positive definite  for all $x\in[0,1]$ 
\[
(f,f) = \int_0^1 (f(x))^2\, d x + \int_0^1 (f'(x))^2\, d x
\]
is the integral of a nonnegative function, and hence is also nonnegative. If $(f,f) = 0$ then $(f(x))^2 = 0$ for all $x\in[0,1]$ and, this means that $f(x)=0$ for all $x\in[0,1]$, i.e., $f=0$. Hence, the mapping is positive definite.
\end{itemize}
\item Suppose that $v, w \in V$. Then
\begin{eqnarray*}
L_u(\alpha v + \beta w) &=& \alpha v + \beta w-2\ip{\alpha v + \beta w,u} u\\
											  &=&\alpha v + \beta w - 2\ip{\alpha v,u} u - 2\ip{\beta w,u} u\\
											  &=& \alpha v - 2 \alpha \ip{ v,u} u + \beta w - 2\beta \ip{ w,u} u\\
											  &=& \alpha \underbrace{\left(v - 2 \ip{ v,u} u \right)}_{ L_u(v)} + \beta \underbrace{\left( w - 2\ip{ w,u}
											  u\right)}_{ L_u(w)}\\
											  &=& \alpha  L_u(v) + \beta  L_u(w)
											  \end{eqnarray*}
for all $\alpha, \beta \in \R $. This shows that $L_u$	is linear. 

Now, we want to show that $\ip{L_u(v),L_u(w)}= \ip{v,w}$ for all $v,w \in V$.

\begin{eqnarray*}
\ip{ L_u(v), L_u(w)} &=& \ip{ v-2\ip{ v, u } u, w-2\ip{ w, u } u } \\
&=& \ip{ v, w } - 2\ip{ v, \ip{ w, u } u } - 2 \ip{ \ip{ v,u } u, w } +4 \ip{\ip{ v, u } u, \ip{ w, u } u } \\
&=& \ip{ v, w } - 2 \ip{ w, u } \ip{ v, u } - 2 \ip{ v,u } \ip{ u, w } + 4 \ip{ v, u } \ip{ w, u } \underbrace{\ip{ u, u }}_{=1} \\
&=& \ip{ v, w } - 2 \ip{ w, u } \ip{ v, u } - 2 \ip{ v,u } \ip{ u, w } + 4 \ip{ v, u } \ip{ w, u } \\
&=& \ip{ v, w } - 2 \ip{ w, u } \ip{ v, u } - 2 \ip{ v,u } \ip{ u, w } + 4 \ip{ v, u } \ip{ u, w } \\
&=& \ip{ v, w } - 2 \ip{ v,u } \ip{ w, u } + 2 \ip{ v, u } \ip{ u, w } \\
&=& \ip{ v, w } - 2 \ip{ v,u } \ip{ u, w } + 2 \ip{ v, u } \ip{ u, w } \\
&=& \ip{ v, w }.
\end{eqnarray*}

Shows  $L_u$ is \emph{unitary}.

\item Using the given orthogonal basis
\[
\{\phi_1, \phi_2, \phi_3\}= \left\{\frac{1}{\sqrt{2}}, \quad \sqrt{\frac{3}{2}}x,  \quad \sqrt{\frac{5}{8}}(3x^2 -1)\right\}.
\] 
\begin{itemize}
\item[i.] We will show that $\| \phi_i \|^2= \ip{\phi_i, \phi_i}=1$ for $i=1,2,3$ with respect to defined norm. 
\begin{eqnarray*}
\ip{\phi_1, \phi_1} &=& \int_{-1}^1 \frac{1}{2} \, dx = \left[\frac{x}{2} \right]_{-1}^{1}= 1\\
\\
\ip{\phi_2, \phi_2} &=& \int_{-1}^1 \frac{3}{2} x^2 \, dx = \left[\frac{1}{2} x^3 \right]_{-1}^{1}= 1 \\
\\
\ip{\phi_3, \phi_3} &=& \int_{-1}^1 \frac{5}{8} (3x^2-1)^2 \, dx = \left[\frac{5}{8} (\frac{9 x^5}{5}-2 x^3 + x) \right]_{-1}^{1}= 1 
\end{eqnarray*} 

\item[ii.] Since $\phi_1$ and $\phi_2$ are orthogonal with respect to the inner product $\ip{\cdot,\cdot}$, i.e., $\ip{\phi_1, \phi_2} = 0$, the best approximation to $g(x) = \sin(\pi x)$ from ${\rm span}\{\phi_1,\phi_2\}$ with respect to the norm $\norm{\cdot}$ is

\begin{eqnarray*}
p(x) = \frac{\ip{g,\phi_1}}{\ip{\phi_1, \phi_1}} \phi_1(x) + \frac{\ip{g,\phi_2}}{\ip{\phi_2, \phi_2}} \phi_2(x) = \ip{g,\phi_1} \phi_1(x) + \ip{g,\phi_2} \phi_2(x)
\end{eqnarray*}

We can compute 
\[
\ip{g,\phi_1} = \int_{-1}^1 \frac{1}{\sqrt{2}} \sin(\pi x) \, dx =   \frac{1}{\sqrt{2}} \left[ \frac{-\cos (\pi x)}{\pi}\right]_{-1}^{1}=0
\]
and
 \[
\ip{g,\phi_2} = \int_{-1}^1 \sqrt{\frac{3}{2}} x \sin(\pi x) \, dx =  \sqrt{\frac{3}{2}} \int_{-1}^1 x \sin(\pi x) \, dx=  \sqrt{\frac{3}{2}}  \frac{2}{\pi} =  \frac{\sqrt{6}}{\pi}.
\]

Then

\begin{eqnarray*}
p(x) =  \ip{g,\phi_1} \phi_1(x) + \ip{g,\phi_2} \phi_2(x) =  \frac{\sqrt{6}}{\pi} \sqrt{\frac{3}{2}} x =\frac{3}{\pi} x
\end{eqnarray*}

\item[iii.] Similar to part (ii) the best approximaton to $g(x) = \sin(\pi x)$ from ${\rm span}\{\phi_1,\phi_2, \phi_3\}$ with respect to the norm $\norm{\cdot}$ is

\begin{eqnarray*}
q(x) &=& \frac{\ip{g,\phi_1}}{\ip{\phi_1, \phi_1}} \phi_1(x) + \frac{\ip{g,\phi_2}}{\ip{\phi_2, \phi_2}} \phi_2(x) + \frac{\ip{g,\phi_3}}{\ip{\phi_3, \phi_3}} \phi_3(x)\\
\\
&=& \ip{g,\phi_1} \phi_1(x) + \ip{g,\phi_2} \phi_2(x) + \ip{g,\phi_3} \phi_3(x)
\end{eqnarray*}
Now let us compute 
 \[
\ip{g,\phi_3} = \int_{-1}^1 \sqrt{\frac{5}{8}}(3x^2 -1) \sin(\pi x) \, dx =  \sqrt{\frac{5}{8}}\int_{-1}^1(3x^2 \sin(\pi x) - \sin(\pi x))  \, dx = 0
\]

From part (ii) we have $ \ip{g,\phi_1} =0$ and $ \displaystyle{\ip{g,\phi_2} = \frac{\sqrt{6}}{\pi}}$. Then the best approximation 

\begin{eqnarray*}
q(x) = \ip{g,\phi_1} \phi_1(x) + \ip{g,\phi_2} \phi_2(x) + \ip{g,\phi_3} \phi_3(x) = \frac{3}{\pi} x
\end{eqnarray*}

\end{itemize}
\end{enumerate}
\end{solution}