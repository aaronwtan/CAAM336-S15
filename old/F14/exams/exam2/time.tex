% time dependent problem

Let $L$ be a linear operator on a vector space $V$ and suppose that $Lv=\lambda v$ with $v \neq 0$ so that $\lambda$
is an eigenvalue of $L$ with corresponding eigenvector $v$.
\begin{enumerate}
\item Show that the operator $\tilde{L}$ defined by
\[
\tilde{L}u = Lu + \mu u \qquad u \in W, \qquad \mu \:\: \mbox{a scalar}
\]
is a linear operator on $W$ and that $v$ is an eigenvector for $\tilde{L}$ corresponding to the eigenvalue $\tilde{\lambda}= \lambda + \mu$. 
\item Now consider the operator $\tilde{L}$
\[
\tilde{L} : C^2_M [0; 1] \rightarrow C[0; 1]\]
where
\[
 C^2_M [0; 1] = \{u \in C^2[0; 1] : u(0) = u'(1) = 0\}.
 \]
 with
\[
\tilde{L}u(x) = \frac{-d^2u(x)}{dx^2}+ u(x)
\]
Show that $\tilde{L}$ is symmetric and positive definite.

\item Calculate the eigenvalues $\tilde{\lambda}_n$ and corresponding eigenfunctions $u_n$ of $\tilde{L}.$

In class we saw that the eigenvalues and eigenfunctions for ^Lu = .. d2
dx2 u were
^
k =

2k .. 1
2
2
2
^  k(x) = sin
q
^
kx

= sin

2k .. 1
2

x

Consider, for $\alpha, \kappa > 0$, the partial differential equation
\[
\pd{u(x,t)}{t}{} + \alpha u(x,t) - \kappa \pd{u(x,t)}{x}{2} = 0.
\]
with initial/boundary conditions
\begin{eqnarray*}
u(x,0) &= \psi(x)\\
u(0,t) &= 0\\
u(1,t) &= 0.
\end{eqnarray*}
Define finite difference points $x_0, x_1, \ldots, x_N, x_{N+1}$; we wish to simulate the solution $u(x,t)$ at points $x_i$ and times $t_j$.  Further, define vector of solution values $u^j$ and finite difference matrix $A$ such that
\[
u^j = \bmatrix{u(x_1,t_j)\cr u(x_2,t_j) \cr \vdots\cr u(x_N,t_j)}, \qquad A = \frac{\kappa}{h^2} \bmatrix{2 &-1 & &&\cr -1 &2 &-1 &&\cr &\ddots &\ddots & \ddots&\cr &&&&-1\cr & & &-1&2}. 
\]
You may use that $Au^j$ gives back the vector of finite difference approximations of $-\kappa\pd{u(x_i,t_j)}{x}{2}$ at all points $x_i$.
\begin{enumerate}
\item Using a forward difference approximation of the derivative $\pd{u}{t}{}$, derive that the resulting finite difference equations for the above equation satisfy the update formula
\[
u^{j+1} = \left[ (1 - dt \alpha) I - dt A\right]u^j
\]
and that this formula can be rewritten as
\[
u^{j+1} = \left[ (1 - dt \alpha) I - dt A\right]^{j+1}u^0.
\]
\item Derive an expression for $\lambda_i$, the eigenvalues of $$ (1 - dt \alpha) I - dt A$$ in terms of $\mu_i$, the eigenvalues of $A$.  Assuming that 
\[
0 < \mu_i \leq 4\frac{\kappa}{h^2}, \quad i = 1,\ldots, N,
\] 
derive a bound on $dt$ 
\[
dt \leq (???)
\]
in terms of $\kappa, h^2$, and $\alpha$ which guarantees $|\lambda_i| \leq 1$.  
%show that 
%\item We showed in class that 
%\[
%| \lambda_i | = | 1- dt(\alpha + \mu_i) | \leq 1
%\]
%is a requirement for $u^j$ to not blow up as $j\rightarrow \infty$.  Show that 
%\[
%dt \leq \frac{2}{\alpha + \mu_i} 
%\]
%satisfies this condition.  
%Assume that the eigenvalues $\mu_i$ of $A$ satisfy 
%$$0 < \mu_i \leq 4\frac{\kappa}{h^2}, \quad i = 1,\ldots, N.$$  
%Verify that the following bound
%\[
%dt \leq \frac{2}{(\alpha + 4\frac{\kappa}{h^2})}
%\]
%gives $|\lambda_i | \leq 1$.  
%guarantees $| \lambda_i | = | 1- dt(\alpha + \mu_i) | \leq 1$.  
\end{enumerate}


\ifthenelse{\boolean{showsols}}{
\begin{solution}
\begin{enumerate}
\item If we choose to satisfy our differential equation at points $x_i$ and times $t_j$, it becomes
\[
\pd{u_i^j}{t}{} + \alpha u_i^j - \kappa \pd{u_i^j}{x}{2} = 0
\]
where $u_i^j = u(x_i,t_j)$.  If we replace the second derivative term by  
\[
- \kappa \pd{u_i^j}{x}{2} = \frac{-\kappa}{h^2}\left(u_{i+1}^j - 2u_i^j + u_{i-1}^j\right)
\]
we get 
\[
\pd{u_i^j}{t}{} + \alpha u_i^j + \frac{\kappa}{h^2}\left(-u_{i+1}^j + 2u_i^j - u_{i-1}^j\right) = 0.
\]
At this point, we may express this in matrix form, to get
\[
\pd{u^j}{t}{} + \alpha u^j + Au^j=  \pd{u^j}{t}{} + (\alpha I + A)u^j = 0
\]
Applying a forward difference approximation in time gives
\[
\pd{u^j}{t}{} \approx \frac{u^{j+1}-u^j}{dt}
\]
which, substituted in, gives
\[
\frac{u^{j+1}-u^j}{dt}+ (\alpha I + A)u^j = 0.
\]
Now, since we wish to use the previous time solution $u^j$ to determine $u^{j+1}$, we multiply the equation on both sides by $dt$ and then rearrange the equation to read
\[
u^{j+1} = u^j - dt(\alpha I + A)u^j = (I-dt(\alpha I + A))u^j.
\]
Since
\[
u^{1} = (I-dt(\alpha I + A))u^0, \qquad u^{2} = (I-dt(\alpha I + A))u^1 = (I-dt(\alpha I + A))^2 u^0, \ldots
\]
we can see that 
\[
u^{j+1} = (I-dt(\alpha I + A))^{j+1} u^0.
\]
\item Assuming that $A$ has eigenvalues $\mu_i$, $\alpha I + A$ has eigenvalues $\alpha + \mu_i$.  Similarly, if a matrix is scaled by a constant, the eigenvalues are also scaled by that same constant.  

Thus, if we add $\alpha I + A$, the eigenvalues are $\alpha + \mu_i$.  Scaling this by $dt$ gives that the matrix $(I-dt(\alpha I + A))$ has eigenvalues 
\[
\lambda_i = 1 - dt(\alpha + \mu_i).  
\]
Since 
\[
0< \mu_i < 4\kappa/h^2,
\]
 $dt(\alpha + \mu_i)$ is always positive, and $|\lambda_i| = |1 - dt(\alpha + \mu_i)|$ is largest when $dt(\alpha + \mu_i)$ is most positive.  Since $\mu_i$ has a maximum value, $|\lambda_i|$  is smaller than 
\[
|\lambda_i| \leq |1 - dt(\alpha + 4\kappa/h^2)|.
\]
If 
\[
dt(\alpha + 4\kappa/h^2) > 2,
\]
then $|\lambda_i | > 1$, and we violate the stability condition.  Thus, we require
\[
dt(\alpha + 4\kappa/h^2) \leq 2
\]
which, after rearrangement, gives
\[
dt \leq \frac{2}{(\alpha + 4\kappa/h^2)}
\]
\end{enumerate}
\end{solution}
}{}