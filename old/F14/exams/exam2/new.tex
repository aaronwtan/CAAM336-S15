Consider the operator $L$ defined on the space of $n$th degree polynomials $P_n[0,1]$ where the notation $P_n$ represent all polynomials
of degree less than or equal to $n$. 
\[
Lu =  - \frac{d}{dx} \left( x(1-x) \frac{d}{dx} u(x) \right) ,
\]
\begin{enumerate}
\item Show that $L$ takes polynomials of $P_n$ to polynomials
of the same degree. In other words check that 
\[
L: P_n [0,1] \rightarrow P_n [0,1]
\]
\emph{Hint: You can define $n$th degree polynomial $p_n(x)= \sum_{i=0}^n c_i x^i$}

\item Case $n=2$ given $p_2(x)= a x^2 +b x + c$ denote $Q= LP$, such that $Q(x)= u x^2 + v x + w$.
Express $ u, v, w$ in terms of $a, b, c$. Write the relationship in matrix notation 
\[
M \left[\begin{array}{c} a \\ b \\ c \end{array}\right] =\left[\begin{array}{c} u \\ v \\ w \end{array}\right].
\]
Express the matrix $M$ explicitly.

\item The eigenvalues of the matrix M is given $\lambda_0 = 0$, $\lambda_1 = 2$ and $\lambda_2 = 6$. Coming
back to polynomial notation, we will denote $p_0$, $p_1$ and $p_3$ eigenvectors associated
respectively to the eigenvalues $\lambda_0 = 0$, $\lambda_1 = 2$ and $\lambda_2 = 6$. Normalized by taking the leading coefficient equal to $1$.\\

Show that $p_0(x)=1$, $p_1(x)= x+\alpha$ and $p_2(x)= x^2+ \beta x +\gamma$ with appropriate values for $\alpha, \beta$ and $\gamma$.

\item Now consider the general case. Assuming that $L$ has, for some $n$, a polynomial of degree $n$ (with
the leading coefficient normalized to $1$	) as an eigenvector, show that the corresponding
eigenvalue should be
\[
\lambda_n = n (1+n)
\]
\emph{Hint: in order to determine $\lambda_n$ you just have to look at the terms of degree $n$ in the
formula $LP=\lambda P$}\\
Coming back to the case $n=2$ show that the above formula holds for the eigenvalues which have been found in that case.

\end{enumerate}