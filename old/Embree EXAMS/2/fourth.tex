Consider the fourth-order linear operator $L: C^4_H[0,1] \to C[0,1]$,
\[ L u = u_{xxxx} + \eps @ u_{xx},\]
where $C^4_H[0,1]$ consists of all $C^4[0,1]$ functions $v$ with 
\emph{hinged} boundary conditions
\[ v(0)= v''(0) = v(1) = v''(1) = 0.\]
Here $\eps$ is a constant and we use the inner product
\[ (f,g) = \int_0^1 f(x) g(x)\, \dop x.\]

\begin{enumerate}
\item The eigenfunctions of $L$ are 
$\psi_k(x) = \sqrt{2} \sin(k\pi x)$, for $k=1,2,3\ldots.$\\
  Find the corresponding eigenvalues $\lambda_k$ of $L$.
\end{enumerate}

\vspace*{1em}
We wish to solve the partial differential equation
\[ u_{t}(x,t) = -\big(u_{xxxx}(x,t)+\eps@u_{xx}(x,t)\big),\]
for $x\in[0,1]$ and $t\ge 0$, with hinged boundary conditions
\[ u(0,t)= u_{xx}(0,t) = u(1,t) = u_{xx}(1,t) = 0\]
and initial condition
\[ u(x,0) = u_0(x).\]

\vspace*{1em}
 
\begin{enumerate}
\item[(b)] To compute the spectral method solution, 
we pose this equation as
\[ u_t = - L u \] 
and seek the solution in the form
\[ u(x,t) = \sum_{j=1}^\infty a_j(t) \psi_j(x).\]
Identify the ordinary differential equation that $a_j$ must satisfy,
solve that differential equation, and give a simplified formula for
$a_j(t)$ that depends only on $u_0$, $\lambda_j$, $\psi_j$, and $t$.

\vspace*{1em}
\item[(c)] Under what conditions on $\eps$ can we ensure that the
solution $u(x,t)$ decays to zero regardless of the initial data $u_0$?
\end{enumerate}

Now consider the following partial differential equation,
which is related to a model of vibrations of a stiff piano wire:
\[ u_{tt}(x,t) = -\big(u_{xxxx}(x,t)+\eps@u_{xx}(x,t)\big),\]
again for $x\in[0,1]$ and $t\ge 0$, with hinged boundary conditions
\[ u(0,t)= u_{xx}(0,t) = u(1,t) = u_{xx}(1,t) = 0\]
and initial conditions
\[ u(x,0) = u_0(x), \qquad 
   u_t(x,0) = v_0(x).\]

\vspace*{1em}
For the rest of this question, assume that $\eps$ is chosen such
that all eigenvalues are nonnegative: $\lambda_k\ge 0$ for all $k$.

\begin{enumerate}
\item[(d)] Now to compute the spectral method solution for this equation,
we write
\[ u_{tt} = - L u \] 
and again seek the solution in the form
\[ u(x,t) = \sum_{j=1}^\infty a_j(t) \psi_j(x).\]
For this second-order problem in time,
identify the ordinary differential equation that $a_j$ must satisfy,
solve that differential equation, and give a simplified formula for
$a_j(t)$ that depends only on $u_0$, $v_0$, $\lambda_j$, $\psi_j$, and $t$.

\emph{Be sure to handle the special case where there is a zero eigenvalue.}

\vspace*{1em}
\item[(e)]
How do solutions to this equation behave as $t\to\infty$ if there is no zero eigenvalue?\\
Does this behavior change if there is a zero eigenvalue?

\vspace*{1em}
\item[(f)]
  Suppose we solve the problem $u_t(x,t) = -u_{xxxx}(x,t)$
  using the finite element method, and when $N$ elements are
  used, the eigenvalues $\lambda$ of $\BM^{-1}\BK$ satisfy
  $0<\lambda < 16 N^4$, with largest eigenvalue
   $\lambda_N \approx 16 N^4$.

  We wish to solve the equation $\Bc'(t) = -\BM^{-1}\BK \Bc(t)$
  using the \emph{forward Euler method} with time step $\Delta t$.
  Let $t_k = (\Delta t)k$.  

  How should we pick the time step $\Delta t$ so that the approximate
  solutions $\Bc_k\approx \Bc(t_k)$ decay to zero as $k\to\infty$?

  If you double $N$, how should $\Delta t$ be adjusted to maintain
  the decay to zero as $k\to\infty$?

  How does your analysis change if we instead use the \emph{backward Euler method}?

\end{enumerate}
