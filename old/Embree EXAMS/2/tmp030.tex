\documentclass[11pt,twoside,reqno,a4paper]{amsart}
\usepackage{amsfonts,amsmath,amscd,amsthm}
\usepackage{graphics}
\usepackage{a4}
\usepackage{showlabels}
\usepackage{citesort}

\begin{document}



1.
Let us consider the problem of finding $\boldsymbol{x}\left(t\right)$ such that
\[
\boldsymbol{x}'\left(t\right)=\boldsymbol{A}\boldsymbol{x}\left(t\right)
\]
and
\[
\boldsymbol{x}\left(0\right)=\boldsymbol{x}_0
\]
where the vector $\boldsymbol{x}_0\in\mathbb{R}^N$ and the symmetric matrix $\boldsymbol{A}\in\mathbb{R}^{N\times N}$. Let $\Delta t>0$ and let $t_k=k\Delta t$ for non-negative integers $k$. We can obtain approximations $\boldsymbol{x}_k$ to $\boldsymbol{x}\left(t_k\right)$ by using the formula
\[
\boldsymbol{x}_{k+1}=\boldsymbol{x}_k+\Delta t\left(1-\theta\right)\boldsymbol{A}\boldsymbol{x}_k+\Delta t\theta\boldsymbol{A}\boldsymbol{x}_{k+1}
\]
for some choice of $\theta\in[0,1]$.

(a) What linear system of equations has to be solved to compute $\boldsymbol{x}_{k+1}$? Your answer should not feature the inverse of any matrices.

(b) In order to analyze the behavior of this method it will be convenient to first write
\[
\boldsymbol{x}_k=\boldsymbol{B}^k\boldsymbol{x}_0
\]
where $\boldsymbol{B}\in\mathbb{R}^{N\times N}$. What matrix is $\boldsymbol{B}$?

(c) Let the matrix $\boldsymbol{A}$ have eigenvalues $\lambda_1,\ldots,\lambda_N$ with corresponding orthonormal eigenvectors $\boldsymbol{v}_1,\ldots,\boldsymbol{v}_N$. The matrix $\boldsymbol{B}$ has the same eigenvectors as $\boldsymbol{A}$. What are the eigenvalues of $\boldsymbol{B}$?

(d) We can write
\[
\boldsymbol{x}_0=\sum_{j=1}^N\left(\boldsymbol{x}_0,\boldsymbol{v}_j\right)\boldsymbol{v}_j
\]
where $\left(\boldsymbol{x}_0,\boldsymbol{v}_j\right)=\boldsymbol{v}_j^T\boldsymbol{x}_0$. We can also write
\[
\boldsymbol{x}_k=\sum_{j=1}^N c_j\boldsymbol{v}_j
\]
where the coefficients $c_j\in\mathbb{R}$. Use the fact that $\boldsymbol{x}_k=\boldsymbol{B}^k\boldsymbol{x}_0$ to obtain a formula for the coefficients $c_j$.

(e) If all of the eigenvalues of $\boldsymbol{A}$ are positive and $\theta>0$, why should the choice of $\Delta t=\frac{1}{\theta\lambda_j}$ be avoided for all $j=1,\ldots,N$?

(f) If all of the eigenvalues of $\boldsymbol{A}$ are negative and $0\le\theta<\frac{1}{2}$, what restriction should be placed on $\Delta t$ so that $\left\|\boldsymbol{x}_k\right\|\rightarrow0$ as $k\rightarrow\infty$ for all $\boldsymbol{x}_0$?

(g) If all of the eigenvalues of $\boldsymbol{A}$ are negative and $\frac{1}{2}<\theta<1$, how will $\left\|\boldsymbol{x}_k\right\|$ behave as $k\rightarrow\infty$?

(h) For certain choices of $\theta$ the method in this problem is actually a method that we have looked at previously in this course. Which methods that we have looked at previously do the choices of $\theta=0$, $\theta=\frac{1}{2}$ and $\theta=1$ correspond to?



\bigskip



2. Let $N$ be a positive integer and let $h=\frac{1}{N+1}$ and $x_k=kh$. Let
\[
V_N=\mbox{span}\left\{\phi_0,\phi_1,...,\phi_{N+1}\right\}
\]
where the $\phi_j$ are the usual piecewise linear hat functions which are such that
\[
\phi_j\left(x_i\right)=\left\{\begin{array}{c}1\mbox{ if }i=j \\ 0\mbox{ if } i\ne j\end{array}\right.
\]
for $i,j=0,1,\ldots,N+1$.

Consider the problem of finding $u\left(x,t\right)$ such that
\[
u_t\left(x,t\right)=u_{xx}\left(x,t\right),\mbox{ }0<x<1,\mbox{ }t>0
\]
with Neumann boundary conditions
\[
u_x\left(0,t\right)=u_x\left(1,t\right)=0,\mbox{ }t>0
\]
and initial condition
\[
u\left(x,0\right)=u_0\left(x\right)\mbox{ }0\le x\le1.
\]

We can obtain a finite element approximation
\[
u_N\left(x,t\right)=\sum_{j=0}^{N+1}\alpha_j\left(t\right)\phi_j\left(x\right)
\]
to $u$ by finding $\boldsymbol{\alpha}\left(t\right)$ such that
\[
\boldsymbol{M}\boldsymbol{\alpha}'\left(t\right)=-\boldsymbol{K}\boldsymbol{\alpha}\left(t\right)
\]
for matrices $\boldsymbol{M}$ and $\boldsymbol{K}$.

(a) Write down the general forms of $\boldsymbol{M}$ and $\boldsymbol{K}$.

(b) What should we take $\boldsymbol{\alpha}\left(0\right)$ to be in order for
\[
u_N\left(x_i,0\right)=u_0\left(x_i\right)
\]
for $i=0,1,\ldots,N+1$.

(c) When $N=1$, the matrix $\boldsymbol{M}^{-1}\boldsymbol{K}$ is such that
\[
\boldsymbol{M}^{-1}\boldsymbol{K}\left[\begin{array}{c}1 \\ 1 \\ 1\end{array}\right]=\left[\begin{array}{c}0 \\ 0 \\ 0\end{array}\right],\mbox{ }\boldsymbol{M}^{-1}\boldsymbol{K}\left[\begin{array}{c}-1 \\ 0 \\ 1\end{array}\right]=\left[\begin{array}{c}-12 \\ 0 \\ 12\end{array}\right],\mbox{ and }\boldsymbol{M}^{-1}\boldsymbol{K}\left[\begin{array}{c}1 \\ -1 \\ 1\end{array}\right]=\left[\begin{array}{c}48 \\ -48 \\ 48\end{array}\right]\mbox{ }.
\]
We can write
\[
-\boldsymbol{M}^{-1}\boldsymbol{K}\boldsymbol{W}=\boldsymbol{W}\boldsymbol{D}
\]
where the matrix
\[
\boldsymbol{W}=\left[\begin{array}{ccc}1 & -1 & 1 \\ -1 & 0 & 1 \\ 1 & 1 & 1\end{array}\right].
\]
Write down the diagonal matrix $\boldsymbol{D}$.

(d) The eigenvectors of $-\boldsymbol{M}^{-1}\boldsymbol{K}$ are not orthonormal and so $\boldsymbol{W}^{-1}\ne\boldsymbol{W}^T$. However, since the matrix $\boldsymbol{W}$ is symmetric there exist matrices $\boldsymbol{V}$ and $\boldsymbol{\Lambda}$ which are such that $\boldsymbol{W}=\boldsymbol{V}\boldsymbol{\Lambda}\boldsymbol{V}^{-1}$ and $\boldsymbol{V}^{-1}=\boldsymbol{V}^T$. The matrix $\boldsymbol{W}$ has eigenvalues $-\sqrt{2},\sqrt{2},2$ with corresponding eigenvectors $\left[\begin{array}{c} -1 \\ -\sqrt{2} \\ 1\end{array}\right],\left[\begin{array}{c} -1 \\ \sqrt{2} \\ 1\end{array}\right],\left[\begin{array}{c} 1 \\ 0 \\ 1\end{array}\right]$. Construct the matrices $\boldsymbol{V}$ and $\boldsymbol{\Lambda}$.

(e) Compute $\boldsymbol{W}^{-1}$. You may use the fact that $\left(\boldsymbol{V}\boldsymbol{\Lambda}\boldsymbol{V}^{-1}\right)^{-1}=\boldsymbol{V}\boldsymbol{\Lambda}^{-1}\boldsymbol{V}^{-1}$.

(f) By writing $-\boldsymbol{M}^{-1}\boldsymbol{K}$ in an appropriate form, compute the exact solution to
\[
\boldsymbol{M}\boldsymbol{\alpha}'\left(t\right)=-\boldsymbol{K}\boldsymbol{\alpha}\left(t\right)
\]
with initial condition
\[
\boldsymbol{\alpha}\left(0\right)=\left[\begin{array}{c}\alpha_0\left(0\right) \\ \alpha_1\left(0\right) \\ \alpha_2\left(0\right)\end{array}\right].
\]

(g) How does $\boldsymbol{\alpha}\left(t\right)$ behave as $t\rightarrow\infty$ when $\alpha_0\left(0\right)=\alpha_1\left(0\right)=\alpha_2\left(0\right)=1$?

(h) When $u_0\left(x\right)=1$ the true solution is $u\left(x,t\right)=1$. What is the error $u-u_N$ when $u_0\left(x\right)=1$?



\bigskip



3. Let us consider the problem of finding $u\left(x,y\right)$ such that
\[
-u_{xx}\left(x,y\right)-u_{yy}\left(x,y\right)=f\left(x,y\right),\mbox{ }-1<x<1,\mbox{ }-1<y<1,
\]
with Neumann boundary conditions
\[
u_x\left(-1,y\right)=u_x\left(1,y\right)=0,\mbox{ }-1\le y\le1
\]
and 
\[
u_y\left(x,-1\right)=0,\mbox{ }u_y\left(x,1\right)=g\left(x\right)\mbox{ }-1\le x\le1.
\]

(a) Show that if $u\in C^2\left[-1,1\right]^2$ is the solution to the above problem then
\[
a\left(u,v\right)=l\left(v\right)\mbox{ for all }v\in C^2\left[-1,1\right]^2
\]
where
\[
a\left(w,v\right)=\int_{-1}^1\int_{-1}^1w_x\left(x,y\right)v_x\left(x,y\right)+w_y\left(x,y\right)v_y\left(x,y\right)\mbox{ }dxdy
\]
and
\[
l\left(v\right)=\int_{-1}^1\int_{-1}^1f\left(x,y\right)v\left(x,y\right)dxdy+\int_{-1}^1g\left(x\right)v\left(x,1\right)\mbox{ }dx.
\]

(b) Let $V_N=\mbox{span}\left\{\phi_1,\ldots,\phi_N\right\}$. Suppose that we want to find
\[
u_N\left(x,y\right)=\sum_{j=1}^N\alpha_j\phi_j\left(x,y\right)
\]
such that
\[
a\left(u_N,v\right)=l\left(v\right)\mbox{ for all }v\in V_N.
\]
We can obtain the coefficients $\alpha_j$ by solving the linear system of equations
\[
\boldsymbol{K}\boldsymbol{\alpha}=\boldsymbol{b}.
\]
where the matrix $\boldsymbol{K}\in\mathbb{R}^{N\times N}$ has entries
\[
K_{ij}=a\left(\phi_j,\phi_i\right)
\]
for $i,j=1,\ldots,N$. What are the entries of the vector $\boldsymbol{b}$?

(c) Compute $\displaystyle{\int_{-1}^1\left(1-s\right)^2ds}$,  $\displaystyle{\int_{-1}^1\left(1+s\right)^2ds}$ and $\displaystyle{\int_{-1}^1\left(1-s\right)\left(1+s\right)ds}$.

(d) Construct the matrix $\boldsymbol{K}$ when $N=4$ and
\[
\phi_1=\frac{1}{4}\left(1-x\right)\left(1-y\right)
\]
\[
\phi_2=\frac{1}{4}\left(1+x\right)\left(1-y\right)
\]
\[
\phi_3=\frac{1}{4}\left(1-x\right)\left(1+y\right)
\]
\[
\phi_4=\frac{1}{4}\left(1+x\right)\left(1+y\right)
\]
for $-1\le x\le1$ and $-1\le y\le1$.

(e) Simplify
\[
\sum_{j=1}^4\phi_j\left(x,y\right).
\]

(f) The matrix $\boldsymbol{K}$ is such that
\[
\boldsymbol{K}\boldsymbol{v}_0=\boldsymbol{0}
\]
where
\[
\boldsymbol{v}_0=\left[\begin{array}{c}1 \\ 1 \\ 1 \\ 1\end{array}\right].
\]
Consequently,
\[
\boldsymbol{K}\boldsymbol{\alpha}=\boldsymbol{b}
\]
will have no solutions if $\boldsymbol{v}_0^T\boldsymbol{b}\ne0$. Show that if $l\left(1\right)\ne0$ then $\boldsymbol{v}_0^T\boldsymbol{b}\ne0$.



\end{document}