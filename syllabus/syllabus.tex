\documentclass[10pt]{article}
\usepackage{fullpage}
\usepackage{hyperref}
\parindent0em\parskip1em
\textheight10in
\pagestyle{empty}
\begin{document}
\vspace*{-6em}
\begin{center}
\large \textsf{\textbf{CAAM 336}}\\[0.5em]
       \textsf{\textbf{DIFFERENTIAL EQUATIONS IN SCIENCE AND ENGINEERING}\\[0.25em]
Spring 2015 $\cdot$ Rice University}
\end{center}


\hspace*{-3em}
\begin{tabular}{rl}
\hline & \\[-.5em]
Lectures:			& Section 1: MWF 1:00-1:50pm, Keck 100 \\
				& Section 2: MWF 1:00-1:50pm, Duncan Hall 1064 \\[.75em]
%
Web Site: 			& http://www.caam.rice.edu/\raisebox{1pt}{$\sim$}caam336 \\[.75em]
%
Instructors:  		& Section 1: Jesse Chan (Jesse.Chan@rice.edu), Duncan Hall 3023, (713) 348-6113 \\
				& Office hours: Monday, Wednesday 2-3PM, or by appointment.\\[.5em]
		  		& Section 2: Travis Thompson (tthompson@rice.edu), Duncan Hall 2038, (713) 348-3336 \\
				& Office hours: Tuesday, Thursday 2-3PM, or by appointment\\[.75em]%Tuesday 2-3PM, Friday 10-11AM.\\[.75em]
%
Teaching Assistants: 	& Caleb Magruder (caleb.magruder@rice.edu), Duncan Hall 2105\\%Xiaodi Deng (xd3@rice.edu), Duncan Hall 2106. \\
				& Office hours: Friday TBD\\ [.5em]				
				& Charles Puelz (cpuelz@rice.edu), Duncan Hall 2108\\%Xiaodi Deng (xd3@rice.edu), Duncan Hall 2106. \\
				& Office hours: Tuesday 3-4PM\\[.75em] %Thursday 12:30-2:30PM\\
Recitations: 		& (Caleb) Monday 7-9PM, Duncan Hall 1070\\
				& (Charles) Tuesday 7-9PM, Duncan Hall TBD
\end{tabular}

\hspace*{-3em}
\begin{tabular}{rl}
\hline & \\[-.5em]
%
%Absence Policy: 	& Students are strongly encouraged to contribute to our class community by attending and\\
%				& participating in lectures. If you can not attend a lecture, please find another student in the class\\
%				& who can communicate what material was covered. Failure to attend a scheduled exam may \\
%				& result in you being given a score of zero for that exam.\\[1.25em]
%
Text: 			& The content of this course is based on material from:\\
				& Mark S. Gockenbach, \emph{Partial Differential Equations: Analytical and Numerical}\\
				& \emph{Methods}, SIAM, Philadelphia, 2011.\\[1.25em]
%
Grade Policy: 		& Homework will account for 40\% of the final grade and exams will account for the \\
				& remaining 60\%.  The last exam will be a scheduled exam which will take place during\\
				& the final exam period. Failure to attend a scheduled exam may result in a score of zero\\
				& for that exam.\\[1.25em]
%
Matlab: 			& Most homework questions will require some MATLAB programming. Any MATLAB codes\\
				& that you write for the homework must be printed out and attached to the corresponding\\
				& homework in order to receive credit. For assistance with parts of the homework requiring \\
				& MATLAB, please consult the class website for additional materials.\\[1.25em] 
%
Honor Code:		& It is an honor code violation to turn in code or solutions which have, in all or in part, been\\
				& copied from another student. It is also an honor code violation to consult solutions to the\\
				& homework or exams from previous sections of this class.\\[1.25em]
%			
Homework: 		& Problem sets will be assigned roughly once a week, due at 5pm on the specified date.\\
				& Mathematically rigorous solutions are expected; strive for clarity and elegance.\\
				& You may collaborate on the problems, but your write-up must be your own independent work.\\
				& Transcribed solutions and copied MATLAB code are both unacceptable.  \\
				& \emph{You may not consult solutions from previous sections of this class.}\\[1.25em] 

Late Policy:		& You may turn in two problem sets at 5pm on the next class day without penalty. Subsequent\\
     				& late assignments will be penalized 20\% each.  Homework will not be accepted more than one\\
				& class period late.  (You \textit{may not} use two `lates' on one assignment.) In exceptional \\
				& circumstances, contact the instructor as soon as possible:  we adhere to Student Health's \\
				& \href{http://health.rice.edu/Content.aspx?id=98}{No Note} policy.\\[1.25em] 
				
Re-grade Policy: 	& If your work has been graded incorrectly, you may submit a re-grade request. Clearly explain the\\
				& perceived error on a separate sheet of paper, staple it to the front of your graded paper, and give\\
				& it to the instructor. \\[1em]

\hline
\end{tabular}
\vspace{-1.5em}
\begin{center}
\em Any student with a disability requiring accommodation in this course is encouraged\\
to contact the instructor during the first week of class, and also to contact\\
 Disability Support Services in the Ley Student Center.
\end{center}
\newpage

\vspace{-1em}
\hspace*{-3em}
\begin{tabular}{rp{42em}}
\hline & \\[-.5em]
Prerequisites:           & CAAM 210 (Introduction to Engineering Computation) and MATH 212 (Multivariable Calculus), \\
                         & or permission of the instructor.   Less formally: you should be able to write elementary MATLAB\\
                         & programs, and be comfortable with vector calculus and basic vector and matrix operations.\\[1.25em]

Course Objectives:       & CAAM 336 students learn to identify partial differential equations
                                and solve canonical cases using both analytical and computational techniques.\\[1em]
%
Course Outcomes:         & Upon completing this course, students should be able to:\\
                         & \ \ \ 1) explain the concept and computation of a best approximation from a subspace;\\
                         & \ \ \ 2) identify linear operators and compute their spectra in basic cases;\\
                         & \ \ \ 3) apply these concepts to obtain exact series solutions to differential equations;\\
                         & \ \ \ 4) develop finite element approximations to the exact solutions;\\
                         & \ \ \ 5) assess the stability requirement for time-dependent approximations;\\
                         & \ \ \ 6) create MATLAB programs that compute and visualize series and finite element solutions. \\[1em] 
%
\hline & \\[-.5em]
\end{tabular}

\vspace*{-1em}
\begin{center} 
   CAAM 336 $\cdot$ Manifesto and Outline
\end{center}

\vspace*{-1em}

Differential equations form the heart of applied mathematics:
they capture an amazing variety of phenomena in fields ranging from 
physical science and engineering to biology, from financial derivatives
to traffic flow.
As useful as these equations are, in practice they can be quite
difficult to solve:  as you read these words, many of the world's 
fastest computers are busily calculating solutions to such equations.
Though the roots of this field reach back to the 18th century,
many important, fascinating problems remain for you to tackle.
% When you engage in science and engineering research, 
% you will almost certainly encounter a differential equation.
Nearly all disciplines in science and engineering rely 
fundamentally on differential equations.


This course focuses on the three most famous partial differential equations:  
the Poisson (diffusion) equation, the heat equation, and the wave equation.
There are two main approaches to solving these equations: 
one that constructs the exact solution as an infinite series 
(the `spectral method') 
and another that uses numerical methods to quickly 
build approximations to the solution.
For historical reasons these two approaches are usually taught
in distinct courses---an unfortunate fissure that masks
beautiful and deep connections.
This course will emphasize these connections.  
By mastering these ideas you will not only learn elegant techniques
for solving classical problems; you will also develop tools 
to solve the problems you will discover for yourselves as 
professional scientists and engineers.

\end{document}
