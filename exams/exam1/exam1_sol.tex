\documentclass[11pt]{article}


%\usepackage{amsmath}
%\let\bmatrix\relax
\usepackage{336notes,ifthen,color}

\lfoot[\small \fancyplain{}{}]
      {\small \fancyplain{}{}}
\rfoot[\small \fancyplain{}{}]
      {\small \fancyplain{}{}}
\cfoot{\small \thepage}

\newboolean{showsols}
\newenvironment{solution}{\vspace*{1em} \hrule\vspace*{0.5em} {\sf Solution.} }
                         {\vspace*{0.75em}\hrule}

\def\pd#1#2#3{\frac{\partial^{#3} #1}{\partial #2^{#3}}}

\begin{document}
\setboolean{showsols}{true}
\thispagestyle{empty}
\vspace*{0em}
   \begin{center}
      \textsf{\textbf{CAAM 336\ $\cdot$\ DIFFERENTIAL EQUATIONS IN SCI AND ENG}}\\[0.5em]
       \textsf{Examination~1 with Solutions}
   \end{center}
   
\vspace*{3em}
%Instructions:
%\begin{enumerate}
%\item Time limit: \textbf{3 uninterrupted hours}.
%\item There are four questions worth a total of 100 points.\\
%      Please do not look at the questions until you begin the exam.
%\item You are allowed one cheat sheet to refer to during the exam. \\
%You \emph{may not} use any outside resources, such as books, notes, problem sets, friends,\\
%      calculators, or MATLAB. 
%\item Please answer the questions thoroughly and justify all your answers.\\
%      \emph{Show all your work to maximize partial credit.}
%
%\item Print your name on the line below:
%
%\vspace*{1em}\rule{6in}{0.5pt}
%
%\item Indicate that this is your own individual effort in compliance with 
%      the instructions above and the honor system by writing out in full 
%      and signing the traditional pledge on the lines below.
%
%\vspace*{1em}\rule{6in}{0.5pt}
%
%\vspace*{1em}\rule{6in}{0.5pt}
%
%\vspace*{1em}\rule{6in}{0.5pt}
%
%\item Staple this page to the front of your exam.
%
%\end{enumerate}

\newpage
\setcounter{page}{1}
\begin{enumerate}
%%%%%%%%%%%%%%%%%%%%%%%%%%%%%%%%%%%%%%%%%%%%%%%%%%%%%%%%%%%%%%%%%%%%%%%%%%%%%%%%

\item {[25 points: (a) = 5, (b),(c) = 10]}
\vspace*{.5em}
\input alternate_steadyheat
\newpage
\input alternate_steadyheat_sol

\newpage
\item {[25 points: (a) = 5, (b),(c) = 10]}
\vspace*{.5em}
\input fd

\newpage
\item {[25 points: (a), (b) = 10,(c)=5]}
\vspace*{.5em}
\input vecSpace

\newpage
\item {[25 points: (a) = 10, (b), (c), (d) = 5]}
\vspace*{.5em}
\input best_approximations
\newpage
\input best_approximations_sol


%old material below this line
%\newpage
%\item {[15 points: (a) = 9, (b)= 6]}
%\vspace*{.5em}
%\input time

%\newpage
%\item {[30 points: (a) = 8, (b) = 7, (c) = 15 each, 5 each part]}
%\vspace*{.5em}
%\input ipl
%\newpage
%\item {[25 points: (a) = 8, (b), (c) = 6, (d) = 5]}
%\vspace*{.5em}
%\input order

%\newpage
%\item {[25 points: (a) = 8, (b), (c) = 6, (d) = 5]}
%\vspace*{.5em}
%\input periodicheat
%
%=======


%\newpage
%\item {[Bonus: (a) = 4, (b) = 1]}
%\vspace*{.5em}
%\input bonus

\end{enumerate}
\end{document}
