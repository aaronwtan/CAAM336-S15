
In class we considered inner products for vectors in $\R^n$ and
functions in $C[0,1]$, and used these inner products to generate
best approximations.  
This problem will introduce you to an inner product of \emph{matrices} 
in $\R^{n\times n}$.
(Related math arises in many modern applications;
e.g., at the roots of how Netflix recommends movies.) 

      Let $V = \R^{2\times 2}$, the set of all $2\times 2$ matrices,
      and define the inner product for matrices as
      \[ (\BA,\BB) = \sum_{j=1}^2 \sum_{k=1}^2 a_{jk} b_{jk},\]
       where
      \[ \BA = \bmatrix{a_{11} & a_{12} \cr a_{21} & a_{22}}, \quad
         \BB = \bmatrix{b_{11} & b_{12} \cr b_{21} & b_{22}}.\]
%      With this inner product we associate the norm $\|\BA\| = \sqrt{(\BA,\BA)}$.

\begin{enumerate}

\item Verify that $(\cdot, \cdot)$ is an inner product on $V=\R^{2\times 2}$.
\vspace*{1em}
\item Given that $V$ (the set of all possible $2\times 2$ matrices) is a vector space, show that the set $V_s$ of all symmetric $2\times 2$ matrices
\[
V_s = \left\{ \BA =   \bmatrix{a_{11} & a_{12} \cr a_{21} & a_{22}} \in \R^{2\times 2}, \qquad a_{12} = a_{21}\right\}
\]
is a subspace of $V$.
\vspace*{1em}
\item Consider the operators $S: \R^{2\times 2} \rightarrow \R^{2\times 2}$ and $K: \R^{2\times 2} \rightarrow \R^{2\times 2}$ 
\[
S(\BA) = \BA + \BA^T, \qquad K(\BA) = \BA - \BA^T,
\]
where $\BA^T$ is the matrix transpose 
\[
\BA^T = \bmatrix{a_{11} & a_{21} \cr a_{12} & a_{22}}.
\]
Show that both of these operators are linear operators.  
\item Show that, for any $\BA\in \R^{2\times 2}$, 
\[
S(K(\BA)) = K(S(\BA)) = {\bf 0}
\]
where $\bf 0$ is the $2\times 2$ zero matrix.  Use this fact to explain how the range and null space of $S$ and range and null space of $K$ are related.
%\textcolor{red}{Todo: finish, add something about the null space of each operator and how that implies that $S(K(\BA)) = K(S(\BA)) = 0$.  }
%\vspace*{1em}
%\item Consider the subspace of $V$ consisting of all symmetric matrices:
%      \[  V_3 = {\rm span}\bigg\{ 
%       \bmatrix{1 & 0 \cr 0 & 0}, 
%       \bmatrix{0 & 0 \cr 0 & 1}, 
%       \bmatrix{0 & 1 \cr 1 & 0}\bigg\}.\]
%      
%      Compute the best approximation $\BM_3$ to the matrix
%      \[ \BM = \bmatrix{ 1 & 3 \cr 7 & 9}\]
%      from the subspace $V_3$.
%
%\item Now let $\BM\in\R^{2\times2}$ be any $2\times 2$ matrix, and let
%           $\BM_3$ be its best approximation from $V_3$.
%           Explain why the error $\BM-\BM_3$ must \emph{always} have zero 
%           diagonal entries.
%
%\item Now consider the subspace
%      \[  \widehat{V}_3 = {\rm span}\Bigg\{ 
%       \bmatrix{{1\over\sqrt{2}}  & 0 \cr 0 & {1\over \sqrt{2}}}, 
%       \bmatrix{{1\over\sqrt{10}} & {2\over\sqrt{10}} \cr {2\over\sqrt{10}} & {1\over \sqrt{10}}}, 
%       \bmatrix{{1\over\sqrt{2}} & 0 \cr 0 & -{1\over\sqrt{2}}}\Bigg\}.\]
%       What is the best approximation $\widehat{\BM}_3$ from $\widehat{V}_3$ to 
%      \[ \BM = \bmatrix{ 1 & 3 \cr 7 & 9}\,?\]
%
%\item Write down a basis for the set of symmetric matrices in $\R^{3\times 3}$ (i.e., $3\times 3$ matrices).\\
%       In general, what is the dimension of the set of symmetric matrices
%       in $\R^{n\times n}$?
\end{enumerate}

\ifthenelse{\boolean{showsols}}{
\begin{solution}
\begin{enumerate}
\item We need to show 2 properties:
\begin{itemize}
\item $(\alpha\BA + \beta\BB, {\bf C}) = \alpha(\BA,{\bf C}) + \beta(\BB,{\bf C})$: we may show this using the formula
\begin{eqnarray*}
(\alpha\BA + \beta\BB, {\bf C}) &=& \sum_{j=1}^2\sum_{k=1}^2 (\alpha a_{jk} + \beta b_{jk})c_{jk} \\
&=& \sum_{j=1}^2\sum_{k=1}^2\left( \alpha a_{jk}c_{jk} + \beta b_{jk}c_{jk}\right)\\
&=& \alpha \sum_{j=1}^2\sum_{k=1}^2 a_{jk}c_{jk} + \beta \sum_{j=1}^2\sum_{k=1}^2 b_{jk}c_{jk}\\
&=& \alpha(\BA,{\bf C}) + \beta(\BB,{\bf C}).
\end{eqnarray*}
\item $(\BA,\BA) \geq 0$, and $(\BA,\BA) = 0$ only if $\BA$ is the zero matrix.  This may be shown by again using the formula
\[
(\BA,\BA) =  \sum_{j=1}^2\sum_{k=1}^2 a_{jk}^2 \geq 0,
\]
because $a_{jk}^2$ is positive.  If $(\BA,\BA) = 0$, then $a_{jk}^2$ must equal 0, and then the matrix $\BA = 0$.  
\end{itemize}
\item Let $\BA,\BB \in V_s$.  We just need to show that $\alpha \BA + \beta\BB \in V_s$, or that $\alpha \BA + \beta\BB$ is a symmetric matrix.  Note that
\[
\alpha \BA + \beta\BB = \left[
\begin{array}{cc}
\alpha a_{11} + \beta b_{11}& \alpha a_{12} + \beta b_{12}\\
\alpha a_{21} + \beta b_{21}& \alpha a_{22} + \beta b_{22}
\end{array}
\right].
\]
Since $\BA,\BB \in V_s$, $a_{12} = a_{21}$ and $b_{12} = b_{21}$, so 
\[
\alpha \BA + \beta\BB = \left[
\begin{array}{cc}
\alpha a_{11} + \beta b_{11}& \alpha a_{12} + \beta b_{12}\\
\alpha a_{21} + \beta b_{21}& \alpha a_{22} + \beta b_{22}
\end{array}
\right]
= \left[
\begin{array}{cc}
\alpha a_{11} + \beta b_{11}& \alpha a_{12} + \beta b_{12}\\
\alpha a_{12} + \beta b_{12}& \alpha a_{22} + \beta b_{22}
\end{array}
\right].
\]
The result is a symmetric matrix, so $\alpha \BA + \beta\BB\in V_s$, and $V_s$ is a subspace of $\R^{2\times 2}$.  
\item For $S$ and $K$ to be linear operators, we need
\[
S(\alpha \BA + \beta\BB) = \alpha S(\BA) + \beta S(\BB), \quad 
K(\alpha \BA + \beta\BB) = \alpha K(\BA) + \beta K(\BB).
\]
You may prove this element-wise or using matrices: for $S$, we have
\[
S(\alpha \BA + \beta\BB) = \alpha \BA + \alpha \BA^T + \beta \BB + \beta \BB^T = \alpha (\BA + \BA^T) + \beta (\BB + \BB^T )=\alpha S(\BA) + \beta S(\BB).
\]
and for $K$ we have
\[
K(\alpha \BA + \beta\BB) = \alpha \BA - \alpha \BA^T + \beta \BB - \beta \BB^T = \alpha (\BA - \BA^T) + \beta (\BB - \BB^T )=\alpha K(\BA) + \beta K(\BB).
\]
\item We can show the result by noting that $(\BA^T)^T = \BA$:
\[
S(K(\BA)) = S(\BA - \BA^T) = (\BA - \BA^T) + (\BA - \BA^T)^T = \BA - \BA^T + \BA^T - \BA = 0
\]
\[
K(S(\BA)) = K(\BA + \BA^T) = (\BA + \BA^T) - (\BA + \BA^T)^T = \BA + \BA^T - \BA^T - \BA = 0
\]
What this implies is that the range of $S$ (and possible matrix $\BA$ such that $\BA = S(\BB)$ for some $\BB$) is contained in the null space of $K$, and the range of $K$ is contained in the null space of $S$ (the two are actually equal, which also receives full credit as an answer).  
\end{enumerate}
\end{solution}
}{}
