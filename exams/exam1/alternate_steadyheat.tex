
The derivation of the heat equation given at the beginning of the course arrived at the condition 
\[
\rho c \frac{\partial u}{\partial t} (x,t) + \frac{\partial q}{\partial x} = f(x)
\]
where $q(x,t)$ is the heat flux per unit volume, $\rho$ is the density of the material and $c$ is the specific heat.  The relationship of \textit{Fourier's Law}, which says that 
\[
q(x,t) = \kappa \frac{\partial u}{\partial x}(x,t),
\]
was used to derive the heat equation.  An alien substance has been found for which Fourier's law has been shown not to hold.  However, scientists have determined that a slightly different relation 
\[
q(x,t) = \kappa\frac{\partial u}{\partial x}(x,t) + \beta u(x,t)
\]
does describe the heat flux.

\begin{enumerate}
\item Write the modified, time-dependent heat equation for the newly discovered substance.  [Hint: What is the solution of the ordinary differential equation $\frac{\partial}{\partial x}y(x) = y(x)$ ? ]

\item Suppose a bar of length $L$ of the alien substance is manufactured with $\kappa = 3$, $\beta = 2$.  Determine a general formula for the steady-state termperature distribution of this bar, which should include two free constants.  (Assume no additional source term f(x) is present).

\item Find formulas for these free constants in the case that the left end, at $x=0$, has a fixed \textit{heat flux} equal to $0$ and the right end, at $x=L$, is held constant at $\delta$ degrees.

%\item  Consider the sub-vector space $W \subset V$ of linear functions given by the span of $\left\{1,x\right\}$.  Find the best approximation $m(x) \in W$ to the function $g(x) \in V$ defined by $g(x) = 2x^3 - x^2 - x + \frac{1}{2}$ by solving the $2\times2$ Gram system $Gx = b$.   You may use the fact that the inverse of a $2\times 2$ matrix is
%\[
%\bmatrix{a & b \cr c & d}^{-1} = \frac{1}{ad-bc}\bmatrix{d & -b \cr -c & a}.
%\]
%
%\item Let $p(x) = a + bx$ be an arbitrary element of the subvector space of linear functions $W$.  Show explicitly the orthogonality condition of the best approximation $m(x)$ holds by computing $\ip{g(x)-m(x),p(x)}$ and showing that it must be zero.
%
%\item Since the Gram matrix is symmetric the spectral theorem guarantees that two orthogonal eigenvectors $v_1$ and $v_2$ can be found.  One such eigenvector of the Gram matrix for this problem is $v_2 = \left[\begin{array}{c} \frac{1}{3}\left(2+\sqrt{13}\right) \\ 1 \end{array}\right] $ corresponding to the linear function $v_2(x) = x + \frac{1}{3}\left(2+\sqrt{13}\right)$.  Apply the Gram-Schmidt procedure to the set of linearly independent functions $\left\{ u_1 = x, u_2 = v_2 \right\}$ to construct an alternate, orthogonal basis for $W$.  Your basis does not need to be orthonormal. 

\end{enumerate}