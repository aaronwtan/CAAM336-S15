Consider the \textit{Euler Bernoulli beam equation},
\[
(k(x)u''(x))'' = f(x), \qquad 0 < x < 1,
\]
Here $k(x)$ is a positive-valued function that describes the material properties of the beam.

\begin{enumerate}
\item Verify that, for $k(x) = 1$, the functions 
\[
\sin(\mu_j x),\quad \cos(\mu_jx), \quad \sinh(\mu_jx),\quad \cosh(\mu_jx)
\]
are all eigenfunctions the operator $Lu = u''''$ with eigenvalues $\lambda_j$, where $\mu_j = \lambda_j^{1/4}$ and where hyperbolic sine/cosine are defined as
\[
\sinh(x) = \frac{e^x-e^{-x}}{2}, \quad \cosh(x) = \frac{e^x+e^{-x}}{2}.
\]
\emph{Hint: the derivatives of $\sinh$ and $\cosh$ may be written in terms of $\sinh$ and $\cosh$.}

\item Consider boundary conditions describing a beam that is \textit{clamped} at both ends:
\[
u(0) = u(1) = 0; \qquad  u'(0) = u'(1) = 0.
\]
With these boundary conditions, the eigenvalues and eigenvectors of this operator are diffcult to compute, even if $k(x)=1$.  As a result, we will consider the finite element approximation of this problem.  

Derive the weak form of the beam equation with the above boundary conditions, i.e., derive the weak problem
\[
a(u,v) = (f, v); \qquad \mbox{for all} \:\:v \in V = C_D^4[0, 1],
\]
where
\[
C_D^4[0, 1] = \{u \in C^4[0, 1] : u(0) = u(1) = u'(0) = u'(1) = 0\}.
\]

Specify the bilinear form $a(u, v)$, and show that it is an inner product on $C_D^4[0, 1]$
\\
\emph{Note: for the problem $-(k u')' = f$, we do not explicitly impose Neumann boundary conditions, they follow 'naturally'. For the beam equation, we must impose all four boundary conditions on the space of test functions, $V =C_D^4[0, 1].$}

\item Suppose that $V_n = span\{\phi_1, \cdots, \phi_n\}$ is an n-dimensional subspace of $C_D^4[0, 1].$
(Do not assume a particular form for the functions $\phi_1, \cdots, \phi_n$ at this point.)

Show how the Galerkin problem
\[
a(u_n, v) = (f, v), \qquad \mbox{for all}\:\: v \in V_n
\]
leads to the linear system $Ku = f$ . Be sure to specify the entries of $K, u$, and $f$ .

\item Now suppose we take for $\phi_1, \cdots, \phi_n$  the standard piecewise linear 'hat' functions used, for example, in Problem 2. Are these functions suitable for this problem? If so, describe the location of the
nonzero entries of the matrix $K$. If not, roughly describe a better choice for the functions $\phi_1, \cdots, \phi_n$ 
and the explain which entries of $K$ are nonzero for that choice.
\end{enumerate}

%%%%%%%%%%%%%%%%%%%%%%%%%%%%%%%%%%%%%%%%%%%%%%%%%%%%%%%%%%%%%%%%%%%%%%%%%%%%%%%%

\ifthenelse{\boolean{showsols}}{
\begin{solution}
\begin{enumerate}

\item For all $v \in V$ multiply both side of BVP with the test function $v$

\[
(k(x)u''(x))'' v  = f(x) v, \qquad \forall v \in V,
\]

Now integrate over $0$ to $1$ 

\[
\int_0^1 (k(x)u''(x))'' v(x) dx = \int_0^1 f(x) v(x) dx   \qquad \forall v \in V\\
\]
Apply integration by parts  the left hand side
\begin{eqnarray*}
\int_0^1 (k(x)u''(x))'' v(x) dx &=& \left[(k(x)u''(x))' v(x)\right]_0^1 - \int_0^1 (k(x)u''(x))' v'(x) dx\\
																&=& \left((k(1)u''(1))' v(1)- (k(0)u''(0))' v(0)\right) - \int_0^1 (k(x)u''(x))' v'(x) dx \\ 
																&=& - \int_0^1 (k(x)u''(x))' v'(x) dx  \:\:\: \mbox{by imposing $v(0)=0, v(1)=0$}\\
																&=& \left[-(k(x)u''(x)) v'(x)\right]_0^1 + \int_0^1 (k(x)u''(x)) v''(x) dx \qquad \mbox{by integration by parts }\\
																&=& \left(-(k(1)u''(1)) v'(1)+ (k(0)u''(0)) v'(0)\right) + \int_0^1 (k(x)u''(x)) v''(x) dx \\
																&=& \int_0^1 (k(x)u''(x)) v''(x) dx  \:\:\: \mbox{by imposing $v'(0)=0, v'(1)=0$} \qquad \forall v\in V
\end{eqnarray*}

Thus we get 
\[
\int_0^1 (k(x)u''(x)) v''(x) dx = \int_0^1 f(x) v(x) dx 
\]
where 
\[
a(u,v)= \int_0^1 (k(x)u''(x)) v''(x) dx \qquad \mbox{and} \qquad (f,v)= \int_0^1 f(x) v(x)\, dx \qquad \forall v\in V
\]
To show that the form $a(u,v)$ in part is an inner product, we must verify the three basic properties:
      \begin{itemize}
      \item \textbf{Symmetry} is apparent by inspection:
            \begin{eqnarray*}
               a(u,v) 
               &=& \int_0^1  k(x) u''(x) v''(x)  \,dx   \\[0.5em]
               &=& \int_0^1  k(x) v''(x) u''(x) \,dx 
               \,=\, a(v,u).
           \end{eqnarray*}

      \item \textbf{Linearity} follows from the
            linearity of differentiation and integration:
            \begin{eqnarray*}
              a(\alpha u + \beta v, w)
               &=& \int_0^1 k(x) (\alpha u + \beta v)''(x) w''(x) \,dx \\[0.5em]
               &=& \int_0^1 k(x) (\alpha u''(x) + \beta v''(x)) w''(x) \,dx \\[0.5em]
               &=& \int_0^1 \alpha k(x) u''(x)  w''(x) + \beta k(x)  v''(x) w''(x) \,dx \\[0.5em]
                &=& \alpha \int_0^1 k(x) u''(x)  w''(x) + \beta \int_0^1 k(x)  v''(x) w''(x) \,dx \\[0.5em]
               &=& \alpha @ a(u,w) + \beta @ a(v,w).
               \end{eqnarray*}
      \item \textbf{Positivity} requires that $a(u,u) \ge 0$ and $a(u,u) = 0$ only when $u=0$.
            Note that
               \begin{eqnarray*}
               a(u,u) &=& \int_0^1 k(x) u''(x) u''(x)  \,dx \\[0.5em]
                      &=& \int_0^1  k(x) \Big(u''(x)\Big)^2 \,dx.
               \end{eqnarray*}
            Since $k(x)$ is  positive for all $x\in[0,1]$, 
            integrand is non-negative, and hence $a(u,u)\ge 0$.
            To have $a(u,u)=0$,we must have  $u''(x) = 0$ for all $x\in[0,1]$, which is only possible
            if $u(x)= bx+c$ by BC $b=c=0$ then $u(x)=0$ for all $x\in[0,1]$, i.e., $u=0$.
      \end{itemize}
      
     
 \item   Let $V_n = span\{\phi_1, \cdots, \phi_n\}$ is an n-dimensional subspace of $C_D^4[0, 1].$ We would like to shot that 
\[
a(u_n, v) = (f, v), \qquad \mbox{for all}\:\: v \in V_n
\]
 leads to a linear system.\\
 

The Galerkin solution can be defined by the linear combination of basis function $u_n = \sum_{j=1}^n u_j \phi_j(x)$ for coefficients  $u_j$. Then 
\[
a(\sum_{j=1}^n u_j \phi_j(x), \phi_i(x)) = (f, \phi_i(x)), \qquad \mbox{for}\:\: i=1, \cdots , n 
\]  
By the linearity of bilinear form 

\[
\sum_{j=1}^n a(\phi_j(x), \phi_i(x))  u_j  = (f, \phi_i(x)), \qquad \mbox{for}\:\: i=1, \cdots , n 
\] 
This leads to a linear system of $N$ equations with $N$ unknown. If we define that system as $Ku=f$ then each component of $K$ and $f$ will be
\[
K_{ij}=  a(\phi_j(x), \phi_i(x))= \int_0^1 k(x) \phi_j''(x) \phi_i''(x) \, dx \qquad \mbox{for} \:\:\:i,j=1, \cdots , n 
\]
and 
\[
f_{i}= (f(x), \phi_i(x))= \int_0^1 f(x) \phi_i(x) \, dx \qquad \mbox{for} \:\:\:i=1, \cdots , n 
\]
$u$ is our solution vector with $u= [u_1, u_2, \cdots, u_n]^T$.

\item Now if we take for $\phi_1, \cdots, \phi_n$  the standard piecewise linear 'hat' functions as in Problem 2. We will get $ \phi_i''(x) = 0$ for $i=1,\cdots , n$. That leads our stiffness matrix $K=0$. Therefore, this hat function is not suitable for Euler Bernoulli beam problem.\\

By the same idea, we want to define a hat function satisfy following property

															\[
															\phi_i(x_j) = \left\{
															\begin{array}{ll}
															1 & \mbox{if }i = j;\\
															0 & \mbox{if} i\neq j;
															\end{array}\right.
															\]

Also for $$K_{ij}= \int_0^1 k(x) \phi_j''(x) \phi_i''(x) \, dx $$ we need second derivative of the hat function. For simplify, let pick a hat function second derivative is constant.\\

In that case , the  new hat function
 
																			 \[
																			\phi_i(x) = \left\{
																			\begin{array}{ll}
																			\displaystyle{\left({x-x_{i-1}\over h}\right)^2} & \mbox{if }x\in [x_{i-1}, x_i];\\
																			\displaystyle{\left({x_{i+1}-x\over h}\right)^2} & \mbox{if }x\in [x_i, x_{i+1});\\
																			0 & \mbox{otherwise};
																			\end{array}\right.
																			\]
 
With these choose we will get again tridiagonal matrix $K$ such that
\[
K_{ii}=  \int_0^1 k(x) \phi_i''(x) \phi_i''(x) \, dx=  \int_{x_{i-1}}^{x_{i+1}} k(x) \phi_i''(x) \phi_i''(x) \, dx
\] 

\[
K_{i(i+1)}=  \int_0^1 k(x) \phi_i''(x) \phi_{i+1}''(x) \, dx=  \int_{x_{i}}^{x_{i+1}} k(x) \phi_i''(x) \phi_{i+1}''(x) \, dx
\] 

By symmetry $K_{ij} = K_{ji}$. \\
Finally, for  $|i-j|>1$, 

\[
K_{ij}=  \int_0^1 k(x) \phi_i''(x) \phi_{j}''(x) \, dx = 0
\] 

\end{enumerate}
\end{solution}
}{}