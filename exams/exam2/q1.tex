In the homeworks, and in class, we used the spectral method to solve several problems of the form $Lu = f$ with boundary conditions.   The spectral method has conditions that must be satisfied before we can apply it.  In this problem we will explore the impact of boundary conditions on several theoretical facets of the spectral method.

We'll study the steady state heat equation
\[
-\pd{u(x)}{x}{2} = f(x).  
\]
We'll also use the usual inner product 
\[
(f,g) = \int_{0}^{1} f(x)g(x)\, dx
\]

\begin{enumerate}
\item Derive the spectral method eigenfunctions and eigenvalues for the steady heat equation under the boundary conditions
\[
\pd{u(0)}{x}{} = u(1) = 0.
\]
\item Consider now boundary conditions 
\[
u(1) =  \pd{u(1)}{x}{} = 0.
\]
Define the vector space $C^2_R[0,1]$ and the operator $L: C^2_R[0,1]\rightarrow C[0,1]$ such that 
\[
C^2_R[0,1] = \left\{ u\in C^2[0,1], \quad u(1)= \pd{u(1)}{x}{} = 0.\right\}, \quad Lu = -\pd{u}{x}{2}.
\]
Show that under these boundary conditions, $L$ is not symmetric, i.e.
\[
(Lu,v) \neq (u,Lv).
\]
\item Consider the more general set of boundary conditions with $a,b,c,d$ as constants
\begin{align}
       u(1) &= a u(0) + b \pd{u(0)}{x}{}             \label{eqn:bvp1}\\
      \pd{u(1)}{x}{} &= c u(0) + d\pd{u(0)}{x}{}\label{eqn:bvp2}
\end{align}
We can define a vector space of functions $V_A$ to be those functions in $C^{2}[0,1]$ satisfying the boundary conditions of the boundary value problem of equation (\ref{eqn:bvp1}) and (\ref{eqn:bvp2}). % Show that $V_A$ is a vector space by showing that $0\in V_A$ and $\alpha f + g \in V_A$ whenever $\alpha$ is a real number and $f$, $g$ are in $V_A$.
 Define $L:V_A \rightarrow C[0,1]$ as 
\[
Lu = -\pd{u}{x}{2}.
\]
We also define a matrix $A$ containing the boundary condition coefficients
\[
 A = \left[
     \begin{array}{cc}
       a & b \\
      c & d\\
     \end{array}
   \right]
\]
Whether or not $L$ is symmetric turns out to depend on the determinant of the matrix $A$, which is ${\rm det}(A) = ad - bc$.  What should ${\rm det}(A)$ be in order for $L$ to be symmetric on $V_A$? Your answer should be a number.

%\item Consider the case where $a_{11} = a_{22} = 1$ and $a_{12} = a_{22} = 0$. Is zero an eigenvalue of $L$ in $V_A$? 
%
%\item Consider the case where $a_{11} = 1$, $a_{22} = -1$, $a_{12} = 1$ and $a_{21} = -2$. What is the null space of $L$ in $V_A$?

%To apply the spectral theory the operator $L = -\frac{ \partial^{2}}{\partial_x^2}$ needs to be symmetric, i.e. $(Lf,g) = (f,Lg)$ for all $f,g \in V_A$.  Determine the conditions on $a_{11}, a_{12}, a_{21}, a_{22}$ so that this is true.  



\end{enumerate}


%%%%%%%%%%%%%%%%%%%%%%%%%%%%%%%%%%%%%%%%%%%%%%%%%%%%%%%%%%%%

\ifthenelse{\boolean{showsols}}{
\begin{solution}

\end{solution}
}{}
