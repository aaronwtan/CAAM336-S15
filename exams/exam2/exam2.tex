\documentclass[11pt]{article}


\usepackage{336notes,ifthen,color}
\let\bmatrix\relax
\usepackage{amsmath}

\lfoot[\small \fancyplain{}{}]
      {\small \fancyplain{}{}}
\rfoot[\small \fancyplain{}{}]
      {\small \fancyplain{}{}}
\cfoot{\small \thepage}

\newboolean{showsols}
\newenvironment{solution}{\vspace*{1em} \hrule\vspace*{0.5em} {\sf Solution.} }
                         {\vspace*{0.75em}\hrule}

\def\pd#1#2#3{\frac{\partial^{#3} #1}{\partial #2^{#3}}}

\begin{document}
\setboolean{showsols}{false}
\thispagestyle{empty}
\vspace*{0em}
   \begin{center}
      \textsf{\textbf{CAAM 336\ $\cdot$\ DIFFERENTIAL EQUATIONS IN SCI AND ENG}}\\[0.5em]
       \textsf{Examination~1}
   \end{center}
   
\vspace*{3em}
Instructions:
\begin{enumerate}
\item Time limit: \textbf{3 uninterrupted hours}.
\item There are four questions worth a total of 100 points.\\
      Please do not look at the questions until you begin the exam.
\item You are allowed one cheat sheet to refer to during the exam. \\
You \emph{may not} use any outside resources, such as books, notes, problem sets, friends,\\
      calculators, or MATLAB. 
\item Please answer the questions throughly (but succintly!) and justify all your answers.\\
      Show your work for partial credit.

\item Print your name on the line below:

\vspace*{1em}\rule{6in}{0.5pt}

\item Indicate that this is your own individual effort in compliance with 
      the instructions above and the honor system by writing out in full 
      and signing the traditional pledge on the lines below.

\vspace*{1em}\rule{6in}{0.5pt}

\vspace*{1em}\rule{6in}{0.5pt}

\vspace*{1em}\rule{6in}{0.5pt}

\item Staple this page to the front of your exam.

\end{enumerate}

\newpage
\setcounter{page}{1}
\begin{enumerate}
%%%%%%%%%%%%%%%%%%%%%%%%%%%%%%%%%%%%%%%%%%%%%%%%%%%%%%%%%%%%%%%%%%%%%%%%%%%%%%%%


\item {[25 points: 5 points for (a), 10 points for (b)-(c)]}
\input q1

\newpage
\item {[20 points: 10 each]}
\input femCoercivity

%\newpage
%\item {[25 points: ]}
%\input q2
\newpage
\item {[28 points: 7 points each]} 
\input periodic

\newpage
\item {[27 points: 8 points (a)-(c), 3 points (d)]}
\input beam


\end{enumerate}
\end{document}
