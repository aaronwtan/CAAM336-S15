In this problem, we consider the finite element method for the equation
\[
-\pd{}{x}{}\left(k(x)\pd{u}{x}{}\right) = f(x).
\]
This models the steady state distribution of temperature in a bar, where $k(x)$ is the diffusivity of the bar at the point $x$.  Diffusivity must be positive for the equation to be physically realistic; however, if $k(x)$ is not positive, the finite element method may run into issues as well.  

%It may be helpful to use the fact that the determinant of a $2\times 2$ matrix is
%\[
%{\rm det}\left(\left[
%\begin{array}{cc}
%a &b\\
%c &d
%\end{array}
%\right]\right) = ad-bc.
%\]

\begin{enumerate}
\item If $k(x) > 0$ for $0 < x < 1$, the weak form
\[
a(u,v) = \int_0^1 k(x)\pd{u}{x}{}\pd{v}{x}{}
\]
defines an inner product for $C^2_D[0,1]$.  Explain roughly why, if $k(x) = 0$ over some interval $[a,b] \subset [0,1]$, $a(u,u)$ may not be positive even if $u$ is nonzero (and hence not an inner product).  It may help to try drawing a function $u(x) \neq 0$ such that $a(u,u) = 0$.
\item 
%Give a formula for the entries of the $2\times 2$ finite element matrix $K_{ij} = a(\phi_j,\phi_i)$ for $k(x)$
%\[
%k(x) = x.
%\]
%where $\phi_j$ are the hat functions
%\[
%\phi_j(x) = \begin{cases}
%\frac{x-x_{j-1}}{h} & x_{j-1} < x \leq x_j\\
%\frac{x_{j+1}-x}{h} & x_{j} < x \leq x_{j+1}\\
%0 & {\rm otherwise}.
%\end{cases}
%\]
%
%\item 
Let $\phi_j$ be the hat function centered around point $x_j$ with $h = 1/(N+1)$.
\[
\phi_j(x) = \begin{cases}
\frac{x-x_{j-1}}{h} & x_{j-1} < x \leq x_j\\
\frac{x_{j+1}-x}{h} & x_{j} < x \leq x_{j+1}\\
0 & {\rm otherwise}.
\end{cases}
\]
Assume that $N = 2$, such that there are two hat functions $\phi_1(x),\phi_2(x)$ centered around points 
\[
x_1 = 1/3, \quad x_2 = 2/3.
\]

Let $k(x)$ be the function
\[
k(x) = \begin{cases}
0 & x \leq 2/3\\
1 & x > 2/3.
\end{cases}
\]
Compute the finite element stiffness matrix for $N=2$.  Verify that $A_{11} = A_{21} = 0$.
%\[
%A = \left[
%\begin{array}{cc}
%0 &0\\
%0 &3
%\end{array}
%\right].
%\]
Explain what complication arises if one attempts to solve this system/invert this matrix.  %\emph{Hint: you may wish to compute the eigenvalues of this matrix.}
\end{enumerate}