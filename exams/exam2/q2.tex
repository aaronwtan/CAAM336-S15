In the homeworks and in class we used the hat function basis $\left\{ \phi_{i}(x) \right\}_{i=1}^{N}$ defined on an evenly
spaced mesh $\left\{  x_0, x_1, \ldots x_N, x_{N+1}  \right\}$ of the interval $[0,1]$ where $x_0=0$, $x_{N+1}=1$ and $h = x_n - x_{n-1}$ is a constant value to boundary value problems like equation (\ref{eqn:bvp2}) using the finite element method. 
 
%\[
\begin{equation}\label{eqn:bvp2}
 \left\{
     \begin{array}{c}
       -\frac{\partial}{\partial_x}\left( k(x) \frac{\partial u}{\partial_x}\right) = f\\
       u(0) = 0, u(1) = 0
     \end{array}
   \right.
\end{equation}
%\]

Suppose that the number of internal mesh points, $N$, on the interval $[0,1]$ is odd. Define a basis $\left\{ \psi_{i} \right\}_{i=1}^{N}$ by $\psi_{i} = \phi_{i}$ for $i \neq \frac{N+1}{2}$  For $i = \frac{N+1}{2}$ we define $\psi_{i}$ to be the "super-hat" function of formula (\ref{eqn:superhat}).  Define $V_{h} = \text{span}\left\{\psi_{0},\ldots,\psi_{N}\right\}$

\begin{equation}\label{eqn:superhat}
\psi_{(N+1)/2} = 
\left\{
     \begin{array}{cc}
     f(x) & x \in [0,x_{(N+1)/2}]\\
     g(x) & x\in [ x_{(N+1)/2},1]
     \end{array}
   \right.
\end{equation}



\begin{enumerate}
\item Show that the set of functions $\left\{\psi_{0},\ldots,\psi_{N}\right\}$ are linearly independent so that they define a basis of $V_{h}$

\item Write the discrete weak problem, also called the variational formulation, for the finite element method corresponding to solving the problem (\ref{eqn:bvp2}) in $V_{h}$.

\item If the solution $u_h(x) \in V_h$ is written $u_h = \Sigma_{i=1}^{N} \alpha_{i}\psi_{i}(x)$ then the discrete weak problem is a matrix problem $A\mathbf{\alpha} = \mathbf{f}$ where $A_{ij} = a(\psi_j,\psi_i)$ and $\mathbf{f}_i = \int_{0}^{1}  f \psi_i$.  Supposing that $k(x) = f(x) = 1$ determine the general form for the entries for the matrix $A$ and vector $\mathbf{f}$.

\item Explain in one sentence the effect of the large support set of the super-hat function on the matrix $A$. (A `support set' of a function $f(x)$ can be thought of as the set of points where f(x) is nonzero)



\end{enumerate}


%%%%%%%%%%%%%%%%%%%%%%%%%%%%%%%%%%%%%%%%%%%%%%%%%%%%%%%%%%%%

\ifthenelse{\boolean{showsols}}{
\begin{solution}

\end{solution}
}{}
