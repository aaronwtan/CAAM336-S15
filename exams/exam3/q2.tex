
In the last part of the semester we have seen the finite element method applied to the time-dependent heat equation; the most fundamental being the case of homogeneous Dirichlet boundary conditions as in \eqref{eqn:diric_heat}.

%\begin{align}
%\begin{array}{rcl}
%		 \multicolumn{3}{c}{  \pd{u}{t}{}   - \kappa \pd{u}{x}{2} = f(x,t)}\label{eqn:diric_wave}\\
%		 &&\\
%		   & u(0,x)  = \psi(x) & \pd{u}{t}{}(0,x) = \gamma(x) \\
%		   & u(t,0)  =   0 & u(t,l)  = 0 
%	\end{array}
%\end{align}
%
\begin{align}
\pd{u(x,t)}{t}{}  - \kappa \pd{u(x,t)}{x}{2} &= f(x,t) \label{eqn:diric_heat}\\
u(x,0) &= \psi(x) \\
u(0,t)  &=  u(1,t)  = 0 
\end{align}

We discretized equation using the finite element method and the space of linear hat functions $S_{0}^{N} = \text{span}\left\{\phi_{1}, \phi_{2}, \ldots, \phi_{N} \right\} $ where $\phi_{i}$ is the hat function corresponding to the internal mesh point $x_i$ on a uniformly spaced mesh of the interval $[0,1]$.  The discrete solution for \eqref{eqn:diric_heat} can be represented as 
\[
u_{h}(x,t) = \sum_{i=1}^{N} \alpha_{i}(t)\phi_{i}(x)
\]
and the initial condition can be approximated as 
\[
u_h(x,0) = \sum_{i=1}^{N} \psi(x_i) \phi_{i}(x).
\]
Let $\vec{\alpha}$, $\vec{\psi}$, and $\vec{b}$ denote coefficient vectors with $\vec{\alpha}_{i} = \alpha_{i}(t)$, $\vec{\psi}_{i} = \psi(x_i)$.

\begin{enumerate}
\item The finite element method applied to \eqref{eqn:diric_heat} results in a first-order system of ordinary differential equations for $\vec{\alpha}(t)$. Write down this system.  Specify the entries of the right-hand side vector $\vec{b}$ in addition to each matrix in the formulation.  That is, clearly indicate formulas for the components $\vec{b}_i$ and individual entries of every matrix involved in your answer. 
\vspace{1em}
\item Write down the numerical scheme for the Backward Euler method applied to the linear system of ordinary differential equations 
\[
M\td{\vec{y}}{t}{} = A\vec{y}(t) + \vec{g}(t),
\]
where $M$ is the mass matrix.  You should specify what the matrix $A$ is, and give a formula for $\vec{y}(t_{n+1})$ in terms of $\vec{y}(t_{n})$
\vspace{1em}
\item Explain how you would deal with \textit{inhomogeneous} (and possibly time-dependent) Dirichlet boundary conditions $u(0,t) = u_L(t)$, $u(1,t) = u_R(t)$.  
%If any additions are made to the function space $S_{0}^{N}$ draw a picture of the additional shape functions. 
If you need to introduce any additional functions to deal with these non-zero boundary conditions, please draw pictures or give expressions for them.
\emph{Hint: Recall that the problem can be split into two parts; what are these parts and how do you deal with them?}

%\item Taking $l=1$,  $\kappa = 1$, $f = 0$, $\psi(x) = 1-x$ in \eqref{eqn:diric_heat} and using two equally spaced internal nodes ($N=2$) gives the mass matrix $M$, stiffness matrix $K$ and  product $-M^{-1}K$ as
%
%\begin{align*}
%M = \left[ \begin{array}{cc} 1/3 & 1/20 \\ 1/20 & 1/3 \end{array} \right]  &&
%K = \left[ \begin{array}{cc} 6 & -3  \\ -3 & 6 \end{array} \right] &&
%-M^{-1}K =  -\frac{180}{391}\left[ \begin{array}{cc} 43 & -26  \\ -26 & 43 \end{array} \right]
%\end{align*}
%
%Using the fact that $-M^{-1}K$ has eigenvalues $\lambda_1 = 540/17$ and $\lambda_2 = 180/23$ with corresponding eigenvectors $v_{1} = \left[ \begin{array}{c} -1 \\ 1 \end{array} \right]$, $v_2 =  \left[ \begin{array}{c} 1 \\ 1 \end{array} \right]$ find the finite element discreet solution $u_{h}(t) = \alpha_{1}(t)\phi_{1}(x) + \alpha_2(t)\phi_{2}(x)$ to equation \eqref{eqn:diric_heat}. (Hint: you might find the fact that  $(-1/6)v_1 + (1/2)v_2 = \left[ \begin{array}{c} 2/3 \\ 1/3 \end{array} \right]$ useful.)

\end{enumerate}


%\begin{figure}[h] 
%\centering
%\includegraphics[width=0.2\textwidth]{superhat.png}
%\caption{The `super hat' function $\psi_{(N+1)/2}$}    \label{fig:zal2D_1r_2r}
%\end{figure}


%%%%%%%%%%%%%%%%%%%%%%%%%%%%%%%%%%%%%%%%%%%%%%%%%%%%%%%%%%%%

\ifthenelse{\boolean{showsols}}{
\begin{solution}

\end{solution}
}{}
