When studying new numerical techniques for modeling fluid phenomena periodic boundary conditions are often employed as they simulate an `infinite' space in a finite interval $\left[ -1, 1 \right]$.  As we saw in exam two, the operator $L$ defined by
\[
Lu = -\frac{\partial^{2}u}{\partial x^2}
\]
with periodic boundary conditions in space 
\[
u( x, -1) = u(x,1), \qquad \frac{\partial u}{\partial x}(x,-1) = \frac{\partial u}{\partial {x}}(x,1)
\]
has eigenvalues $\lambda_{0} = 0$ with eigenvector $\Psi_{0} = 1$ and $\lambda_{n} = \left( n\pi \right)^{2}$ corresponding to the pair of eigenvectors $\Psi_{1,n} = \sin(n\pi x)$ and $\Psi_{2,n} = \cos(n\pi x)$.  All of the eigenvectors are mutually orthogonal with respect to the inner product $\left(u,v\right) = \int_{-1}^{1} uv \, dx$.  In this problem we will consider a modified wave equation with periodic boundary conditions on $\left[ -1, 1 \right]$.  
\begin{align}
\label{eqn:modif_wave_damping}
&\pd{u}{t}{2}  - \pd{u}{x}{2} + \pi^2 u  = 0 \\
		  u(-1,t)  &=   u(1,t), \quad \pd{u}{x}{}(-1,t)  = \pd{u}{x}{}(1,t)\\
		  		   u(x,0) &= G(x), \quad \pd{u}{t}{}(x,0) = H(x) 
\end{align}

\begin{enumerate}
\item Show that the modified spatial operator $-\pd{u}{x}{2} + \pi^2 u$ with periodic boundary conditions has eigenvalue $\tilde{\lambda}_0 = \left(\lambda_{0} + \pi^2\right)$ with eigenvector $\tilde{\Psi}_0 = \Psi_0$ and that $\tilde{\Psi}_{1,n} = \Psi_{1,n}$, $\tilde{\Psi}_{2,n} = \Psi_{2,n}$ are eigenvectors which both have eigenvalue $\tilde{\lambda}_{j} = \left( \lambda_{j} + \pi^2  \right)$ .  

\item As a result of part (a) we write the solution to  the system \eqref{eqn:modif_wave_damping} as in equation \eqref{eqn:soln_exp} and expand the initial conditions as given in equations \eqref{eqn:initcond_one_expand} and \eqref{eqn:initcond_two_expand}.
\begin{align}
	u(x,t) &= \alpha_0(t) + \sum_{k=1}^{\infty} \left( \alpha_{k}(t)\tilde{\Psi}_{1,n}(x) + \beta_{k}(t)\tilde{\Psi}_{2,n}(x) \right)\label{eqn:soln_exp} \\
	G(x) & = \alpha_0(0) + \sum_{k=1}^{\infty} \left( \alpha_{k}(0){\Psi}_{1,n}(x) + \beta_{k}(0)\tilde{\Psi}_{2,n}(x) \right) \label{eqn:initcond_one_expand}\\
	H(x) & = \td{\alpha_0(0)}{t}{} +  \sum_{k=1}^{\infty} \left( \td{\alpha_{k}(0)}{t}{}{\Psi}_{1,n}(x) + \td{\beta_{k}(0)}{t}{}\tilde{\Psi}_{2,n}(x) \right)\label{eqn:initcond_two_expand}
\end{align}
Use these expansions in the modified damped wave equation \eqref{eqn:modif_wave_damping} and write down the series of initial boundary value problems (ordinary differential equations) determining the unknown coefficients $\alpha_0(t)$ and $\alpha_{k}(t)$, $\beta_{k}(t)$ for $k = 1,2,\ldots$. 

\item Solve the modified damped wave equation \eqref{eqn:modif_wave_damping} with  $G(x) = 1 + \sin(\pi x)$ and $H(x) = 1 + \cos(\pi x)$.  You will need the fact that the solution to the differential equation \eqref{eqn:ivp_soln} 
\begin{align}
\td{y}{t}{2} + \theta^{2}y & = 0 \label{eqn:ivp_soln}\\
y(0)  = A, \quad &\td{y}{t}{}(0)  = B \nonumber
\end{align}
is given by the equation 
\[
y(t) = A \cos(\theta t) + \frac{B}{\theta} \sin(\theta t).
\]



\end{enumerate}


%%%%%%%%%%%%%%%%%%%%%%%%%%%%%%%%%%%%%%%%%%%%%%%%%%%%%%%%%%%%

\ifthenelse{\boolean{showsols}}{
\begin{solution}

\end{solution}
}{}
